% Options for packages loaded elsewhere
\PassOptionsToPackage{unicode}{hyperref}
\PassOptionsToPackage{hyphens}{url}
\PassOptionsToPackage{dvipsnames,svgnames*,x11names*}{xcolor}
%
\documentclass[
]{article}
\usepackage{lmodern}
\usepackage{amssymb,amsmath}
\usepackage{ifxetex,ifluatex}
\ifnum 0\ifxetex 1\fi\ifluatex 1\fi=0 % if pdftex
  \usepackage[T1]{fontenc}
  \usepackage[utf8]{inputenc}
  \usepackage{textcomp} % provide euro and other symbols
\else % if luatex or xetex
  \usepackage{unicode-math}
  \defaultfontfeatures{Scale=MatchLowercase}
  \defaultfontfeatures[\rmfamily]{Ligatures=TeX,Scale=1}
\fi
% Use upquote if available, for straight quotes in verbatim environments
\IfFileExists{upquote.sty}{\usepackage{upquote}}{}
\IfFileExists{microtype.sty}{% use microtype if available
  \usepackage[]{microtype}
  \UseMicrotypeSet[protrusion]{basicmath} % disable protrusion for tt fonts
}{}
\makeatletter
\@ifundefined{KOMAClassName}{% if non-KOMA class
  \IfFileExists{parskip.sty}{%
    \usepackage{parskip}
  }{% else
    \setlength{\parindent}{0pt}
    \setlength{\parskip}{6pt plus 2pt minus 1pt}}
}{% if KOMA class
  \KOMAoptions{parskip=half}}
\makeatother
\usepackage{xcolor}
\IfFileExists{xurl.sty}{\usepackage{xurl}}{} % add URL line breaks if available
\IfFileExists{bookmark.sty}{\usepackage{bookmark}}{\usepackage{hyperref}}
\hypersetup{
  pdftitle={Further Analysis},
  pdfauthor={Kexing Ying},
  colorlinks=true,
  linkcolor=Maroon,
  filecolor=Maroon,
  citecolor=Blue,
  urlcolor=red,
  pdfcreator={LaTeX via pandoc}}
\urlstyle{same} % disable monospaced font for URLs
\usepackage[margin = 1.5in]{geometry}
\usepackage{graphicx}
\makeatletter
\def\maxwidth{\ifdim\Gin@nat@width>\linewidth\linewidth\else\Gin@nat@width\fi}
\def\maxheight{\ifdim\Gin@nat@height>\textheight\textheight\else\Gin@nat@height\fi}
\makeatother
% Scale images if necessary, so that they will not overflow the page
% margins by default, and it is still possible to overwrite the defaults
% using explicit options in \includegraphics[width, height, ...]{}
\setkeys{Gin}{width=\maxwidth,height=\maxheight,keepaspectratio}
% Set default figure placement to htbp
\makeatletter
\def\fps@figure{htbp}
\makeatother
\setlength{\emergencystretch}{3em} % prevent overfull lines
\providecommand{\tightlist}{%
  \setlength{\itemsep}{0pt}\setlength{\parskip}{0pt}}
\setcounter{secnumdepth}{5}
\usepackage{tikz}
\usepackage{amsthm}
\usepackage{mathtools}
\usepackage{lipsum}
\usepackage[ruled,vlined]{algorithm2e}
\theoremstyle{definition}
\newtheorem{theorem}{Theorem}
\newtheorem{prop}{Proposition}
\newtheorem{corollary}{Corollary}[section]
\newtheorem*{remark}{Remark}
\theoremstyle{definition}
\newtheorem{definition}{Definition}[section]
\newtheorem{lemma}{Lemma}[section]
\newcommand{\diag}{\mathop{\mathrm{diag}}}
\newcommand{\hess}{\mathop{\mathrm{Hess}}}

\title{Further Analysis}
\author{Kexing Ying}
\date{May 15, 2020}

\begin{document}
\maketitle

{
\hypersetup{linkcolor=}
\setcounter{tocdepth}{2}
\tableofcontents
}
\hypertarget{euclidean-spaces}{%
\section{Euclidean Spaces}\label{euclidean-spaces}}

For \(n \ge 1\), the \(n\)-dimensional \emph{Euclidean space} denoted by
\(\mathbb{R}^n\), is the set of ordered \(n\)-tuples
\(\mathbf{x} = (x_1, x_2, \cdots, x_n)\) for \(x_i \in \mathbb{R}\).
Recall that \(\mathbb{R}^n\) is a vector space over \(\mathbb{R}\), we
can use the usual vector space operations, i.e.~vector addition and
scalar multiplication. Furthermore, \(\mathbb{R}^n\) forms a inner
product space with the operation, \[
  \langle \cdot, \cdot \rangle : \mathbb{R}^n \times \mathbb{R}^n \to 
  \mathbb{R} : (\mathbf{x}, \mathbf{y}) \mapsto \sum_{i = 1}^n x_i y_i.
\] Thus, as a inner product space induces a normed vector space, we find
a natural norm defined for \(\mathbb{R}^n\) by, \[
  \| \cdot \| : \mathbb{R}^n \to \mathbb{R} : \mathbf{x} \mapsto 
  \sqrt{\langle \mathbf{x}, \mathbf{x} \rangle} = \sqrt{\sum_{i = 1}^n x_i^2}.
\] By manually checking, we find that this norm satisfy the norm axioms,
i.e.~it satisfy the \emph{triangle inequality}, is \emph{absolutely
scalable}, and \emph{positive definite} (In fact, we do not need the
norm to be non-negative as it can deduced from the other axioms).

\hypertarget{preliminary-concepts-in-mathbbrn}{%
\subsection{\texorpdfstring{Preliminary Concepts in
\(\mathbb{R}^n\)}{Preliminary Concepts in \textbackslash mathbb\{R\}\^{}n}}\label{preliminary-concepts-in-mathbbrn}}

Sequences in \(\mathbb{R}^n\) can be defined similarly to that of
\(\mathbb{R}\), and we carry over all notations in all suitable places.

\begin{definition}[Convergence in \(\mathbb{R}^n\)]
  We say a sequence \((\mathbf{x}_i)_{i = 1}^\infty \subseteq \mathbb{R}^n\) 
  converges to \(\mathbf{x} \in \mathbb{R}^n\) if and only if for all 
  \(\epsilon > 0\), there exists some \(N \in \mathbb{N}\) such that for all 
  \(i \ge N\), \(\|\mathbf{x}_i - \mathbf{x}\| < \epsilon\).
\end{definition}

\begin{prop}
  A sequence \((\mathbf{x}_i)_{i = 1}^\infty \in \mathbb{R}^n\) converges to 
  \(\mathbf{x} \in \mathbb{R}^n\) if and only if each component of 
  \(\mathbf{x}_i\) converges to the corresponding component of \(\mathbf{x}\).
\end{prop}

In the first dimension, we've considered the topology of \(\mathbb{R}\)
including the examination of open and closed sets. We extend this idea
for higher dimensions. The most basic examples we have of an open set
(or closed set for that matter) in \(\mathbb{R}\) are the open and
closed intervals respectively. This is extended in \(\mathbb{R}^n\) to
be sets of the form \[
  \prod_{i = 1}^n (a_i, b_i) := \{\mathbf{x} \mid a_i < \mathbf{x}_i < b_i, 
  \forall 1 \le i \le n\},
\] and similarly for closed intervals. However, while this is nice to
look at, it is not very useful. So for this reason, we again will extend
the notion of open and closed sets for \(\mathbb{R}^n\).

\begin{definition} [Open Ball]
  Let \(\mathbf{x} \in \mathbb{R}^n\) and \(r \in \mathbb{R}^+\), we define the 
  open ball of radius \(r\) about \(\mathbf{x}\) as the set 
  \[B_r(\mathbf{x}) := \{\mathbf{y} \in \mathbb{R}^n \mid 
  \| \mathbf{x} - \mathbf{y}\| < r\}\].
\end{definition}
\begin{definition} [Open]
  A set \(U \subseteq \mathbb{  R}^n\) is open in \(\mathbb{R}^n\) if and only if 
  for all \(\mathbf{x} \in U\), there is some \(r \in \mathbb{R}^+\) such that 
  \(B_r(\mathbf{x}) \subseteq U\).
\end{definition}
\begin{definition}[Closed]
  A set \(U \subseteq \mathbb{R}^n\) is closed if and only if its complement is 
  open.
\end{definition}

Straight away from the definition, we can see that every open ball is
open (see
\href{https://github.com/JasonKYi/learn_mspaces/blob/master/src/metric_spaces/basic.lean\#L215}{here}).
Furthermore, we find the union and intersection of two open sets is
open. In fact, the union and any collection of open sets is also open,
however, this is not necessarily true for closed sets.

\begin{definition} [Continuity at a Point]
  Let \(A \subseteq \mathbb{R}^n\) be an open set, and let 
  \(f : A \to \mathbb{R}^m\). We say \(f\) is continuous at \(p \in A\) if and 
  only if for all \(\epsilon > 0\) there exists \(\delta > 0\) such that for all 
  \(x \in A \cap B_\delta(p)\), \(\| f(x) - f(p) \| < \epsilon\).
\end{definition}

If the function \(f\) is continuous at every point of \(A\), then we say
\(f\) is continuous on \(A\).

\begin{definition}
  Let \(A \in \mathbb{R}^n\) be open in \(\mathbb{R}^n\) and let 
  \(f : A \to \mathbb{R}^m\). For \(p \in A\), we say that the limit of \(f\) as 
  \(\mathbf{x}\) tends to \(\mathbf{p}\) in \(A\) is equal to 
  \(\mathbf{q} \in \mathbb{R}^m\) if and only if for all \(\epsilon > 0\), there 
  exists \(\delta > 0\) such that for all \(x \in A \cap B_\delta(p), x \neq p\),
  \(\|f(x) - \mathbf{q}\| < \epsilon\).
\end{definition}

This is the same notion we used for continuity in the first dimension to
say that \(f\) is continuous at \(p\) if and only if
\(\lim_{x \to p} f(x) = q\).

\begin{prop}
  Let \(f : \mathbb{R}^n \to \mathbb{R}^m\) be a function. Then \(f\) is 
  continuous if and only if for all open subsets \(U\) of \(\mathbb{R}^m\), 
  \(f^{-1}(U)\) is open in \(\mathbb{R}^n\).
\end{prop}
\proof

See
\href{https://github.com/JasonKYi/learn_mspaces/blob/master/src/metric_spaces/basic.lean\#L257}{here}
for the proof. \qed

\hypertarget{derivative-of-maps-in-euclidean-spaces}{%
\subsection{Derivative of Maps in Euclidean
Spaces}\label{derivative-of-maps-in-euclidean-spaces}}

Let \(\Omega\) be a open in \(\mathbb{R}^n\), and
\(f : \Omega \to \mathbb{R}^m\) be a ``nice behaving map''. We poses the
question on how we should define the notion of derivatives for this
mapping at some point \(p \in \Omega\). We recall that the derivative at
a point \(p\) in the first dimension is defined to be
\[f'(p) = \lim_{x \to p}\frac{f(x) - f(p)}{x - p}.\] While we see that
this equation makes no sense if we simply generalise this equation to
higher dimensions, we see the following result.

\begin{lemma}
  Let \(f : S \to \mathbb{R}\) where \(S \subseteq \mathbb{R}\), then \(f\) is 
  differentiable at some \(p \in S\) if and only if there exists some 
  \(\lambda \in \mathbb{R}\) such that 
  \[\lim_{x \to p} \left|\frac{f(x) - A_\lambda(x)}{x - p}\right| = 0,\]
  where \(A_\lambda(x) = \lambda(x - p) + f(p)\).
\end{lemma}
\proof

Follows from algebraic manipulation. \qed

With this, we can conclude that \(f(x) - A_\lambda(x)\) tends to zero
faster than \(x - p\). We will generalise this result to higher
dimensions.

We may rewrite
\(A_\lambda(x) = \lambda(x - p) + f(p) = \lambda x + (f(p) + \lambda p)\),
so, we see that \(\lambda\) is the translation of a linear map
\(\lambda x\),
i.e.~\(A_\lambda = (x \mapsto x + (f(p) + \lambda p)) \circ (x \mapsto \lambda x)\).
We can easily generalise such maps to higher dimensions and we call such
maps \emph{affine maps}.

\begin{definition}[Differentiable Functions in \(\mathbb{R}^n\)]
  Recall the definition of linear maps for general vector spaces which we will 
  use in the context of Euclidean spaces. Let \(L(\mathbb{R}^n; \mathbb{R}^m)\) 
  denote the set of linear maps from \(\mathbb{R}^n\) to \(\mathbb{R}^m\), 
  \(\Omega \subseteq \mathbb{R}^n\) be open, and \(f : \Omega \to \mathbb{R}^m\) 
  be a function. Then we say \(f\) is differentiable at some point 
  \(p \in \Omega\) if and only if there exists some 
  \(\Lambda \in L(\mathbb{R}^n; \mathbb{R}^m)\), such that 
  \[ \lim_{x \to p} \frac{\| f(x) - (\Lambda(x - p) + f(p))\|}{\|x - p\|} = 0.
  \]
  If this is true, we write \(Df(p) = \Lambda\) and call \(\Lambda\) the 
  derivative of \(f\) at \(p\).
\end{definition}
\begin{remark}
  Some book refers to \(Df(p)\) as the total derivative of \(f\) at \(p\).
\end{remark}

It is often useful to express the derivative criterion as
\[\lim_{h \to 0} \frac{\|f(p + h) - f(p) - \Lambda(h)\|}{\| h \|} = 0.\]

\begin{prop}
  Let \(f_i : (a, b) \to \mathbb{R}\) be differentiable for all \(i\), then the 
  function, \(f : (a, b) \to \mathbb{R}^m : x \mapsto (f_i(x))_{i = 1}^m\) is 
  differentiable for all \(p \in (a, b)\).
\end{prop}
\proof

Let the Jacobian \(\Lambda = \text{diag}(\lambda_i)\) where
\(\lambda_i\) is the derivative of \(f_i\) at \(p\). Then I claim,
\(Df(p) = \Lambda p\).

Consider
\[\lim_{h \to 0} \frac{\|f(p + h) - f(p) - \Lambda h\|}{\| h \|} = 
  \lim_{h \to 0} \frac{\sqrt{\sum_{i = 0}^m \left|f_i(p + h) - f_i(p) - \lambda_i h\right|}}
  {\| h \|}.\] However, for all \(i\),
\(\| h \| \ge \left| h_i \right|\), so \[ \lim_{h \to 0}
  \frac{\sqrt{\sum_{i = 0}^m \left|f_i(p + h) - f_i(p) - \lambda_i h\right|}}
  {\| h \|} \le \lim_{h \to 0} \sum_{i = 0}^m
  \sqrt{\frac{\left|f_i(p + h) - f_i(p) - \lambda_i h\right|}
  {\left| h_i \right|}} = 0\] \qed

A lot of results from the first dimension generalises easily to higher
dimensions. Similar to that of the first dimension, the chain rule in
general Euclidean spaces states,

\begin{theorem}
  Let \(\Omega \subseteq \mathbb{R}^n\) and \(\Omega' \subseteq \mathbb{R}^m\) 
  be open sets with \(g : \Omega \to \Omega'\) be differentiable at \(p \in \Omega\) 
  and \(f : \Omega' \to \mathbb{R}^l\) be derivatives at \(g(p)\). Then 
  \(h = f \circ g\) is differentiable at \(p\) with derivative 
  \[Dh(p) = Df(g(p)) \circ Dg(p).\] 
\end{theorem}
\proof

Similar to the proof of the Chain rule in the first dimension using
algebraic manipulation. \qed

Omitted many examples here, check official lecture notes for these
examples.

\hypertarget{directional-derivatives}{%
\subsection{Directional Derivatives}\label{directional-derivatives}}

Although the definition of derivative in dimension one and higher is
similar, it is different in that we verify whether a linear map is the
total derivative at a point instead of computing the limit of some
equation. This is difficult as often times, it is not easy to guess what
the derivative of a function is. Thus, it is useful to somehow identify
candidate linear maps for the derivative.

Assume \(\Omega \subseteq \mathbb{R}^n\) is open and
\(f : \Omega \to \mathbb{R}^m\) is a differentiable function at some
\(p \in \Omega\). Let \(v \in \mathbb{R}^n\) be a unit vector. We would
like to identify \(Df(p)[v] \in \mathbb{R}^m\).

By the definition of derivatives in higher dimensions,
\[\lim_{h \to 0} \frac{\|f(p + h) - f(p) - \Lambda h\|}{\| h \|} = 0. \]
So, by letting \(t \in \mathbb{R}\), we have \begin{align*}
  0 & = \lim_{t \to 0} \frac{\|f(p + tv) - f(p) - \Lambda (tv)\|}{\| tv \|}\\
    & = \lim_{t \to 0} \frac{\|f(p + tv) - f(p) - t\Lambda v\|}{\left| t \right|}\\
    & = \lim_{t \to 0} \frac{\|f(p + tv) - f(p)\|}{\left|t\right|} - 
        \Lambda v,\\
\end{align*} So,
\(\lim_{t \to 0} \|f(p + tv) - f(p)\| / \left|t\right| = \Lambda v\).
Thus, by finding the limits of the above equation for each basis vector
\(v \in B\), we find the Jacobian \([\Lambda]_B\).

\begin{remark}
  For notation, we denote the limit as 
  \(\lim_{t \to 0} \|f(p + tv) - f(p)\| / \left|t\right|\) as 
  \(\partial f / \partial v \mid_p\) and we call it the directional derivative 
  of \(f\) in the direction of \(v\) at \(p\). We will normally consider the 
  directional derivatives in the direction of the standard basis and we call 
  them the partial derivatives of \(f\) at \(p\).
\end{remark}

\begin{theorem}
  Let \(\Omega \subseteq \mathbb{R}^n\) be open, and 
  \(f : \Omega \to \mathbb{R}^m : x \mapsto [f_i(x)]\) for 
  \(i \in \{1, \cdots m\}\) is differentiable at some \(p \in \Omega\). Then 
  the Jacobian of \(f\) at \(p\) is \([\partial f_i / \partial e_j \mid_p]_{i, j}\).
\end{theorem} 
\proof

We recall that the Jacobian is simply the matrix form of the linear map
that is the derivative. So, for all \(v \in \mathbb{R}^n\), we have
\(J v = Df(p)(v)\). As, \(v \in \mathbb{R}^n\), it can be represented as
a sum of the standard basis, that is there exists
\(v_i \in \mathbb{R}\), \(v = \sum_{i = 1}^n v_i e_i\), so,
\(Df(p)(v) = Df(p)(\sum_{i = 1}^n v_i e_i)  = \sum_{i = 1}^n v_i Df(p)(e_i)  = \sum_{i = 1}^n v_i \partial f / \partial e_i \mid_p  = [\sum_{i = 1}^n v_i \partial f_j / \partial e_i \mid_p]_j  = [\partial f_i / \partial e_j \mid_p]_{i, j} v\).
(We used the fact that
\([\partial f / \partial e_i]_j  = \partial f_j / \partial e_i\).) \qed

\begin{remark}
  We note that the reverse is not true, that is the existence of partial 
  derivatives does not imply differentiability. A counter example of this is 
  \(f : \mathbb{R}^2 \to \mathbb{R} : (x, y) \mapsto \text{if } x = y = 0, \text{ then } 0,
  \text{ else}, \frac{xy}{\sqrt{x^2 + y^2}}\).
\end{remark}

\begin{theorem}\label{strong}
  Let \(\Omega \subseteq \mathbb{R}^n\) is open, and \(f : \Omega \to \mathbb{R}\) 
  be a function. Suppose that the partial derivatives of \(f\), \(D_i f(x)\) exists 
  for all \(i = 1 \cdots n\) exists at all \(x \in \Omega\).  Furthermore, 
  if the map \(x \mapsto D_i f(x)\) is continuous for all \(i\) at some point 
  \(p\). Then \(f\) is differentiable at \(p\).
\end{theorem}

\hypertarget{higher-derivatives}{%
\subsection{Higher Derivatives}\label{higher-derivatives}}

Similar to the first dimension, we would like to think about how to
differentiate more than once.

Let \(\Omega \subseteq \mathbb{R}^n\) is open, and
\(f : \Omega \to \mathbb{R}^m\) be differentiable everywhere on
\(\Omega\). Then we may consider the map
\[Df : \Omega \to \mathcal{L}(\mathbb{R}^n; \mathbb{R}^m) : p \mapsto Df(p).\]
As there is a trivial isomorphism between
\(\mathcal{L}(\mathbb{R}^n; \mathbb{R}^m)\) and the matrices of
dimension \(m \times n\), we can represent every linear map from
\(\mathbb{R}^n\) to \(\mathbb{R}^m\) as an element of
\(\mathbb{R}^{m \times n}\). Thus, \(Df\) can be represented as a map
from \(\Omega\) to \(\mathbb{R}^{m \times n}\). So, we may ask if \(Df\)
is continuous at some \(p\) and furthermore, we can ask if \(Df\) is
differentiable at some \(p \in \Omega\). If this is the case, we have
the second derivative
\[DDf : \Omega \to \mathcal{L}(\mathbb{R}^n; \mathbb{R}^{m \times n}).\]

\begin{definition}[Second Derivative]
  Let \(\Omega \subseteq \mathbb{R}^n\) be open and \(f : \Omega \to \mathbb{R}^m\) 
  be differentiable everywhere on \(\Omega\) with derivative 
  \(Df : \Omega \to \mathcal{L}(\mathbb{R}^n; \mathbb{R}^m)\). Then the second 
  derivative of \(f\) at some \(p \in \Omega\) is the linear map 
  \(\Lambda \in \mathcal{L}(\mathbb{R}^n; \mathbb{R}^{m \times n})\) such that 
  \[\lim_{x \to p} \frac{\| Df(x) - Df(p) - \Lambda(x - p) \|}{\| x - p\|} = 0.\]
\end{definition}

Thus, with this definition, we can easily extend the notion of
derivatives any number of times to get the \(k\)-th derivative of a
function. However, this is formally difficult and requires the notion of
multilinear maps. Luckily, instead of working with this difficult
definition whenever we would like to work with higher derivatives, we
can instead look at whether the \(k\)-th derivative exists and whether
or not it is continuous by theorem \ref{strong}.

Now that we have established the notion of higher derivatives we would
like to ask how higher partial derivatives interacts. That is, when does
\(D_i D_j f(p) = D_j D_i f(p)\)?

\begin{theorem}[Schwartz' Theorem]
  Let \(\Omega \subseteq \mathbb{R}^n\) be open and \(f : \Omega \to \mathbb{R}\) 
  be differentiable at every \(p \in \Omega\). Suppose that for some 
  \(i, j \in \{1, \cdots, n\}\) the second partial derivatives \(D_i D_j f\) and 
  \(D_j D_i f\) exists and is continuous for all \(p \in \Omega\), then 
  \[D_i D_j f(p) = D_j D_i f(p)\]
  for all \(p \in \Omega\).
\end{theorem}

If \(f : \Omega \to \mathbb{R}\), we call the matrix of second partial
derivatives of \(f\) at some point \(p\) the \textbf{Hessian} of \(f\)
at \(p\) and we write
\(\mathop{\mathrm{Hess}}f(p) = [D_i D_j f(p)]_{i, j = 1, \cdots, n}\).
Given that the hypothesis of Schwartz's theorem holds, we find that
\([\mathop{\mathrm{Hess}}f(p)]_{i, j} = [\mathop{\mathrm{Hess}}f(p)]_{j, i}\)
so the Hessian is symmetric.

\hypertarget{taylors-theorem}{%
\subsection{Taylor's Theorem}\label{taylors-theorem}}

We recall that the first derivative of a map
\(f : \Omega \to \mathbb{R}^m\) allows us to find an affine map at some
point \(p \in \Omega\) such that the error decreases faster than that of
\(\| x - p\|\). The existence of higher derivatives allows us to obtain
better estimates with error decreasing even faster.

Let us first introduce some notations. We define a multi-index
\(\alpha\) an element of \(\mathbb{N}^n\) and we write
\(\left| \alpha \right| = \sum \alpha_i\). Furthermore, given some
function \(f : \Omega \to \mathbb{R}^n\), we write
\[D^\alpha f := (D_1)^{\alpha_1}\cdots(D_n)^{\alpha_n} f,\] and given
\(h \in \mathbb{R}^n\), we write
\[h^\alpha := h_1 ^ {\alpha_1} \cdots h_n^{\alpha_n}.\] Lastly, we write
\(\alpha! := \alpha_1! \cdots \alpha_n!\).

With that, we can state the Taylor's theorem.

\begin{theorem}[Taylor's Theorem]
  Suppose \(p \in \mathbb{R}^n\) and \(f : B_r(p) \to \mathbb{R}\) is \(k\)-times
  differentiable on \(B_r(p)\) for some \(r > 0\) and \(k \ge 1\). Then for any 
  \(h \in \mathbb{R}^n\) with \(\|h\| < r\), we have,
  \[
    f(p + h) = \sum_{\substack{\alpha \in \mathbb{N}^n,\\ \left|\alpha\right| \le k - 1}} 
      D^\alpha f(p) \frac{h^\alpha}{\alpha!} + R_k(p, h),
  \]
  where there exists some \(x \in \mathbb{R}^n\) with \(0 < \| x - p\| < \| h \|\)
  such that 
  \[
    R_k(p, h) = \sum_{\substack{\alpha \in \mathbb{N}^n \\ \left| \alpha \right| = k}} 
      D^\alpha f(x)\frac{h^\alpha}{\alpha!}.
  \]
\end{theorem}
\proof

See one dimensional version from year one. (Essentially boils down to
finding the Taylor expansion of the restriction of \(f\) on
\(\{p + th \mid t \in \mathbb{R}\} \cap B_r(p)\) which is isomorphic to
a open set in the real line.) \qed

\hypertarget{inverse-and-implicit-function-theorem}{%
\subsection{Inverse and Implicit Function
Theorem}\label{inverse-and-implicit-function-theorem}}

The inverse and implicit function theorem are two important theorems and
we shall look at them in this section.

From last year, we looked at the inverse function theorem in the first
dimension. Let \(f : (a, b) \to \mathbb{R}\) be continuously
differentiable, and suppose there exists \(p \in (a, b)\) such that
\(f'(p) \neq 0\). Then there exists a neighbourhood \(I\) around \(p\)
such that \(f \mid_I : I \to f(I)\) is bijective and thus has a inverse
on \(I\), \(f^{-1}\) that is differentiable and
\[(f^{-1})'(y)) = \frac{1}{f'(f^{-1}(y))}.\]

This theorem can be generalised into higher dimensions.

\begin{theorem}[\(C^1\) Inverse Function Theorem]
  Let \(\Omega \subseteq \mathbb{R}^n\) be open and \(f : \Omega \to \mathbb{R}^m\) 
  be continuously differentiable on \(\Omega\), and there exists some \(q \in \Omega\) 
  such that \(Df(q) \in \mathcal{L}(\mathbb{R}^n; \mathbb{R}^m)\) is invertible.
  Then, there exist open neighbourhoods around \(q\) and \(f(q)\), namely \(U, V\)
  respectively, such that,
  \begin{itemize}
    \item \(f : U \to V\) is a bijection;
    \item \(f^{-1} : V \to U\) is continuously differentiable;
    \item for all \(v \in V\), \(Df^{-1}(v) = (Df(f^{-1}(v)))^{-1}\).
  \end{itemize}
\end{theorem}

\begin{lemma}[Contraction Mapping Theorem]
  Let \(X\) be a complete metric space and let \(\phi : X \to X\) be a contraction 
  of \(X\). Then there exists an unique \(x\) such that \(\phi(x) = x\).
\end{lemma}

\begin{remark}
  We shall examine exactly what this theorem states in the later sections on 
  metric spaces and topology.
\end{remark}

\proof (Part 1 of the \(C^1\) Inverse Function Theorem). We denote
\(\|A\| = \sup_{\|x\| \le 1} \|Ax\|\) within this proof. Suppose
\(q \in \Omega\) and \(Df(q) = A\) is invertible, then let us define
\(\epsilon := 1 / \|Df(q)\|\). As \(f\) is continuously differentiable,
there exists some open neighbourhood of \(q\) -- \(U\) such that for all
\(x \in U\) \[\|Df(x) - A\| < \epsilon,\] (simply choose \(U\) to have
diameter smaller than 1). Now, for all \(y \in \mathbb{R}^m\), we define
\[\phi_y(x) = x + A^{-1}(y - f(x)).\] It is easy to see that \(\phi_y\)
is differentiable with derivative
\(D\phi_y(x) = I - A^{-1}Df(x) = A^{-1}(A - Df(x)),\) so
\(\|D\phi_y(x)\| = \|A^{-1}\| \|(A - Df(x))\| < \|A^{-1}\|\epsilon = 1\).
By the mean value theorem,
\(\| \phi_y(x_1) - \phi_y(x_2) \| \le \| x_1 - x_2 \|\), that is
\(phi_y\) is a contraction on \(B\), and hence has a unique fixed point.
Now as \(f(x) = y\) if and only if \(x\) is a fixed point of \(\phi_y\),
we are done. \qed

It is in general not easy to find the inverse of a function in the
higher dimensions, so the inverse function theorem can help us obtain
some properties about the inverse that is otherwise difficult or
unobtainable.

The inverse function theorem can be used to show existence and
uniqueness of solutions of non-linear system of equations. Given
\(f_i(x_1, \cdots, x_n) = y_i\) for \(i = 1,\cdots , n\), we can define
\(F(\mathbf{x}) = (f_i(\mathbf{x}))^T\). Then, by looking at some open
neighbourhood containing \(\mathbf{y}\), it might be possible to
determine \(F^{-1}(\mathbf{y})\).

Let \(\Omega, \Omega' \subseteq \mathbb{R}^n\) be open. Then, we say
\(f : \Omega \to \Omega'\) is a \(C^1\)-diffeomorphism is

\begin{itemize}
  \item \(f : \Omega \to \Omega'\) is a bijection;
  \item \(f : \Omega \to \Omega'\) is continuously differentiable;
  \item for all \(x \in \Omega\), \(Df(x)\) is invertible.
\end{itemize}

\begin{remark}
  In fact, the set of all \(C^1\)-diffeomorphisms from some open 
  \(\Omega \subseteq \mathbb{R}^n\) to itself forms a group under composition.
\end{remark}

\begin{theorem}[Implicit Function Theorem -- Simple ver.]
  Let \(\Omega \subseteq \mathbb{R}^2\) be open and \(f : \Omega \to \mathbb{R}\) 
  is continuously differentiable, moreover, suppose there exists 
  \(q = (a, b) \in \Omega\) such that \(f(a, b) = 0\) and \(D_2 f(a, b) \neq 0\). 
  Then, there exists open \(A, B \subseteq \mathbb{R}\) and \(g : A \to B\)
   such that \(a \in A\), \(b \in B\) and \((x, y) \in A \times B\) satisfies 
   \(f(x, y) = 0\) if and only if \(y = g(x)\). Furthermore, \(g\) is 
   continuously differentiable.
\end{theorem}

\proof

Wlog. We assume \(D_2f(p) > 0\), then as \(f\) is continuously
differentiable, \(D_2f(p)\) is continuous and thus, there exists some
open neighbourhoods around \(a\) and \(b\) --
\(A, B = (a - \delta_a, a + \delta_a), (b - \delta_b, b + \delta_b)\)
respectively, such that for all \(u \in A \times B\), \(D_2f(u) > 0\)
(this can be obtained by drawing the a square insider the open ball).
Now, suppose we define \[h : B \to \mathbb{R} : y \mapsto f(a, y).\] As
\(h'(y) = D_2f(a, y) > 0\), we see that \(h\) is strictly increasing.
Furthermore, as \(h(b) = f(a, b) = 0\), we have
\(h(b - \delta_b / 2) < h(b) = 0\), and similarly,
\(h(b + \delta_b / 2) > 0\). Thus, there exists some
\(\delta_a > \delta' > 0\) such that \(f(x, b - \delta_b / 2) < 0\) and
\(f(x, b + \delta_b / 2) > 0\) for all
\(x \in (a - \delta', a + \delta')\). Now, by the intermediate value
theorem, for all \(x \in (a - \delta', a + \delta')\) there exists
(uniquely) some \(y_x \in (b - \delta_b / 2, b + \delta_b / 2)\) such
that \(f(x, y_x) = 0\) (unique as \(D_2f(x, y) > 0\)). Thus, we can
define
\[g : (a - \delta', a + \delta') \to (b - \delta_b / 2, b + \delta_b / 2) : x \mapsto y_x.\]
We see straight away that \(g\) is continuously differentiable as \(f\)
is. So we are done. \qed

There is a more general version of this theorem applying to arbitrary
dimensions.

\begin{theorem}[Implicit Function Theorem]
  Let \(\Omega \subseteq \mathbb{R}^n, \Omega' \subseteq \mathbb{R}^m\) be open, 
  and \(f : \Omega \times \Omega' \to \mathbb{R}^m\) be continuously differentiable 
  on \(\Omega \times \Omega'\). Suppose there exists 
  \(p = (a, b) \Omega \times \Omega'\) such that \(f(p) = 0\) and 
  \(D_{n + j} f^i(p)\) is invertible for all \(1 \le i, j \le m\). Then there 
  are \(A \subseteq \Omega, B \subseteq \Omega'\) with \(a \in A, b \in B\) 
  such that there exists a map \(g : A \to B\) in which, 
  \(f(x, y) = 0\) if and only if \(y = g(x)\) for some \(x \in A\).
  Furthermore, \(g\) is continuously differentiable.
\end{theorem}

\hypertarget{metric-and-topological-spaces}{%
\section{Metric and Topological
Spaces}\label{metric-and-topological-spaces}}

\hypertarget{metric-spaces}{%
\subsection{Metric Spaces}\label{metric-spaces}}

\begin{definition}[Metric]
  Let \(X\) be some set. Then, a metric \(d\) on \(X\) is a function from 
  \(X^2\) to \(\mathbb{R}\) such that, 
  \begin{itemize}
    \item for all \(x, y \in X\), \(d(x, y) \ge 0\) and \(d(x, y) = 0\) if and 
      only if \(x = y\);
    \item for all \(x, y \in X\), \(d(x, y) = d(y, x)\);
    \item for all \(x, y, z \in X\), \(d(x, y) \le d(x, z) + d(z, y)\).
  \end{itemize}
\end{definition}

We call the ordered pair \((X, d)\) where \(d\) is a metric on \(X\), a
metric space. In general, for short hand, we simply refer to the metric
space \((X, d)\) as \(X\).

\begin{definition}[Subspace]
  Let \((X, d)\) be a metric space. Then, for all \(Y \subseteq X\), \((Y, d\mid_Y)\) 
  forms a metric space where 
  \[d\mid_y : Y \times Y \to \mathbb{R} : (x, y) \mapsto d(x, y).\]
  We call this metric space a metric subspace of \((X, d)\).
\end{definition}

Up to now, we have seen examples of metric spaces in terms of normed
vector spaces. It is not difficult to see that all normed vector spaces
induces a metric space (where \(d(u, v) := \|u - v\|\)) while the
reverse is in general not true. We shall formally define the notion of
norm here.

\begin{definition}[Norm]
  Let \(V\) be some vector space over a field \(\mathbb{R}\). Then a function 
  \(\| \cdot \| : V \to \mathbb{R}\) is a norm on \(V\) if 
  \begin{itemize}
    \item for all \(v \in V\), \(\|v\| \ge 0\) and \(\|v\| = 0\) if and only if 
      \(v = 0\);
    \item for all \(v \in V\), and \(\lambda \in \mathbb{R}\), 
      \(\|\lambda v\| = \left| \lambda \right| \|v\|\);
    \item for all \(u, v \in V\), \(\|u + v\| \le \|u\| + \|v\|\).
  \end{itemize}
\end{definition}

We see that this definition can be extending to vector spaces over
arbitrary fields if and only if there is a notion of absolute value on
that field.

Similar to metric spaces, we call the ordered pair \((V, \|\cdot\|)\)
where \(\|\cdot\|\) is a norm on \(V\) a normed vector space.

\begin{definition}[\(n\)-Norm]
  Given \(\mathbb{R}^n\), the \(n\)-norm \(\| \cdot \|_n\) on \(\mathbb{R}^n\) is 
  the norm such that for all \(v \in \mathbb{R}^n\), 
  \(\|v\|_n = \left(\sum v_i^n \right)^{\frac{1}{n}}\). We also define the 
  \(\infty\)-norm as \(\|v\|_\infty = \max\{\left|v_i\right|\}\).
\end{definition}

As one might expect, \(d_n\) is the metric induced by the \(n\)-norm and
similarly, \(d_\infty\) is the metric induced by the \(\infty\)-norm.

\begin{definition}[Open Ball]
  Let \((X, d)\) be a metric space and suppose \(x \in X\). Then a ball 
  of radius \(\epsilon > 0\) at \(x\) is the set 
  \(B_\epsilon(x) = \{y \in X \mid d(x, y) < \epsilon\}.\)
\end{definition}

This is also called the \(\epsilon\)-ball about \(x\) or the
\(\epsilon\)-neighbourhood of \(x\).

\begin{definition}[Open]
  Let \((X, d)\) be a metric space and let \(U \subseteq X\). We say that \(U\) 
  is open on \((X, d)\) if and only if for all \(u \in U\), there exists some 
  \(\delta > 0\) such that \(B_\delta(u) \subseteq U\).
\end{definition}

\begin{lemma}
  For all metric spaces \((X, d)\), \(\varnothing\) and \(X\) are open.
\end{lemma}
\proof

Follows from definition. \qed

\begin{lemma}
  Let \((X, d)\) be a metric space and let \(\mathcal{C}\) be a collection of 
  open sets, then \(\bigcup \mathcal{C}\) is open. On the other hand, if 
  \(\mathcal{C}\) is finite, then \(\bigcap\mathcal{C}\) is also open.
\end{lemma}
\proof

Easy. \qed

\begin{definition}[Topologically Equivalent]
  Let \(d_1, d_2\) be two metrics on \(X\). We call \(d_1\) and \(d_2\) 
  topologically equivalent if and only if for all \(U \subset X\), \(U\) is open
  with respect to \(d_1\) if and only if is it open with respect to \(d_2\).
\end{definition}

With this definition, we see that the family of
\(\{d_n \mid n \in \bar{\mathbb{N}}\}\) is all topologically equivalent
to one another.

There is another notion of equivalence for metrics called the Lipschitz
equivalence.

\begin{definition}[Lipschitz Equivalence]
  Let \(d_1, d_2\) be two metrics on \(X\). We call \(d_1\) and \(d_2\) 
  Lipschitz equivalent if and only if there exists \(M_0, M_1 \in \mathbb{R}^+\) 
  such that for all \(x, y \in X\),
  \[M_0 d_1(x, y) \le d_2(x, y) \le M_1 d_1(x, y).\] 
\end{definition}

It is not difficult to see that Lipschitz equivalence is a stronger
notion of equivalence since if two metrics are Lipschitz equivalent then
they are topologically equivalent. This can be proved by choosing
\(\delta' = M_0\delta\).

Sometimes, it is also useful to induce a new metric using an existing
one by pushing it over some function. It turns out, this is possible
over continuous and injective functions.

\begin{theorem}\label{induce_metric}
  Let \((X, d_X)\) be a metric space and let \(f : Y \to X\) be continuous and 
  injective. Then 
  \[d_Y : Y \times Y \to \mathbb{R} : (y_1, y_2) \mapsto d_X(f(y_1), f(y_2))\]
  is a metric on \(Y\).
\end{theorem}
\proof

\(d_Y\) is trivially non-negative and symmetric so let us consider
whenever \(d_Y = 0\). Let \(y_1, y_2 \in Y\) such that
\(d_Y(y_1, y_2) = 0\), then by definition, \(f(y_1) = f(y_2)\), and so,
as \(f\) is injective, \(y_1 = y_2\).

Thus, it suffices to prove the triangle inequality for \(d_Y\). let
\(y_1, y_2, y_3 \in Y\), then \begin{align*}
    d_Y(y_1, y_2) & = d_X(f(y_1), f(y_2)) \\
      & \le d_X(f(y_1), f(y_3)) + d_X(f(y_3), f(y_2))\\
      & = d_Y(y_1, y_3) + d_Y(y_3, y_2).
  \end{align*} \qed

\hypertarget{basic-notions-in-metric-spaces}{%
\subsection{Basic Notions in Metric
Spaces}\label{basic-notions-in-metric-spaces}}

In this section, we will establish some basic notions regarding metric
spaces.

\begin{definition}[Convergence]
  Let \((X, d)\) be a metric space and suppose \((x_n)_{n = 1}^\infty\) is a 
  sequence in \(X\). Then, we say \(x_n\) converges to some \(x \in X\) is and 
  only if for all \(\epsilon > 0\), there exists some \(N \in \mathbb{N}\) such 
  that for all \(n \ge N\), \(d(x, x_n) < \epsilon\).
\end{definition}

As this definition of convergence is essentially the same as the one we
looked at last year, it follows some similar properties.

\begin{lemma}
  Let \((X, d)\) be a metric space and suppose \((x_n)_{n = 1}^\infty\) is a 
  sequence in \(X\). Then, if \(x_n\) converges in \(X\), the its limit is unique.
\end{lemma}
\proof

Suppose \(x_n \to a\) and \(x_n \to b\) for some \(a, b \in X\). Then
for all \(\epsilon\), there exists \(N \in \mathbb{N}\) such that for
all \(n \ge N\), \(d(x_n, a), d(x_n, b) < \epsilon / 2\) and thus, by
triangle inequality,
\(d(a, b) < \epsilon / 2 + \epsilon / 2 = \epsilon\). As \(\epsilon\) is
an arbitrary positive number, \(d(a, b) = 0\) and so \(a = b\). \qed

\begin{definition}[Closed Set]
  Let \((X, d)\) be a metric space and \(U \subseteq X\). Then we say \(U\) is 
  closed in \(X\) is and only if for any convergent sequence \((x_n)_{n \ge 1}\) 
  in \(U\) converges in \(U\).
\end{definition}

We see that this is the same definition we had for closed sets in
\(\mathbb{R}\) and similarly to what we had found last year, a set is
closed if and only if its complement is open.

\begin{prop}\label{closed_iff}
  Let \((X, d)\) be a metric space and let \(U \subseteq X\). Then \(U\) is 
  closed in \(X\) if and only if \(U^c\) is open.
\end{prop}
\proof

\((\implies)\) We prove the contrapositive. Suppose \(U^c\) is not open,
then there exists \(x \in U^c\) such that for all \(\epsilon > 0\),
\(B_\epsilon(x) \not\subseteq U^c\). So, for all \(n \in \mathbb{N}\),
\(B_{1 / n}(x) \cap U \neq \varnothing\). So, by defining the sequence
\((x_n)_{n \ge 1}\) such that \(x_n \in B_{1 / n}(x) \cap U\) we have a
sequence in \(U\). Now, we see that \(x_n \to x\) as for all
\(\epsilon\), we can choose \(N \ge 1 / \epsilon\) so, for all
\(n \ge N\),
\(x_n \in B_{1 / n}(x) \subseteq B_{1 / N}(x) \subseteq B_\epsilon (x)\).
But \(x \not\in U\) and thus, \(U\) is not closed.

\((\impliedby)\) The reverse is by similar argument. Suppose there
exists a sequence \((x_n)_{n \ge 1}\) in \(U\) such that \(x_n\)
converges to \(x \in U^c\). Then, if \(U^c\) is open, there exists some
\(\epsilon > 0\) such that \(B_\epsilon (x) \subseteq U^c\). But now as
\(x_n \to x\), there exists \(N \in \mathbb{N}\) such that for all
\(n \ge N\), \(x_n \in B_\epsilon(x)\) implying
\(B_\epsilon(x) \not\subseteq U^c\) since \(x_n \not\in U^c\) \# \qed

We might sometimes find the above condition as the definition of closed
sets instead, that is, a set is closed if and only if its complement is
open. However, this does not matter by the above proposition. The
question of which definition to use is rather pedagogical since the
definition we presented resembles more of that we had defined for
Euclidean spaces while the latter definition resembles the topological
definition which we shall examine later.

\begin{definition}[Interior, Isolated, Limit and Boundary Point]
  Let \((X, d)\) be a metric space and \(V \subseteq X\). Then
  \begin{itemize}
    \item a point \(x \in V\) is an interior point (or inner point) of \(V\) if 
      and only if there exists some \(\delta > 0\) such that \(B_\delta \subseteq V\);
    \item \(x \in V\) is an isolated point of \(V\) is and only if there exists 
      there exists some \(\delta > 0\) such that \(B_\delta(x) \cap V = \{x\}\);
    \item \(x \in X\) is a limit point (or accumulation point) of \(V\) if and 
      only if for all \(\epsilon > 0\), 
      \((B_\epsilon \cap V) \setminus \{x\} \neq \varnothing\);
    \item \(x \in X\) is a boundary point of \(V\) if and only if for all 
      \(\epsilon > 0\), \(B_\epsilon(x) \cap V \neq \varnothing\) and 
      \(B_\epsilon(x) \cap V^c \neq \varnothing\).
  \end{itemize}
\end{definition}

Straight away, we see that every element of an open set is an interior
point while having no boundary points.

\begin{definition}
  Let \((X, d)\) be a metric space and \(V \subseteq X\). Then, 
  \begin{itemize}
    \item the interior of \(V\), \(V^\circ\) is the set of all interior points of \(V\);
    \item the closure of \(V\), \(\bar{V}\) is the union of \(V\) and all limit points of \(V\);
    \item the boundary of \(V\), \(\partial V\) is the set of boundary points of \(V\).
  \end{itemize}
\end{definition}

Again, the notion of the closure might be different in other
literatures. We might see the closure of \(V\) as the intersection of
all closed sets that are greater or equal to \(V\), that is the smallest
closed set containing \(V\). We once again find these two definitions to
be equivalent.

\begin{prop}
  Let \((X, d)\) be a metric space and \(V \subseteq X\). Then, 
  \[\bar{V} = \bigcap_{\substack{V \subseteq U \subseteq X \\ 
    U \mathop{\mathrm{closed}}}} U.\]
\end{prop}
\proof

It is obvious that the intersection of collections of closed sets is
closed, so to prove this proposition, it suffices to show that
\(\bar{V}\) is closed in \(X\) and for all closed
\(V \subseteq U \subseteq X\), \(U\) contains the limit points of \(V\).

Let \(x \in V\) be a limit point of \(V\), then, for all
\(\epsilon > 0\), \(B_\epsilon(x) \cap V \neq \varnothing\). Thus, we
can construct a sequence in \(V\) converging to \(x\) by the same method
as proposition \ref{closed_iff}. Now, as \(V \subseteq U\), this is also
a sequence in \(U\), to its limit \(x\), is also in \(U\).

Now, we show \(\bar{V}\) is closed. Suppose there exists some sequence
\((x_n)_{n \ge 1}\), \(x_n \to x\), we need to show \(x \in \bar{V}\).
But, we see straight away \(x\) is a limit point of \(V\) since for all
\(\epsilon > 0\), there exists some \(N \in \mathbb{N}\) such that for
all \(n \ge N\), \(x_n \in B_\epsilon(x)\), and as \(x_n \in V\),
\(x_n \in B_\epsilon(x) \cap V\), so the intersection is not empty and
we are done! \qed

\begin{prop}
  Let \((X, d)\) be a metric space and \(V \subseteq X\). Then, 
  \[V^\circ = \bigcup_{\substack{U \subseteq V \\ U \mathop{\mathrm{open}}}} U.\]
\end{prop}
\proof

Similarly, it suffices to show that \(V^\circ\) is open and for all
\(U \subseteq V\), \(U\) is open \(\implies\) \(U \subseteq V^\circ\).
\(V^\circ\) is open as, for all \(v \in V^\circ\), there exists some
\(\delta > 0\) such that \(B_\delta(v) \subseteq V\). Now, we see that
for all \(v' \in B_\delta(v)\), we define
\(\delta ' = \delta - d_X(v, v')\), then, for all
\(u \in B_{\delta '}(v')\),
\(d_X(u, v) \le d_X(u, v') + d_X(v', v)  < \delta ' + d_X(v', v) = \delta\),
so \(B_{\delta '}(v') \subseteq B_\delta(v)  \subseteq V\), that is
\(v'\) is an interior point of \(V\). So \(v' \in V^\circ\) and
\(B_\delta(v) \subseteq V^\circ\).

Now, let \(U \subseteq V\) be an open set. Then for all \(u \in U\),
there exists some \(\delta > 0\) such that
\(B_\delta(u) \subseteq U \subseteq V\), and so, \(u\) is an interior
point of \(V\) and \(U \subseteq V^\circ\). \qed

\begin{prop}
  Let \((X, d)\) be a metric space and \(V \subseteq X\). Then, 
  \(\partial V = \bar{V} \setminus V^\circ\).
\end{prop}
\proof

Let \(v \in \partial V\). If \(v \in V\) then it follows
\(v \in \bar{V}\). On the other hand, if \(v \in V^c\), then \(v\) is a
limit point of \(V\) as for all \(\epsilon > 0\),
\(B_\epsilon(x) \cap V =  (B_\epsilon(x) \cap V) \setminus \{v\} \neq \varnothing\).
Thus, by excluded middle \(v \in \bar{V}\). Now, if \(v \in V^\circ\)
then there exists some \(\delta > 0\) such that
\(B_\delta(v) \subseteq V\) implying
\(B_\delta(v) \cap V^c = \varnothing\) contradicting
\(v \in \partial V\). Thus,
\(\partial V \subseteq \bar{V} \setminus V^\circ\).

Now, let \(v \in \bar{V} \setminus V^\circ\). As \(v \in \bar{V}\), for
all \(\epsilon > 0\), \(B_\epsilon (v) \cap V \neq \varnothing\), thus
it suffices to show \(B_\epsilon (v) \cap V^c \neq \varnothing\).
Suppose \(B_\epsilon (v) \cap V^c = \varnothing\), then
\(B_\epsilon (v) \subseteq V\) implying \(v \in V^\circ\) \#. So
\(\bar{V} \setminus V^\circ \subseteq \partial V\), and hence
\(\partial V = \bar{V} \setminus V^\circ\). \qed

\hypertarget{maps-between-metric-spaces}{%
\subsection{Maps Between Metric
Spaces}\label{maps-between-metric-spaces}}

Just as how we can consider properties about functions of real numbers,
we can define equivalent properties for function between metric spaces.

\begin{definition}[Continuity at a Point]
  Let \((X, d_X), (Y, d_Y)\) be two metric spaces and let \(f : X \to Y\) be 
  a map. Then, we say that \(f\) is continuous at some \(x \in X\) if and only if 
  for all \(\epsilon > 0\) there exists a \(\delta > 0\) such that 
  \(f(B_\delta(x)) \subseteq B_\epsilon(f(x))\).  
\end{definition}

\begin{definition}[Uniformly Continuous]
  Let \((X, d_X), (Y, d_Y)\) be two metric spaces and let \(f : X \to Y\) be 
  a map. Then, we say that \(f\) is uniformly continuous if and only if 
  for all \(\epsilon > 0\) there exists a \(\delta > 0\) such that for 
  all \(x \in X\), \(f(B_\delta(x)) \subseteq B_\epsilon(f(x))\).  
\end{definition}

This is essentially the same idea as the definition of continuity in
Euclidean spaces. While, it makes sense to define it this way, we shall
encounter a different notion of continuity later in this course --
continuity on topological spaces. While we shall not talk about it too
much here, it turns out that these notions of continuity is consistent
on metric spaces, that is, a function is continuous on some metric
spaces if and only if it is continuous on their induced topologies.

\begin{prop}
  Let \((X, d_X), (Y, d_Y)\) be two metric spaces and let \(f : X \to Y\) be 
  a map. Then \(f\) is continuous on \(X\) if and only if \(f^{-1}(U)\) is open 
  in \(X\) for all open \(U \subset Y\)
\end{prop}
\proof

\((\implies)\) Suppose \(f\) is continuous, then, for all
\(v \in f^{-1}(U)\), where \(U\) is open in \(Y\), we have
\(f(v) \in U\). So, as \(U\) is open, there exists some \(\epsilon > 0\)
such that \(B_\epsilon(f(v)) \subseteq U\). By continuity of \(f\),
there exists some \(\delta > 0\) such that
\(f(B_\delta(v)) \subseteq B_\epsilon(f(v)) \subset U\) so,
\(B_\delta(v) \subset f^{-1}(U)\), and thus, \(f^{-1}(U)\) is open.

\((\impliedby)\) Suppose now that \(f^{-1}(U)\) is open in \(X\) for all
open \(U \subset Y\), then for all \(x \in X\), \(\epsilon > 0\), as
\(B_\epsilon(f(x))\) is open, \(f^{-1}(B_\epsilon(f(x)))\) is also open.
Now, as \(x \in f^{-1}(B_\epsilon(f(x)))\), there exists some
\(\delta > 0\) such that
\(B_\delta(x) \subseteq f^{-1}(B_\epsilon(f(x)))\) and so,
\(f(B_\delta(x)) \subseteq B_\epsilon(f(x))\). \qed

\begin{prop}
  Let \((X, d_X), (Y, d_Y), (Z, d_Z)\) be metric spaces and let 
  \(f\) and \(g : Y \to Z\) be maps, then, if \(f, g\) are both continuous at 
  some \(x \in X\), then \(g \circ f\) is continuous at \(x\).
\end{prop}
\proof

Trivial. \qed

\begin{definition}[Homeomorphism]
  Let \((X, d_X), (Y, d_Y)\) be two metric spaces and let \(f : X \to Y\) be 
  a map. Then, we call \(f\) a homeomorphism between \(X\) and \(Y\) if and only 
  if \(f\) is a bijection and both \(f\) and \(f^{-1}\) are continuous. If such 
  a homeomorphism exists between two metric spaces, then we call them homeomorphic.
\end{definition}

A classic example of homeomorphisms between metric spaces is the
homeomorphism between \((-1, 1)\) and \(\mathbb{R}\) by scaling \(\tan\)
(of course this is not the only homeomorphism between the two, we see
that \(x \mapsto \log(1 + x) - \log(1 - x)\) is also a homeomorphism).
With this, we can conclude that \(\mathbb{R}\) is homeomorphic to any
open intervals. Furthermore, we find that \((a_1, b_1]\) is homeomorphic
to \([a_2, b_2)\) by considering that \((-1, 1] \cong [-1, 1)\) by
\(x \mapsto -x\).

\begin{definition}[Lipschitz Continuous]
  Let \((X, d_X), (Y, d_Y)\) be two metric spaces and let \(f : X \to Y\) be 
  a map. We call \(f\) Lipschitz continuous if and only if there exists some 
  \(M > 0\) such that for all \(x_1, x_2 \in X\), 
  \[d_Y(f(x_1), f(x_2)) \le M d_X(x_1, x_2).\]
\end{definition}

\begin{definition}[bi-Lipschitz]
  Let \((X, d_X), (Y, d_Y)\) be two metric spaces and let \(f : X \to Y\) be 
  a map. We call \(f\) bi-Lipschitz if and only if there exists \(M_0, M_1 > 0\) 
  such that for all \(x_1, x_2 \in X\), we have 
  \[M_2 d_X(x_1, x_2) \le d_Y(f(x_1), f(x_2)) \le M_1 d_X(x_1, x_2).\]
\end{definition}

We see that bi-Lipschitz is stronger than Lipschitz continuity which is
in turn a stronger condition than uniform continuity by simply choosing
\(\delta = \epsilon / 2M\).

\begin{definition}[Isometry]
  Let \((X, d_X), (Y, d_Y)\) be two metric spaces and let \(f : X \to Y\) be 
  a map. We call \(f\) an isometry if and only if for all \(x_1, x_2 \in X\),
  \[d_Y(f(x_1), f(x_2)) = d_X(x_1, x_2).\]
\end{definition}

\hypertarget{topological-spaces}{%
\subsection{Topological Spaces}\label{topological-spaces}}

We generalise metric spaces such that instead of defining the notion of
openness through metrics, we bundle it with in the structure itself --
namely topologies.

\begin{definition}[Topologial Space]
  The ordered pair \((X, \mathcal{T})\) is call a topologically space if and only 
  if \(\mathcal{T}\) is a collection of subsets of \(X\) such that, 
  \begin{itemize}
    \item \(X \in \mathcal{T}\);
    \item for all \(U, V \in \mathcal{T}\), \(U \cap V \in \mathcal{T}\);
    \item for all \(\mathcal{I} \subset \mathcal{T}\), 
      \(\bigcup \mathcal{I} \in \mathcal{T}\).
  \end{itemize}
  We call \(\mathcal{T}\) a topology on \(X\) and for all \(U \in \mathcal{T}\), 
  we say \(U\) is open.
\end{definition}

In many literatures, topologies must also satisfy the condition that
\(\varnothing \in \mathcal{T}\). We have omitted it here since the above
conditions imply that \(\varnothing \in \mathcal{T}\).

\begin{prop}
  Let \((X, \mathcal{T})\) be a topological space. Then \(\varnothing \in \mathcal{T}\).
\end{prop}
\proof

Let \(\mathcal{I}\) be the empty collection of subsets of \(X\). Then,
as for all \(U \in \mathcal{I}, U \in \mathcal{T}\) (as \(\mathcal{I}\)
is empty), by definition
\(\varnothing = \bigcup \mathcal{I} \in \mathcal{T}\). \qed

We can construct topologies quite easily. In fact, all metrics \(d\)
induces a topology with their notion of openness,
i.e.~\(\mathcal{T} := \{U \subseteq X \mid U \text{ is open w.r.t } d\}\).
With this definition, we have the topology induced by to discrete
metric, resulting in the discrete topology where
\(\mathcal{T} = \mathcal{P}(X)\). Another trivial topology is the coarse
topology where \(\mathcal{T} := \{\varnothing, X\}\).

\begin{prop}
  Let \((X, \mathcal{T})\) be a topological space and \(U \subseteq X\), then, 
  \(U \in \mathcal{T}\) if and only if for all \(u \in U\), there exists some 
  open neighbourhood \(\mathcal{N}\) of \(u\), \(\mathcal{N} \subseteq U\).
\end{prop}

Again, by open neighbourhood of \(u\) we mean an open set containing
\(u\). \proof If \(U \in \mathcal{T}\) then we can simply choose
\(\mathcal{N} = U\) and we are done so let us consider the backwards
direction. Suppose for all \(u \in U\), there exists an open
neighbourhood of \(u\), \(\mathcal{N}_u \subseteq U\). Then, by defining
the collection of open sets,
\(\mathcal{I} := \{\mathcal{N}_u \mid u \in U\}\), we have
\(U = \bigcup \mathcal{I} \in \mathcal{T}\). \qed

We in general refers to the above property where there exists some open
neighbourhood \(\mathcal{N}\) of \(u\) such that
\(\mathcal{N} \subseteq U\) as \(U\) is \emph{locally open} at \(u\).

With a structure like this with clear correspondence with the metric
spaces, we can straight away translate some definitions from metric
spaces to topological spaces.

\begin{definition}[Subspace]
  Let \((X, \mathcal{T})\) be a topological space and suppose \(Y \subseteq X\). 
  Then the subspace \((Y, \mathcal{T}_Y)\) is a topological space where 
  \[\mathcal{T}_Y := \{Y \cap U \mid U \in \mathcal{T}\}.\]
\end{definition}

\begin{definition}[Interior Point and Interior]
  Let \((X, \mathcal{T})\) be a topological space and suppose \(U \subseteq X\).
  Then we call \(x \in U\) an interior point of \(U\) if and only if \(U\) is 
  locally open as \(x\). Furthermore, we define the interior of \(U\), denoted 
  \(U^\circ\) as the set of interior points of \(U\).
\end{definition}

Again, alike the metric spaces definition, we see that the interior of
any set is open since, for all \(u \in U^\circ\), there exists
\(\mathcal{N}_u \subseteq U\). Now as \(\mathcal{N}_u\) is open, for all
\(n \in \mathcal{N}_u\), there exists
\(\mathcal{N}_n \subseteq \mathcal{N}_u \subseteq U\), that is \(n\) is
also an interior point of \(U\). Thus,
\(\mathcal{N}_u \subseteq U^\circ\) and hence, \(U^\circ\) is locally
open everywhere and therefore, is open.

\begin{definition}[Closed Sets]
  Let \((X, \mathcal{T})\) be a topological space and suppose \(U \subseteq X\). 
  Then we say \(U\) is closed if and only if \(U^c\) is open.
\end{definition}

We can deduce some properties about closed sets straight away.

\begin{prop}
  Let \((X, \mathcal{T})\) be a topological space. Then 
  \begin{itemize}
    \item \(\varnothing\) and \(X\) are both closed;
    \item the intersection of a collection of closed sets is also closed;
    \item the union of any finite number of closed sets is also closed.
  \end{itemize}
\end{prop}
\proof

Trivial. \qed

\begin{definition}[Limit Point and Closure]
  Let \((X, \mathcal{T})\) be a topological space and suppose \(U \subseteq X\),
  then we call \(u \in U\) a limit point of \(U\) if and only if for all 
  \(\mathcal{N} \in \mathcal{T}\) containing \(x\), 
  \(U \cap \mathcal{N} \neq \varnothing\). We denote the set of limit points 
  of \(U\) by \(\bar{U}\) and we call it the closure of \(U\).
\end{definition}

Similarly, we see that the set of limit points is closed. This is
because for all \(u \in \bar{U}^c\), \(u\) is not a limit point of
\(U\), so there exists some open neighbourhood of \(u\), \(\mathcal{N}\)
such that \(U \cap \mathcal{N} = \varnothing\). Now, as \(\mathcal{N}\)
is open, it is locally open, so for all \(n \in \mathcal{N}\) there
exists an open neighbourhood of \(n\), \(\mathcal{N}_n\) such that
\(\mathcal{N}_n \subseteq \mathcal{N}\) and so
\(U \cap \mathcal{N}_n = \varnothing\) implying \(n \in \bar{U}^c\) and
hence, \(\mathcal{N} \subseteq \bar{U}^c\), that is, \(\bar{U}^c\) is
locally open, and hence, is open.

Furthermore, we see that the interior and the set of limit points of
\(U\) are the largest open set contained by \(U\) and the smallest
closed set containing \(U\) respectively. We will provide a proof of the
former while the proof of the latter is similar.

\begin{prop}
  Let \((X, \mathcal{T})\) be a topological space and let \(U \subseteq X\). 
  Then 
  \[U^\circ = \bigcup \{V \in \mathcal{T} \mid V \subseteq U\},\]
  and 
  \[\bar{U} = \bigcap \{V \subseteq X \mid U \subseteq V \wedge V^c \in \mathcal{T}\}.\]
\end{prop}
\proof

\(U^\circ \subseteq \bigcup \{V \in \mathcal{T} \mid V \subseteq U\}\)
trivially as \(U^\circ \in \mathcal{T}\) as shown previously.

Let \(v \in \bigcup \{V \in \mathcal{T} \mid V \subseteq U\}\), then
there exists some \(V \in \mathcal{T}\) such that \(v \in V\) and
\(V \subseteq U\), so \(U\) is locally open at \(v\), and hence
\(v \in U^\circ\). Thus,
\(U^\circ = \bigcup \{V \in \mathcal{T} \mid V \subseteq U\}\) as
required. \qed

Lastly, we find the interior and closure operations to be monotonic.

\begin{prop}
  Let \((X, \mathcal{T})\) be a topological space and suppose 
  \(U, V \subseteq X\), \(U \subseteq V\). Then \(U^\circ \subseteq V^\circ\) 
  and \(\bar{U} \subseteq \bar{V}\).
\end{prop}
\proof

See problem sheet 7. \qed

\hypertarget{convergence-in-topological-spaces}{%
\subsection{Convergence in Topological
Spaces}\label{convergence-in-topological-spaces}}

Just as all the spaces we have studied thus far, there is also a notion
of convergence in the topological spaces.

\begin{definition}
  Let \((X, \mathcal{T})\) be a topological space and suppose 
  \((x_n)_{n \ge 1}\) is a sequence in \(X\). Then we say that \((x_n)\) converges 
  in \((X, \mathcal{T})\) if and only if there exists some \(x \in X\) such 
  that for all \(U \in \mathcal{T}\), \(x \in U\), there exists \(N \in \mathbb{N}\), 
  such that for all \(n \ge N\), \(x_n \in U\).

  Whenever this is true, we say that \(x_n\) converges to \(x\) and we write 
  \(x_n \to x\).
\end{definition}

However, unlike the Euclidean and metric spaces, it is not necessarily
true that every convergent sequence has a unique limit. This is easy to
see by considering that, given the coarse topology on
\(X, \mathcal{T}\), every sequence is convergent to any point within
\(X\), thus a counterexample. Because of this, we propose the notion of
\emph{Hausdorff} which does imply the uniqueness of limits.

\begin{definition}[Hausdorff]
  A topological space \((X, \mathcal{T})\) is called Hausdorff if and only if 
  for all \(x, y \in X\), \(x \neq y\), there exists \(U, V \in \mathcal{T}\) 
  such that \(x \in U\) and \(y \in V\) and \(U \cap V = \varnothing\).
\end{definition}

We in general refer to a topological space that is Hausdorff as a
Hausdorff space.

\begin{theorem}
  Let \((X, \mathcal{T})\) be a Hausdorff space and suppose \((x_n)_{n \ge 1}\) 
  is a convergent sequence in \(X\). Then, if there exists \(x, y \in X\) such 
  that \(x_n \to x\) and \(x_n \to y\), then \(x = y\).
\end{theorem}
\proof

Suppose otherwise that \(x \neq y\). Then, as \(X\) is Hausdorff, it
separates \(x\) and \(y\), so there exists \(U, V \in \mathcal{T}\),
such that \(x \in U\) and \(y \in V\) such that
\(U \cap V = \varnothing\). By the definition of convergence, there
exists \(N_1, N_2 \in \mathbb{N}\) such that for all \(n \ge N_1\),
\(x_n \in U\) and for all \(n \ge N_2\) \(x_n \in V\). But then
\(x_{\max\{N_1, N_2\}} \in U \cap V = \varnothing\) \#.\\
\qed

With the notion of convergence of sequence defined, we see that the
notion of closeness on topological spaces is related to the definition
of closeness on metric spaces.

\begin{prop}
  Let \((X, \mathcal{T})\) be a topological space and suppose \(U \subseteq X\). 
  Then \(U\) is closed if and only if for all \((x_n)_{n \ge 1}\) in \(U\), if 
  \((x_n)\) converges then it converges in \(U\).
\end{prop}
\proof

Identical to the proof for metric spaces. \qed

\begin{prop}
  Let \((X, \mathcal{T})\) be a Hausdorff space and suppose \(x \in X\). 
  Then the singleton \(\{x\}\) is closed.
\end{prop}
\proof

Consider for all \(y \in X \setminus \{x\}\), \(y \neq x\) and thus, by
Hausdorff, there exists \(U, V \in \mathcal{T}\) such that
\(x \in U, y \in V\), and \(U \cap V = \varnothing\), and so,
\(x \not\in V\), that is \(V \subseteq X \setminus \{x\}\). Now, as
\(V\) by definition is a open set containing \(y\) in
\(X \setminus \{x\}\), \(X \setminus \{x\}\) is locally open everywhere,
and hence \(X \setminus \{x\}\) is open and \(\{x\}\) is closed. \qed

\begin{prop}
  The induced topology of any metric space is Hausdorff.
\end{prop}
\proof

Let \((X, d)\) be a metric space and let \(x, y\) be unique points of
\(X\). Then, by letting \(\epsilon = d(x, y) / 2 > 0\), we have
\(B_\epsilon(x),  B_\epsilon(y) \in \mathcal{T}_d\) containing \(x\) and
\(y\) respectively that are disjoint. Hence, \((X, \mathcal{T}_d)\) is
Hausdorff. \qed

\hypertarget{maps-on-topological-spaces}{%
\subsection{Maps on Topological
Spaces}\label{maps-on-topological-spaces}}

As alluded towards during our discussion of continuous maps between
metric spaces, there is a natural definition of continuity on
topological spaces by considering the preimage of an open set.

\begin{definition}[Continuous on Topological Spaces]
  Let \((X, \mathcal{T}_X)\), \((Y, \mathcal{T}_Y)\) be topological spaces and 
  suppose \(f : X \to Y\) is a mapping between the two. Then we say \(f\) is 
  continuous on \(X\) if and only if for all \(U \in \mathcal{T}_Y\), 
  \(f^{-1}(U) \in \mathcal{T}_X\).
\end{definition}

To prove that a map is continuous on some topology, it is often helpful
to consider the preimage of closed sets instead. We will show that this
is possible.

\begin{prop}
  Let \((X, \mathcal{T}_X)\), \((Y, \mathcal{T}_Y)\) be topological spaces and 
  suppose \(f : X \to Y\) is a mapping between the two. Then, \(f\) is continuous 
  if and only if the \(f\)-preimage of any closed sets is closed.
\end{prop}
\proof

We note that \(f^{-1}(U)^c = f^{-1}(U^c)\) for all \(U \subseteq Y\) and
the result follows. \qed

\begin{prop}
  Let \((X, \mathcal{T}_X)\), \((Y, \mathcal{T}_Y)\) and \((Z, \mathcal{T}_Z)\) 
  be topological spaces and suppose \(f : X \to Y\) and \(g : Y \to Z\) are 
  continuous. Then \(g \circ f : X \to Z\) is also continuous.
\end{prop}
\proof

Trivial. \qed

Again with all structures, we would like to consider a notion of
equivalence.

\begin{definition}[Homeomorphism]
  Let \((X, \mathcal{T}_X)\), \((Y, \mathcal{T}_Y)\) be topological spaces and 
  suppose \(f : X \to Y\) is a mapping between them. Then we call \(f\) a 
  homeomorphism if and only if \(f\) is bijective and both \(f\) and \(f^{-1}\) 
  are continuous. 

  If such a homeomorphism exists between \(X\) and \(Y\) then we say \(X\) and 
  \(Y\) are homeomorphic or topologically equivalent and we write \(X \cong Y\).
\end{definition}

\begin{definition}[Topological Property]
  Given a predicate \(P\) on the set of topological spaces (that is given 
  a topological space \(X\), \(P(X)\) is either true or false), we say \(P\) is 
  a topological property if and only if for all homeomorphic \(X, Y\), 
  \(P(X) \to P(Y)\).
\end{definition}

Consider the set of all topological spaces. One can show that
homeomorphisms gives us an equivalence relation on this set, and thus,
the set of all topological spaces is partitioned into equivalence
classes. In general, we would rather study theses classes instead of
individual topological spaces themselves and this definition of what is
means to be a topological property gives use the means to do exactly
that. That is, we study topological properties on theses equivalence
classes by simply examining the property on a single element of this
class.

\begin{prop}
  Hausdorff is a topological property. 
\end{prop}
\proof

Let \((X, \mathcal{T}_X)\), \((Y, \mathcal{T}_Y)\) be topological spaces
with homeomorphism \(f : X \cong Y\). Suppose that \(X\) is Hausdorff,
and let \(y_1, y_2 \in Y\), \(y_1 \neq y_2\). Then, as \(f\) is
bijective, there exists \(x_1, x_2 \in X\), such that \(f(x_1) = y_1\)
and \(f(x_2) = y_2\). Now, as \(X\) is Hausdorff, there exists
\(U, V \in \mathcal{T}_X\) such that \(x_1 \in U\), \(x_2 \in V\) and
\(U \cap V = \varnothing\). So, as \(f\) is a homeomorphism, \(f^{-1}\)
is continuous, and thus,
\((f^{-1})^{-1}(U), (f^{-1})^{-1}(V) \in \mathcal{T}_Y\). Now, as
\(x_1 = f^{-1}(y_1) \in U\) and \(x_2 = f^{-1}(y_2) \in V\), we have
\(y_1 \in (f^{-1})^{-1}(U)\) and \(y_2 \in (f^{-1})^{-1}(V)\).
Furthermore, as \(U \cap V = \varnothing\),
\((f^{-1})^{-1}(U) \cap (f^{-1})^{-1}(V) =  \varnothing\), and so \(Y\)
is also Hausdorff. \qed

\hypertarget{connectedness}{%
\subsection{Connectedness}\label{connectedness}}

We would like to consider a notion of connectedness in both the metric
and topological spaces. To motivate this, let us recall the intermediate
value theorem for the real numbers.

\begin{prop}
  Let \(f : [a, b] \to \mathbb{R}\) be continuous and let \(r \in \mathbb{R}\) 
  such that \(f(a) < r < f(b)\) Then, there exists some \(c \in [a, b]\) such 
  that \(f(c) = r\).
\end{prop}

We would like to generalise this function and as we will see, this is
closely related to the notion of connectedness.

\begin{definition}[Connected Sets]
  Let \((X, \mathcal{T})\) be a toplogical space and let \(S \subseteq X\). 
  We say that \(S\) is disconnected if and only if there exists 
  \(U, V \in \mathcal{T}\) such that \(S \subseteq U \sqcup V\) and 
  \(S \cap U, S \cap V \neq \varnothing\). We say \(S\) is connected if and only 
  if it is not disconnected.
\end{definition}

We can easily generalise this notion of connectedness to the entire
topological space and furthermore, to metric spaces and its subsets. We
say a topological space is disconnected if and only if it is
disconnected in itself, and for metric spaces, we can simply apply the
definition to the induced topology.

\begin{prop}
  Let \((X, \mathcal{T})\) be a topological space, then \(X\) is disconnected 
  if and only if there exists some surjective continuous map \(f : X \to \{0, 1\}\),
  where we equip \(\{0, 1\}\) with the discrete topology. 
\end{prop}
\proof

As \(\{0, 1\}\) is equipped with the discrete topology, both \(\{0\}\)
and \(\{1\}\) are open, and thus,
\(f^{-1}(\{0\}), f^{-1}(\{1\}) \in \mathcal{T}\). Now, as \(f\) is
surjective, \(f^{-1}(\{0\}) \sqcup f^{-1}(\{1\}) = X\) and furthermore,
neither \(f^{-1}(\{0\})\) nor \(f^{-1}(\{1\})\) are empty, we have \(X\)
is disconnected.

Now, suppose \(X\) is disconnected and \(U, V \in \mathcal{T}\) such
that \(X = U \sqcup V\), then we can easily construct the function
\[f = 0 \text{ if } x \in U, 1 \text{ otherwise.}\] We see straight away
that \(f\) is a continuous surjection so we are done! \qed

The above proposition can be rearranged into alternative forms that are
useful for some situations. One can easily prove that the above
condition holds whenever \(\{0, 1\}\) is considered as a set of
\(\mathbb{R}\) equipped with the Euclidean metric. This results in the
direct corollary that an interval in \(\mathbb{R}\) is connected.

\begin{corollary}
  Let \(I \subseteq \mathbb{R}\) be an interval. Then \(I\) is connected.
\end{corollary}
\proof

Suppose otherwise, then there exists some continuous
\(f : I \to \mathbb{R}\) such that \(f(I) = \{0, 1\}\). Then there
exists \(a, b \in I\) such that \(f(a) = 0\) and \(f(b) = 1\). WLOG. let
us assume that \(a < b\), then by the intermediate value theorem, there
exists some \(c \in [a, b]\) such that \(f(c) = 1 / 2\). But, by
definition \(1 / 2 \not\in f(I)\) \#. \qed

\begin{prop}
  Let \((X, \mathcal{T}_X)\), \((Y, \mathcal{T}_Y)\) topological spaces and let 
  \(f : X \to Y\) be a continuous mapping. Then if \(X\) is connected, so is 
  \(f(X)\) connected in \(Y\).
\end{prop}
\proof

Suppose otherwise, then there exists some continuous surjective
\(g : f(X) \to \{0, 1\}\). Thus, as \(f : X \to f(X)\) is obviously
surjective, and as the composition of continuous surjective functions
are continuous surjective, we have \(g \circ f : X \to \{0, 1\}\) is
continuous surjective implying \(X\) is disconnected \#. \qed

From this, we can deduce that \emph{connectedness} is a topological
property as if \(f : X \cong Y\) is a homeomorphism, then \(f(X) = Y\)
and \(f^{-1}(Y) = X\) implying one being connected implies the other is
also connected.

Finally, let us generalise the intermediate value theorem for arbitrary
topological spaces, and thus, also for metric and Euclidean spaces.

\begin{theorem}
  Let \((X, \mathcal{T})\) be a connected topological space and suppose 
  \(f : X \to \mathbb{R}\) is continuous such that there exists \(a, b \in X\), 
  \(f(a) < r < f(b)\) for some \(r \in \mathbb{R}\), then there exists some 
  \(c \in X\) such that \(f(c) = r\).
\end{theorem}
\proof

Suppose otherwise that for all \(x \in X\), \(f(x) \neq r\), then let us
consider the sets \(U := f^{-1}(-\infty, r)\) and
\(V := f^{-1}(r, \infty)\). We see straight away that
\(U \cap V = \varnothing\) and \(U, V \in \mathcal{T}\) as \(f\) is
continuous. Furthermore, for all
\(x \in X, f(x) \in  (-\infty, r) \cup (r, \infty)\), so
\(X = U \cup V\), and thus, \(X\) is disconnected /\#. \qed

With that, we have provided an alternative proof of the intermediate
value theorem as any closed intervals of \(\mathbb{R}\) is connected.

\begin{prop}
  Let \((X, \mathcal{T})\) be a topological space and let \((A_n)_{n \in I}\) 
  be a family of connected subsets of \(X\) indexed by some set \(I\) such 
  that for all \(i, j \in I\), \(i \neq j\), \(A_i \cap A_j \neq \varnothing\).
  Then, \(\bigcup_{i \in I} A_i\) is also connected.
\end{prop}
\proof

Suppose otherwise, that \(\bigcup_{i \in I} A_i\) is disconnected, then
there exists some \(U, V \in \mathcal{T}\) such that
\(\bigcup_{i \in I} A_i \subseteq U \sqcap V\). Now, if there exists
some \(i \in I\) such that \(A_i \not\subseteq U\) and
\(A_i \not\subseteq V\), we have \(A_i\) is disconnected, that is
\(A_i \subseteq U\) or \(A_i \subseteq V\). Thus,
\(I = \{i \in I \mid A_i \subseteq U\} \sqcup  \{i \in I \mid A_i \subseteq V\}\),
furthermore, as neither sets are empty since otherwise
\(U \cap \bigcup_{i \in I} A_i = \varnothing\) or
\(V \cap \bigcup_{i \in I} A_i = \varnothing\), we have, there exists
\(i, j \in I\), \(A_i \subseteq U\) and \(A_j \subseteq V\). But, by
assumption \(A_i \cap A_j \neq \varnothing\) and so
\(U \cap V \neq \varnothing\) \#. \qed

This fits well with our intuition that we can ``glue'' connected sets
together (with some overlaps) such that the resulting set to also be
connected. Another intuitive method of visualising connectedness is that
a set is connected if we can draw a continuous line from any point
within the set to any other points. This results in the definition of
\emph{path-connected} sets which is a stronger notion of connectedness
then connected.

\begin{definition}[Path]
  Let \((X, \mathcal{T})\) be a topological space and let \(x_0, x_1 \in X\). 
  Then a path from \(x_0\) to \(x_1\) is a continuous function 
  \(\gamma : [0, 1] \to X\), such that \(\gamma (0) = x_0\) and \(\gamma (1) = x_1\).
\end{definition}

\begin{definition}[Path-Connected]
  Let \((X, \mathcal{T})\) be a topological space. We call \(X\) path-connected 
  if and only if for all \(x_0, x_1 \in X\), there exists some path connecting 
  \(x_0\) to \(x_1\).
\end{definition}

\begin{prop}
  Let \((X, \mathcal{T})\) be a topological space. Then, if \(X\) is path-connected, 
  it is connected.
\end{prop}
\proof

Suppose \(X\) is disconnected. Then there exists some continuous
surjection \(f : X \to \{0, 1\}\) with respect to the discrete topology.
Now, as \(f\) is surjective, there exists some \(x_0, x_1 \in X\) such
that \(f(x_0) = 0\) and \(f(x_1) = 1\). By path-connectedness of \(X\),
there exists some path \(\gamma\) from \(x_0\) to \(x_1\). So, by
letting \(I = \gamma[0, 1]\), we have the continuous surjection
\(f \mid_I \circ \gamma : [0, 1] \to \{0, 1\}\) implying that the
interval \([0, 1]\) is disconnected \#. \qed

\begin{prop}
  Let \((X, \mathcal{T}_X)\), \((Y, \mathcal{T}_Y)\) be two topological spaces, 
  and suppose \(f : X \cong Y\) is a homeomorphism. Then, \(X\) is path-connected 
  if and only if \(Y\) is path-connected, that is path-connectedness is a 
  topological property.
\end{prop}
\proof

Follows straight away by composing \(f\) and some appropriately chosen
path. \qed

This notion of connectedness is useful and a nice corollary of the above
is that \(\mathbb{R}^2 \not\cong \mathbb{R}\). This is true since if
they are homeomorphic, by say \(f\), we have the restriction
homeomorphism on the subspaces
\(\mathbb{R}^2 \setminus \{0\} \cong \mathbb{R} \setminus \{f(0)\}\).
But, we can easily see that while \(\mathbb{R}^2 \setminus \{0\}\) is
path-connected, \(\mathbb{R} \setminus \{f(0)\}\) is not, \#.

\hypertarget{extras}{%
\section{Extras}\label{extras}}

This is a temporary section and all contents within this section will be
moved to their appropriate sections once the notes is complete.

\hypertarget{completeness-is-not-a-topological-property}{%
\subsubsection{Completeness is not a Topological
Property}\label{completeness-is-not-a-topological-property}}

Completeness is not a topological property. Heuristically this makes
sense as in general, it does not make sense to call a metric space
complete as completeness is a property of metric spaces and not all
topologies are metric spaces.

\begin{theorem}
  Completeness is not a Topological Property.
\end{theorem}
\proof

To show this, we need find two metrics on some set that induces the same
topology but one is complete while the other is not. We see that
\((0, 1)\) is not complete with respect to the Euclidean metric but
\(\mathbb{R}\) is with \(\mathbb{R}\) being homeomorphic to \((0, 1)\).
Thus, the main idea is to pull back the metric on \(\mathbb{R}\) on to
\((0, 1)\) over the homeomorphism.

By theorem \ref{induce_metric}, as \(\tan : (0, 1) \to \mathbb{R}\) is
injective we have a induced metric on \((0, 1)\), \(d_{\tan}\). I claim
the topologies induced by \((X, d_{\| \cdot \|})\) and \((X, d_{\tan})\)
are the same, that is for all \(U \subseteq (0, 1)\), \(U\) is open in
\(d_{\| \cdot \|}\) if and only if it is open in \(d_{\tan}\). But this
is true as \(\tan\) and \(\arctan\) are both continuous, so we are done!
\qed

\hypertarget{properties-of-the-quotient-map}{%
\subsubsection{Properties of the Quotient
Map}\label{properties-of-the-quotient-map}}

\begin{theorem}
  Let \((X, \mathcal{T})\) be a topological space and let \(\sim\) be an equivalence 
  relation on \(X\). Let us denote the quotient map from \(X\) to \(X / \sim\) as 
  \(q\) where \(q(x) = [x] = \{x' \in X \mid x \sim x'\}\). Then, \(q\) is continuous
  and surjective.
\end{theorem}
\proof

Surjectivity is trivial while \(q\) being continuous is true by
definition. \qed

\end{document}
