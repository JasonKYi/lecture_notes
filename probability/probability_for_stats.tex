% Options for packages loaded elsewhere
\PassOptionsToPackage{unicode}{hyperref}
\PassOptionsToPackage{hyphens}{url}
\PassOptionsToPackage{dvipsnames,svgnames*,x11names*}{xcolor}
%
\documentclass[
]{article}
\usepackage{lmodern}
\usepackage{amssymb,amsmath}
\usepackage{ifxetex,ifluatex}
\ifnum 0\ifxetex 1\fi\ifluatex 1\fi=0 % if pdftex
  \usepackage[T1]{fontenc}
  \usepackage[utf8]{inputenc}
  \usepackage{textcomp} % provide euro and other symbols
\else % if luatex or xetex
  \usepackage{unicode-math}
  \defaultfontfeatures{Scale=MatchLowercase}
  \defaultfontfeatures[\rmfamily]{Ligatures=TeX,Scale=1}
\fi
% Use upquote if available, for straight quotes in verbatim environments
\IfFileExists{upquote.sty}{\usepackage{upquote}}{}
\IfFileExists{microtype.sty}{% use microtype if available
  \usepackage[]{microtype}
  \UseMicrotypeSet[protrusion]{basicmath} % disable protrusion for tt fonts
}{}
\makeatletter
\@ifundefined{KOMAClassName}{% if non-KOMA class
  \IfFileExists{parskip.sty}{%
    \usepackage{parskip}
  }{% else
    \setlength{\parindent}{0pt}
    \setlength{\parskip}{6pt plus 2pt minus 1pt}}
}{% if KOMA class
  \KOMAoptions{parskip=half}}
\makeatother
\usepackage{xcolor}
\IfFileExists{xurl.sty}{\usepackage{xurl}}{} % add URL line breaks if available
\IfFileExists{bookmark.sty}{\usepackage{bookmark}}{\usepackage{hyperref}}
\hypersetup{
  pdftitle={Probability for Statistics},
  pdfauthor={Kexing Ying},
  colorlinks=true,
  linkcolor=Maroon,
  filecolor=Maroon,
  citecolor=Blue,
  urlcolor=red,
  pdfcreator={LaTeX via pandoc}}
\urlstyle{same} % disable monospaced font for URLs
\usepackage[margin = 1.5in]{geometry}
\usepackage{graphicx}
\makeatletter
\def\maxwidth{\ifdim\Gin@nat@width>\linewidth\linewidth\else\Gin@nat@width\fi}
\def\maxheight{\ifdim\Gin@nat@height>\textheight\textheight\else\Gin@nat@height\fi}
\makeatother
% Scale images if necessary, so that they will not overflow the page
% margins by default, and it is still possible to overwrite the defaults
% using explicit options in \includegraphics[width, height, ...]{}
\setkeys{Gin}{width=\maxwidth,height=\maxheight,keepaspectratio}
% Set default figure placement to htbp
\makeatletter
\def\fps@figure{htbp}
\makeatother
\setlength{\emergencystretch}{3em} % prevent overfull lines
\providecommand{\tightlist}{%
  \setlength{\itemsep}{0pt}\setlength{\parskip}{0pt}}
\setcounter{secnumdepth}{5}
\usepackage{tikz}
\usepackage{amsthm}
\usepackage{mathtools}
\usepackage{lipsum}
\usepackage[ruled,vlined]{algorithm2e}
\newtheorem{theorem}{Theorem}
\newtheorem{prop}{Proposition}[theorem]
\newtheorem{corollary}{Corollary}[theorem]
\newtheorem*{remark}{Remark}
\theoremstyle{definition}
\newtheorem{definition}{Definition}[section]

\title{Probability for Statistics}
\author{Kexing Ying}
\date{May 15, 2020}

\begin{document}
\maketitle

{
\hypersetup{linkcolor=}
\setcounter{tocdepth}{2}
\tableofcontents
}
\hypertarget{introduction}{%
\section{Introduction}\label{introduction}}

\hypertarget{probability-measures}{%
\subsection{Probability Measures}\label{probability-measures}}

Last year we saw briefly constructions and definitions relevant to
working with probabilities such as \(\sigma\)-algebras, random variables
and more. We will revisit them here with a more general (and more
technical) approach.

\begin{definition}[\(\sigma\)-algebra]
  Let \(X\) be a set. A \(\sigma\)-algebra on \(X\), \(\mathcal{A}\) is a 
  collection of subsets of \(X\) such that 
  \begin{itemize}
    \item \(\varnothing \in \mathcal{A}\) 
    \item for all \(A \in \mathcal{A}\), \(A^C \in \mathcal{A}\)
    \item for all \((A_n)_{n = 1}^\infty \subseteq \mathcal{A}\), 
      \(\bigcup_n A_n \in \mathcal{A}\).
  \end{itemize}
\end{definition}

\begin{prop}
  Let \(X\) be a set and \(I\) a non-empty collection of \(\sigma\)-algebras on 
  \(X\). Then \(\bigcap I\) is also a \(\sigma\)-algebra on \(X\).
\end{prop}

This proposition is easy to check and thus, it makes sense to consider
the \(\sigma\)-algebra generated by some set.

\begin{definition}[Generator of \(\sigma\)-algebra]
  Let \(X\) be a set and \(S \subseteq \mathcal{P}(X)\) a collection of subsets 
  of \(X\). Then the \(\sigma\)-algebra generated by \(S\) is 
  \[
    \sigma(S) := \bigcap \{\mathcal{A} \supseteq S \mid \mathcal{A} 
      \text{ is a \(\sigma\)-algebra on \(X\)} \}
  \]
\end{definition}

By the fact that the power set of \(X\) is a \(\sigma\)-algebra
containing \(S\), we see that
\(\{\mathcal{A} \supseteq S \mid \mathcal{A} \text{ is a \(\sigma\)-algebra on \(X\)} \}\)
is non-empty and so for all \(S \subseteq \mathcal{P}(X)\),
\(\sigma(S)\) a (and the smallest) \(\sigma\)-algebra on \(X\).

With this, we can construct a commonly seen \(\sigma\)-algebra, the
Borel \(\sigma\)-algebra. Given some topological space \(X\), the Borel
\(\sigma\)-algebra on \(X\) is the \(\sigma\)-algebra generated by
\(\mathcal{T}_X\), i.e.~\(\mathcal{B}(X) = \sigma(\mathcal{T}_X)\). We
will most commonly work with the Borel \(\sigma\)-algebra on the real
numbers \(\mathcal{B}(\mathbb{R})\).

We call the ordered pair \((X, \mathcal{A})\) where \(\mathcal{A}\) is a
\(\sigma\)-algebra n \(X\) a \emph{measurable space}.

\begin{definition}[Measure]
  Given a measurable space \((X, \mathcal{A})\), a measure on this measurable 
  space \(\mu : \mathcal{A} \to [0, \infty]\) is a function such that 
  \begin{itemize}
    \item \(\mu(\varnothing) = 0\)
    \item for all disjoint sequence 
      \((A_n)_{n = 1}^\infty \subseteq \mathcal{A}\),
      \(\mu\left(\bigsqcup_n A_n\right) = \sum_n \mu(A_n)\)
  \end{itemize}
\end{definition}

With measures defined, we can add an additional restriction to create a
\emph{probability space}.

\begin{definition}[Probability Measure]
  Let \(\mu\) be a measure on the measurable space \((X, \mathcal{A})\), then 
  \(\mu\) is a probability measure if and only if \(\mu(X) = 1\). We then call 
  the order triplet \((X, \mathcal{A}, \mu)\) a probability space.
\end{definition}

To distinguish probability space from normal measure spaces, we will
often write \((\Omega, \mathcal{F}, \mathbb{P})\) to denote a
probability space. We will call \(\Omega\) the \emph{sample space},
\(\mathcal{F}\) the \emph{events} and for all \(A \in \mathcal{F}\),
\(\mathbb{P}(A)\) the \emph{probability} of the event \(A\).

\hypertarget{random-variables}{%
\subsection{Random variables}\label{random-variables}}

Now that we have the basic notion of a probability space, we would like
to play around with it using \emph{random variables}. In the most
general sense, random variables are simply functions from the
probability space to another measurable space, most commonly the real
numbers equipped with \(\mathcal{B}(\mathbb{R})\).

\begin{definition}[Measurable Functions]
  Let \((X, \mathcal{A})\) and \((Y, \mathcal{B})\) be two measurable spaces and 
  \(f : X \to Y\) a mapping between the two. We call \(f\) measurable if and 
  only if for all \(A \in \mathcal{B}\), \(f^{-1}(A) \in \mathcal{A}\).
\end{definition}

\begin{definition}[Random Variables]
  Let \((\Omega, \mathcal{F}, \mathbb{P})\) be a probability space and 
  \(E, \mathcal{A}\) be a measurable space. Then an \(E\)-valued random variable 
  is a measurable function \(X : \Omega \to E\).
\end{definition}

In general, we will only be working with real valued random variables,
so the image measurable space is
\((\mathbb{R}, \mathcal{B}(\mathbb{R}))\).

Often, when we have a random variable \(X : \Omega \to \mathbb{R}\), we
might ask questions such as ``what is the probability that \(X \in A\)''
for some \(A \subseteq \text{Im} X\). We now see that this question is
asking for exactly \(\mathbb{P}(X \in A) = \mathbb{P}(X^{-1}(A))\) (this
makes sense as \(X\) is measurable).

Another property that is useful for random variables is the notion of
independence.

\begin{definition}[Independence of Events]
  Let \((\Omega, \mathcal{F}, \mathbb{P})\) be a probability space and 
  \((A_n) \subseteq \mathcal{F}\) a sequence of events. Then \((A_n)\) is said 
  to be independent if and only if for all \textit{finite} index set \(I\),
  \[\mathbb{P}\left(\bigcap_{n \in I} A_n \right) 
    = \prod_{n \in I} \mathbb{P}(A_n).\] 
\end{definition}

\begin{definition}[Independence of \(\sigma\)-algebras]
  Let \((\Omega, \mathcal{F}, \mathbb{P})\) be a probability space and 
  \((\mathcal{A}_n)\) be a sequence of sub-\(\sigma\)-algebras of 
  \(\mathcal{F}\). Then \((\mathcal{A}_n)\) is said to be independent if and 
  only if for all \((A_n) \subseteq \mathcal{F}\) a sequence of events such that 
  \(A_i \in \mathcal{A}_i\), \((A_n)\) is independent.
\end{definition}

Equipped with these two notions of independence, it makes sense to
create a notion of some \(\sigma\)-algebra induced by arbitrary
measurable functions and with that the notion of independence of random
variables is also induced.

\begin{definition}[\(\sigma\)-algebra Generated by Functions]
  Let \(E\) be a set and \(\{f_i : E \to \mathbb{R} \mid i \in I\}\) be an 
  indexed family of real-valued functions. Then the \(\sigma\)-algebra on \(E\) 
  generated by these functions is 
  \[
    \sigma(\{f_i \mid i \in I\}) := 
    \sigma(\{f^{-1}(A) \mid A \in \mathcal{B}(\mathbb{R})\}).  
  \]
\end{definition}

Note that with this definition, we created the smallest
\(\sigma\)-algebra on \(E\) such that all \(f_i\) are measurable.

\begin{definition}[Independence of Random Variables]
  Let \((\Omega, \mathcal{F}, \mathbb{P})\) be a probability space and \((X_n)\) 
  be a sequence of real-valued random variables. Then \((X_n)\) is said to be 
  independent if and only if the family of \(\sigma\)-algebras \(\sigma(X_n)\) 
  is independent.
\end{definition}

We will check that this definition of independence of random variables
behave as intended, that is
\(\mathbb{P}(X \in A, Y \in B) = \mathbb{P}(X \in A) \mathbb{P}(Y \in B)\).

\begin{theorem}
  Let \((\Omega, \mathcal{F}, \mathbb{P})\) be a probability space and 
  \(X, Y\) be real-valued random variables. THen \(X, Y\) are independent if 
  and only if for all \(A, B \in \mathcal{B}(\mathbb{R})\), 
  \[\mathbb{P}(X \in A, Y \in B) = \mathbb{P}(X \in A) \mathbb{P}(Y \in B).\]
\end{theorem}
\proof

Recall that
\(\mathbb{P}(X \in A, Y \in B) =  \mathbb{P}((X \in A) \cap (Y \in B)) =  \mathbb{P}(X^{-1}(A) \cap Y^{-1}(B))\).
Now, if \(\sigma(X)\) and \(\sigma(Y)\) are independent, as
\(X^{-1}(A) \in \sigma(X)\) and \(Y^{-1}(B) \in \sigma(Y)\), by
definition, we have
\(\mathbb{P}(X \in A, Y \in B) = \mathbb{P}(X \in A)\mathbb{P}(Y \in B)\).

Similarly, if the equality in question is true for all
\(A, B \in \mathcal{B}(\mathbb{R})\), then the \(\sigma\)-algebras are
independent by definition, and thus, so are the random variables. \qed

\end{document}
