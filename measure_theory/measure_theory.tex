% Options for packages loaded elsewhere
\PassOptionsToPackage{unicode}{hyperref}
\PassOptionsToPackage{hyphens}{url}
\PassOptionsToPackage{dvipsnames,svgnames*,x11names*}{xcolor}
%
\documentclass[
]{article}
\usepackage{lmodern}
\usepackage{amssymb,amsmath}
\usepackage{ifxetex,ifluatex}
\ifnum 0\ifxetex 1\fi\ifluatex 1\fi=0 % if pdftex
  \usepackage[T1]{fontenc}
  \usepackage[utf8]{inputenc}
  \usepackage{textcomp} % provide euro and other symbols
\else % if luatex or xetex
  \usepackage{unicode-math}
  \defaultfontfeatures{Scale=MatchLowercase}
  \defaultfontfeatures[\rmfamily]{Ligatures=TeX,Scale=1}
\fi
% Use upquote if available, for straight quotes in verbatim environments
\IfFileExists{upquote.sty}{\usepackage{upquote}}{}
\IfFileExists{microtype.sty}{% use microtype if available
  \usepackage[]{microtype}
  \UseMicrotypeSet[protrusion]{basicmath} % disable protrusion for tt fonts
}{}
\makeatletter
\@ifundefined{KOMAClassName}{% if non-KOMA class
  \IfFileExists{parskip.sty}{%
    \usepackage{parskip}
  }{% else
    \setlength{\parindent}{0pt}
    \setlength{\parskip}{6pt plus 2pt minus 1pt}}
}{% if KOMA class
  \KOMAoptions{parskip=half}}
\makeatother
\usepackage{xcolor}
\IfFileExists{xurl.sty}{\usepackage{xurl}}{} % add URL line breaks if available
\IfFileExists{bookmark.sty}{\usepackage{bookmark}}{\usepackage{hyperref}}
\hypersetup{
  pdftitle={Lebesgue Measure \& Integration},
  pdfauthor={Kexing Ying},
  colorlinks=true,
  linkcolor=Maroon,
  filecolor=Maroon,
  citecolor=Blue,
  urlcolor=red,
  pdfcreator={LaTeX via pandoc}}
\urlstyle{same} % disable monospaced font for URLs
\usepackage[margin = 1.5in]{geometry}
\usepackage{graphicx}
\makeatletter
\def\maxwidth{\ifdim\Gin@nat@width>\linewidth\linewidth\else\Gin@nat@width\fi}
\def\maxheight{\ifdim\Gin@nat@height>\textheight\textheight\else\Gin@nat@height\fi}
\makeatother
% Scale images if necessary, so that they will not overflow the page
% margins by default, and it is still possible to overwrite the defaults
% using explicit options in \includegraphics[width, height, ...]{}
\setkeys{Gin}{width=\maxwidth,height=\maxheight,keepaspectratio}
% Set default figure placement to htbp
\makeatletter
\def\fps@figure{htbp}
\makeatother
\setlength{\emergencystretch}{3em} % prevent overfull lines
\providecommand{\tightlist}{%
  \setlength{\itemsep}{0pt}\setlength{\parskip}{0pt}}
\setcounter{secnumdepth}{5}
\usepackage{tikz}
\usepackage{amsthm}
\usepackage{mathtools}
\usepackage{lipsum}
\usepackage[ruled,vlined]{algorithm2e}
\theoremstyle{definition}
\newtheorem{theorem}{Theorem}
\newtheorem{prop}{Proposition}
\newtheorem{corollary}{Corollary}[theorem]
\newtheorem*{remark}{Remark}
\theoremstyle{definition}
\newtheorem{definition}{Definition}[section]
\newtheorem{lemma}{Lemma}[section]
\newcommand{\diag}{\mathop{\mathrm{diag}}}
\newcommand{\Arg}{\mathop{\mathrm{Arg}}}
\newcommand{\hess}{\mathop{\mathrm{Hess}}}

\title{Lebesgue Measure \& Integration}
\author{Kexing Ying}
\date{January 11, 2021}

\begin{document}
\maketitle

{
\hypersetup{linkcolor=}
\setcounter{tocdepth}{2}
\tableofcontents
}
\newpage

\hypertarget{motivation}{%
\section{Motivation}\label{motivation}}

We recall from \textbf{Analysis I} the definition of the Darboux
integral. While this notion of integration was sufficient for our use
case last year, as we shall see, there are some limitations with this
notion of integration. These limitations will be addressed by the means
of measure theory.

\begin{definition}[Darboux Integrable]
  A function \(f : [a, b] \to \mathbb{R}\) is called Darboux integrable if for 
  any partition \(\mathcal{P} = \{a = t_0 < t_1 < \cdots < t_{n - 1} < t_n = b \}\)
  for some \(n \ge 1\) if \([a, b]\), by defining the lower and upper Darboux sums, 
  \[L(f, \mathcal{P}) = \sum_{i = 1}^n (t_i - t_{i - 1}) \inf_{t \in [t_{i-1}, t_i]} f(t),\]
  and
  \[U(f, \mathcal{P}) = \sum_{i = 1}^n (t_i - t_{i - 1}) \sup_{t \in [t_{i-1}, t_i]} f(t),\]
  one has 
  \[\sup_{\mathcal{P}} L(f, \mathcal{P}) = \inf_{\mathcal{P}} U(f, \mathcal{P}).\]
  If this is the case we define the integral of \(f\) over \([a, b]\) to be this 
  value, i.e. 
  \[\int_a^b f := \sup_{\mathcal{P}} L(f, \mathcal{P}) = \inf_{\mathcal{P}} U(f, \mathcal{P}).\]
\end{definition}

Many functions are Darboux integrable and in fact, as demonstrated last
year, all functions in \(C_{pw}^\circ([a, b])\), that is piecewise
continuous functions on \([a, b]\) are Darboux integrable. Nonetheless,
however, the class of Darboux integrable functions is also rather
limited.

Consider the Dirichlet function \[\mathbf{1}_{\mathbb{Q}}(x) := 
  \begin{cases}
    1, \hspace{2mm} x \in \mathbb{Q};\\
    0, \hspace{2mm} x \in \mathbb{R}\setminus\mathbb{Q}.
  \end{cases}\] That is, the indicator function for \(\mathbb{Q}\). We
see that \(\mathbf{1}_\mathbb{Q}\) is not Darboux integrable since both
\(\mathbb{Q}\) and \(\mathbb{R}\setminus\mathbb{Q}\) are dense in
\(\mathbb{R}\) and so, for any partition \(\mathcal{P}\) of \([a, b]\),
\(L(\mathbf{1}_\mathbb{Q}, \mathcal{P}) = 0\) while
\(U(\mathbf{1}_\mathbb{Q}, \mathcal{P}) = 1\). This is not ideal, since,
as \(\mathbb{Q}\) is countable while \(\mathbb{R}\setminus\mathbb{Q}\)
is not, we intuitively expect that a satisfactory theory of integration
would assign \(\int_a^b \mathbf{1}_\mathbb{Q} = 0\).

Moreover, by defining
\(P = (q_n)_{n \in \mathbb{N}} \subseteq \mathbb{Q}\) be some
enumeration of \(\mathbb{Q} \cap [a, b]\), we can define the following
sequence of functions, \[f_n(x) := 
  \begin{cases}
    1, \hspace{2mm} x \in \{q_0, \cdots, q_n\};\\
    0, \hspace{2mm} \text{otherwise}. 
  \end{cases}\] It is not difficult to see that \(\int_a^b f_n = 0\) for
all \(n\) and \(f_n \to \mathbf{1}_\mathbb{Q}\) pointwise. However, this
implies
\[0 = \lim_{n \to \infty} \int_a^b f_n \neq \int_a^b \lim_{n \to \infty} f_n = 
  \int_a^b \mathbf{1}_\mathbb{Q},\] and in fact, the right hand side is
not even defined (as \(\mathbf{1}_\mathbb{Q}\) is not Darboux
integrable)!

To solve this issue we will introduce the notion of the Lebesgue measure
and furthermore, its associated Lebesgue integral which extends our
Darboux integral such that it has the ``nice'' properties we desire.

We will in this course also look at \(L^p\) spaces. From the perspective
of analysis, it is often convenient to work in Banach spaces (complete
normed vector spaces) such that we can utilise many existing theorems we
have proved in \textbf{Analysis II}, e.g.~Banach's fixed point theorem.
For instance, one can endow \(C_{pw}^\circ([a, b])\) with the
(semi-)norm \[\|f\|_{L^1} := \int_a^b \left| f \right|. \] Then, by
considering the aforementioned sequence
\((f_n) \subseteq C_{pw}^\circ([a, b])\), one can easily show that
\((f_n)\) is a Cauchy sequence with respect to \(\| \cdot \|_{L^1}\).
However, \(f_n \to \mathbf{1}_\mathbb{Q}\) pointwise. This motivates us
to introduce the Banach space \(L^1([a, b])\) of integrable functions,
and more generally, \(L^p\)-spaces later in the course.

Lastly, as we have seen within last term's probability module, measure
theory lays below as the foundations for probability theory. As a quick
reminder, we recall that a probability space is a special type of
measure space and random variables defined on these probability spaces
are simply measurable functions to \(\mathbb{R}\) (or more exotic
fields). This can be interpreted with connotations to real world
situations in several ways.

\newpage

\hypertarget{measures-and-measure-spaces}{%
\section{Measures and Measure
Spaces}\label{measures-and-measure-spaces}}

\hypertarget{measures-and-measure-spaces-1}{%
\subsection{Measures and Measure
Spaces}\label{measures-and-measure-spaces-1}}

As we would like an adequate theory to assign a notion of ``size'' on
sets, we need to construct a function from a set of sets to
\(\overline{\mathbb{R}^+_0}\). To achieve this, the natural idea is to
construct a function with domain being the power set, however, this is
not necessarily always possible or meaningful. Thus, instead of
assigning every subset of some set a size, we only look at some
collection of ``nice'' sets.

\begin{definition}[Algebra]
  Let \(X\) be some set and suppose we denote \(\mathcal{X}\) for the power set of \(X\),
  then a family \(\mathcal{A} \subseteq \mathcal{P}(X)\) is called an algebra 
  over \(X\) if 
  \begin{itemize}
    \item \(X \in \mathcal{A}\);
    \item for all \(A \in \mathcal{A}\), \(A^c = X \setminus A \in \mathcal{A}\);
    \item if \((A_k)_{i = 1}^n\) is a finite sequence of sets in \(\mathcal{A}\), 
      then \(\bigcup_{k = 1}^n A_k \in \mathcal{A}\).
  \end{itemize}
\end{definition}

\begin{definition}[\(\sigma\)-algebra]
  Let \(X\) be a set, then a \(\sigma\)-algebra \(\mathcal{A}\) on \(X\) is an 
  algebra on \(X\) such that \(\mathcal{A}\) is closed under countable unions, 
  i.e. if \((A_k)_{k = 1}^\infty\) is a sequence of sets in \(\mathcal{A}\), 
  then \(\bigcup_{k = 1}^\infty A_k \in \mathcal{A}\).
\end{definition}

As \(\sigma\)-algebras (and algebras) are simply sets of sets, there is
an induced order on \(\sigma\)-algebras by \(\subseteq\). If there are
two \(\sigma\)-algebras \(\mathcal{A}, \mathcal{B}\) such that
\(\mathcal{A} \subseteq \mathcal{B}\), then we say \(\mathcal{A}\) is
coarser than \(\mathcal{B}\).

Trivially, we find \(\{\varnothing, X\}\) is a \(\sigma\)-algebra.
Indeed, this is the coarsest \(\sigma\)-algebra. Furthermore, given a
set \(X\) and a subset \(A \subseteq X\), we have
\(\{\varnothing, A, A^c, X\}\) is also a \(\sigma\)-algebra.

While every \(\sigma\)-algebra is also an algebra, the converse is not
true. An counter-example of this is by consider the algebra
\[\mathcal{A} := \{\varnothing\} \cup \left\{ U \mid \exists \bigcup_{k = 1}^m (a_k, b_k], 
  m \ge 1, 0 \le a_k < b_k \le 1 \right\},\] on \(X = (0, 1]\). We see
that \(\mathcal{A}\) is an algebra (since
\((a, b]^c = (0, a] \cup (b, 0]\)) however \(\mathcal{A}\) is not a
\(\sigma\)-algebra on \(X\) since we can define the sequence
\(A_k = (0, 1 - 1 / k] \in \mathcal{A}\) but
\(\bigcup_k A_k = (0, 1) \not\in \mathcal{A}\).

\begin{prop}
  Let \(\mathcal{F}\) be an arbitrary collection of \(\sigma\)-algebra (or algebras) 
  over \(X\). Then the intersections 
  \[\bigcap \mathcal{F} := \bigcap_{\mathcal{A} \in \mathcal{F}} \mathcal{A},\]
  is a \(\sigma\)-algebra (or algebra).
\end{prop}
\proof

Straight forward by definition. \qed

With this, we have a notion of infimum on \(\sigma\)-algebras and hence,
we can also define a notion closure.

\begin{definition}[\(\sigma\)-algebra Generated by a set]
  Let \(\mathcal{C} \subseteq \mathcal{P}(X)\), then 
  \[\sigma(\mathcal{C}) := \bigcap \{\mathcal{A} \mid \mathcal{C} \subseteq \mathcal{A} 
    \wedge \mathcal{A} \in \mathcal{F}\},\]
  where \(\mathcal{F}\) is the set of all \(\sigma\)-algebras on \(X\).
\end{definition}

As previously shown, \(\sigma(\mathcal{C})\) is a intersection of
\(\sigma\)-algebras, and so the name suggests, the \(\sigma\)-algebra
generated by \(\mathcal{C}\) is the smallest \(\sigma\)-algebra
containing \(\mathcal{C}\). Indeed, we find
\(\sigma(\varnothing) = \{\varnothing, X\}\) and
\(\sigma(A) = \{\varnothing, A, A^c, X\}\). Moreover, we see that
\(\mathcal{C}\) is a \(\sigma\)-algebra if and only if
\(\sigma(\mathcal{C}) = \mathcal{C}\).

\begin{definition}[Borel \(\sigma\)-algebra]
  If \((X, \mathcal{T})\) is a topological space, then the Borel \(\sigma\)-algebra 
  over \(X\) is 
  \[B(X) := \sigma(\mathcal{T}).\]
\end{definition}

Unlike topologies, the unions and intersections in a \(\sigma\)-algebra
is treated symmetrically.

\begin{prop}
  If \(\mathcal{A}\) is a \(\sigma\)-algebra, then if \((A_k)_{k = 1}^\infty\) is 
  a sequence of sets in \(\mathcal{A}\), then 
  \[\bigcap_{k = 1}^\infty A_k \in \mathcal{A}.\]
\end{prop}
\proof

By considering de Morgen's identity, we have
\(\bigcap_{k = 1}^\infty A_k =  (\bigcup_{k = 1}^\infty A_k^c)^c\). So,
since \(\bigcup_{k = 1}^\infty A_k^c \in  \mathcal{A}\) as each
component is, \(\bigcup_{k = 1}^\infty A_k^c)^c\) and hence
\(\bigcap_{k = 1}^\infty A_k\) is also in \(\mathcal{A}\). \qed

\begin{definition}[Measurable Space]
  A set \(X\) equipped with a \(\sigma\)-algebra \(\mathcal{A}\) is called a 
  measurable space and is written as a tuple \((X, \mathcal{A})\). Furthermore, 
  if \(A \subseteq X\) is in \(\mathcal{A}\), then we say \(A\) is a measurable 
  set.
\end{definition}

\begin{definition}[Measure]
  Let \((X, \mathcal{A})\) is a measurable space. Then a measure on \((X, \mathcal{A})\) 
  is a function \(\mu : \mathcal{A} \to [0, \infty]\) such that 
  \begin{itemize}
    \item \(\mu(\varnothing) = 0\);
    \item if \((A_k)_{k = 1}^\infty \subseteq \mathcal{A}\) is a sequence of 
      pairwise disjoint sets, then 
      \[\mu\left(\bigcup_{k = 1}^\infty A_k\right) = \sum_{k = 1}^\infty \mu(A_k)\].
  \end{itemize}
  We call the second property \(\sigma\)-additivity. 
\end{definition}

\begin{definition}[Measure Space]
  A measurable space \((X, \mathcal{A})\) equipped with the measure \(\mu\) is 
  called a measure space and is written as a triplet \((X, \mathcal{A}, \mu)\).
\end{definition}

A commonly used measure on any arbitrary measurable space
\((X, \mathcal{A})\) is the counting measure \(\mu\). As the name
suggests, for all \(A \in \mathcal{A}\), \(\mu(A) = |A|\) if \(A\) is
finite and \(\infty\) otherwise. Another example of a measure is the
Dirac measure \(\delta_x : \mathcal{A} \to [0, \infty]\) for some
\(x \in X\) where for all \(A \in \mathcal{A}\), \(\delta_x(A) = 1\) if
\(x \in A\) and 0 otherwise. For the last example, let \(X\) be
uncountable and let
\(\mathcal{A} := \{A \subseteq X \mid A \text{ or } A^c \text{ is uncountable}\}\).
Then, one can show that \(\{X, \mathcal{A}\}\) forms a measurable space
and we find that the function \(\mu : \mathcal{A} \to [0, \infty]\)
defined as \(\mu(A) = 0\) if \(A\) is countable and \(\mu(A) = 1\) if
\(A^c\) is countable is a measure on \((X, \mathcal{A})\).

\begin{prop}
  Let \((X, \mathcal{A}, \mu)\) be a measure space. Then, 
  \begin{itemize}
    \item if \(A, B \in \mathcal{A}\) and \(A \subseteq B\), then \(\mu(A) \le \mu(B)\);
    \item if \(n \ge 1\), \((A_k)_{k = 1}^n\) is a sequence of pairwise disjoint sets 
      in \(\mathcal{A}\), then \[\mu\left(\bigcup_{k = 1}^n A_k\right) = 
        \sum_{k = 1}^n \mu (A_k);\]
    \item if \((A_k)_{k = 1}^\infty\) is a sequence of monotonically increasing 
      sets in \(\mathcal{A}\), then 
        \[\mu\left(\bigcup_{k = 1}^\infty A_k\right) = \lim_{k \to \infty}\mu(A_k).\]
  \end{itemize}
  We note that the limit in part 3 exists since the limit in monotonically increasing 
  on the extended reals (so bounded by \(\infty\)). 
  \begin{itemize}
    \item if \((A_k)_{k = 1}^\infty\) is a sequence of monotonically decreasing 
      sets in \(\mathcal{A}\), if \(\mu(A_1) < \infty\), then
      \[\mu\left(\bigcap_{k = 1}^\infty A_k\right) = \lim_{k \to \infty}\mu(A_k).\]
    \item if \(A \in \mathcal{A}\) and \((A_k)_{k = 1}^\infty\) is a sequence in 
      \(\mathcal{A}\), then \(\mu(A) \le \sum_{k = 1}^\infty \mu(A_k)\).
  \end{itemize}
  The last property is refereed to as \(\sigma\)-sub-additivity.
\end{prop}
\proof

Part 2 is trivial.

(Part 1) As \(B = A \sqcup B \setminus A = A \sqcup (A^c \cap B)\) where
\(A^c \cap B\) is measurable since both \(A^c\) and \(B\) are. So, by
\(\sigma\)-additivity,
\[\mu(B) = \mu(A \sqcup (A^c \cap B)) = \mu(A) + \mu(A^c \cap B) \ge \mu(A).\]
(Part 3) Define \(B_1 = A_1\) and
\(B_{k + 1} = A_{k + 1} \setminus A_k\). Then, \((B_k)_{k = 1}^\infty\)
is a sequence of disjoint subset in \(\mathcal{A}\). So, by
\(\sigma\)-additivity, \[\mu\left(\bigcup_{k = 1}^\infty A_k\right) = 
    \mu\left(\bigcup_{k = 1}^\infty B_k\right) = 
    \sum_{k = 1}^\infty \mu(B_k) = \lim_{n \to \infty} \sum_{k = 1}^n \mu(B_k)
    = \lim_{n \to \infty} \mu\left(\bigcup_{k = 1}^n B_k\right)= \lim_{n \to \infty} \mu(A_k). \]
(Part 4) Define \(B_k = A_1 \setminus A_k\), then
\((B_k)_{k = 1}^\infty\) is a sequence of monotonically increasing sets
in \(\mathcal{A}\). So, by part 3,
\[\mu\left(\bigcup_{k = 1}^\infty B_k \right) = \lim_{k \to \infty} \mu(B_l).\]
Furthermore, as \(A \subseteq B\) implies
\(\mu (B) - \mu(A) = \mu(B \setminus A)\), we have
\[\mu(A_1) - \mu \left(\bigcap_{k = 1}^\infty A_k\right) = 
    \mu\left(A_1 \setminus \bigcap_{k = 1}^\infty A_k\right) = 
    \mu\left(\bigcup_{k = 1}^\infty B_k \right) = \lim_{k \to \infty} \mu(B_k)
    = \mu(A_1) - \lim_{k \to \infty} \mu(A_k),\] hence the result.

(Part 5) Exercise. \qed

\hypertarget{hahn-carathuxe9odory-extension-theorem}{%
\subsection{Hahn-Carathéodory Extension
Theorem}\label{hahn-carathuxe9odory-extension-theorem}}

While it is fun to work with abstract measures in general, it is also
useful to consider specific measures. In this section we will look at
how to construct measures on arbitrary spaces. The general idea for
constructing measures is that, given a \emph{premeasure}
\(\tilde{\mu}\), we may extend it to an \emph{outer measure} \(\mu^*\),
and finally with the outer measure, we may restrict it to a measure
\(\mu\).

\begin{definition}[Premeasure]
  Let \(\mathcal{A}\) be an algebra on \(X\). A function \(\mu : \mathcal{A} \to [0, \infty]\) 
  is a premeasure if it satisfies \(\mu(\varnothing) = 0\) and countable additivity; 
  that is, given \((A_n)_{n = 1}^\infty \subseteq \mathcal{A}\) that is pairwise disjoint, 
  \[\mu\left(\bigcup_{n = 1}^\infty A_n\right) = \sum_{n = 1}^\infty \mu(A_n),\]
  provided \(\bigcup_{n = 1}^\infty A_n \in \mathcal{A}\).
\end{definition}
\begin{definition}[Outer Measure]
  A function \(\mu : \mathcal{P}(X) \to [0, \infty]\) is an outer measure if it 
  satisfies \(\mu(\varnothing) = 0\) and sub-additivity, that is, 
  if \((A_k)_{k = 1}^\infty\) is a sequence of subsets of \(X\), and 
  \(A \subseteq \bigcup A_k\) then 
  \[\mu(A) \le \sum_{k = 1}^\infty \mu(A_k).\]
\end{definition}

\begin{lemma}[Monoticity of Outer Measures]
  Let \(\mu : \mathcal{P}(X) \to [0, \infty]\) be an outer measure on \(X\). Then 
  for all \(A \subseteq B \subseteq X\), 
  \[\mu(A) \le \mu(B).\]
\end{lemma}
\proof

By defining the sequence \(A_1 = B\) and \(A_n = \varnothing\) for all
\(n > 1\), by the sub-additivity property of outer measure, we have
\(A \subseteq B = \bigcup A_n\), and so
\[\mu(A) \le \sum\mu(A_n) = \mu(B) + \mu(\varnothing) + \cdots = \mu(B).\]
\qed

In order to prove the aforementioned argument, we will require some
proposition. However, as will be demonstrated below, the notion of
premeasure is in fact, too strict. Thus, for that reason, let us
introduce the following definition.

\begin{definition}[Cover]
  A family of sets \(\mathcal{K} \subseteq \mathcal{P}(X)\) is called a cover 
  of \(X\) if \(\varnothing \in \mathcal{K}\) and there exists a sequence 
  \((K_n)_{n = 1}^\infty\) in \(\mathcal{K}\) such that 
  \[X \subseteq \bigcup_{n = 1}^\infty K_n.\]
\end{definition}
\begin{remark}
  We note that this is a different definition of covers than that we had learnt 
  for the definition compactness of topological spaces.
\end{remark}

Straight away, we see that every algebra \(\mathcal{A}\) of \(X\) is a
cover of \(X\) as \(X \in \mathcal{A}\) and so we can simply let the the
countable sequence be \(K_n = X\). Thus, every proposition we prove
about covers apply also to algebras.

\begin{prop}\label{cath_mea}
  Let \(\mathcal{K}\) be a cover of \(X\), and \(\tilde{\mu} : \mathcal{K} \to [0, \infty]\) 
  be a map such that \(\tilde{\mu}(\varnothing) = 0\). Then  
  \[\mu^* : \mathcal{P}(X) \to [0, \infty] := A \mapsto 
    \inf \left\{\sum_{j = 1}^\infty \tilde{\mu}(K_j) \mid (K_j)_{j = 1}^\infty 
    \subseteq \mathcal{K} \wedge A \subseteq \bigcup_{j = 1}^\infty K_j \right\}\]
  is an outer measure on \(X\).
\end{prop}
\proof

\(\mu^*\) is well defined since for all \(A \subseteq X\), by the
definition of covers, there exists some \((K_n)_{n = 1}^\infty\),
\(\bigcup K_n \supseteq X \supseteq A\), and so, we are taking the
infimum of a non-empty set. Clearly \(\mu^*(\varnothing) = 0\) since we
can choose \(K_n = \varnothing\) and so, it remains to show
sub-additivity.

Let \((A_n)_{n = 1}^\infty\) be a sequence of subsets of \(X\), and
suppose \(A \subseteq \bigcup A_n\). For each \(n \in \mathbb{N}\) and
\(\epsilon > 0\), by the definition of infimum, there exists some
\((K_{n, j})_{j = 1}^\infty \subseteq \mathcal{K}\) such that
\(A_n \subseteq \bigcup K_{n, j}\) and
\[\sum_{j = 1}^\infty \tilde{\mu}(K_{n, j}) < \mu^*(A_n) + 2^{-n} \epsilon.\]
Now, by the fact that \(A \subseteq \bigcup A_n\), we have
\(A \subseteq \bigcup_n \bigcup_j K_{n, j}\), and so, again by the
definition of \(\mu^*\), we have
\[\mu^*(A) \le \sum_{n = 1}^\infty \sum_{j = 1}^\infty\tilde{\mu}(K_{n, j})
    < \sum_{n = 1}^\infty (\mu^*(A_n) + 2^{-n}\epsilon) 
    = \sum_{n = 1}^\infty \mu^*(A_n) + \epsilon.\] As \(\epsilon > 0\)
was arbitrary, \[\mu^*(A) \le \sum_{n = 1}^\infty \mu^*(A_n),\] so
\(\mu^*\) is a outer measure. \qed

With that, we can naturally construct a outer measure from a premeasure
(and in fact, weaker notions suffices). This, outer measure however, in
some sense overshoots measures as it is defined on all subsets of \(X\),
and does not attain \(\sigma\)-additivity but only
\(\sigma\)-sub-additivity.

\begin{lemma}\label{cath_sig}
  Let \(\mu^*\) be an outer measure on \(X\), then 
  \[\Sigma := \{A \subseteq X \mid \mu^*(B) = \mu^*(B \cap A) + \mu^*(B \cap A^c)\}\]
  is a \(\sigma\)-algebra on \(X\).
\end{lemma}
\begin{remark}
  By sub-additivity, we see that \(\Sigma\) is equivalently defined by requiring 
  \(\mu^*(B) \ge \mu^*(B \cap A) + \mu^*(B \cap A^c)\).
\end{remark}
\proof

\(\varnothing \in \Sigma\) since for all \(B \subseteq X\),
\(B \cap \varnothing = \varnothing\) and \(B \cap \varnothing^c = B\)
and so \[\mu^*(B \cap \varnothing) + \mu^*(B \cap \varnothing^c) 
    = \mu^*(\varnothing) + \mu^*(B) = \mu^*(B).\] Let \(A \in \Sigma\),
then for all \(B \subseteq X\),
\[\mu^*(B) = \mu^*(B \cap A) + \mu^*(B \cap A^c) 
    = \mu^*(B \cap (A^c)^c) + \mu^*(B \cap A^c),\] and so,
\(A^c \in \Sigma\).

To show that \(\Sigma\) is closed under union, let us first show that
\(\Sigma\) is closed under finite intersections. Let
\(A_1, A_2 \in \Sigma\), then for all \(B \subseteq X\), \[\begin{split}
    \mu^*(B) & = \mu^*(B \cap A_1) + \mu^*(B \cap A_1^c)\\
      & = \mu^*(B \cap A_1 \cap A_2) + \mu^*(B \cap A_1 \cap A_2^c) + \mu^*(B \cap A_1^c)\\
      & = \mu^*(B \cap (A_1 \cap A_2)) + \mu^*(B \cap (A_1 \cap A_2)^c \cap A_1)
         + \mu^*(B \cap (A_1 \cap A_2)^c \cap A_1^c)\\
      & = \mu^*(B \cap (A_1 \cap A_2)) + \mu^*(B \cap (A_1 \cap A_2)^c),
  \end{split}\] and so \(A_1 \cap A_2 \in \Sigma\). Now, suppose
\((A_n)\) is a sequence of disjoint sets in \(\Sigma\), it suffices to
show \(\bigcup_{n = 1}^\infty A_n \in \Sigma\) (we can assume this since
given a sequence of sets, we may define another disjoint sequence such
that their union are equal). Since the sequence is pairwise disjoint, we
have \[\begin{split}
    \mu^*(B) & = \mu^*(B \cap A_1) + \mu^*(B \cap A_1^c)\\
      & = \mu^*(B \cap A_1) + \mu^*(B \cap A_1^c \cap A_2) + \mu^*(B \cap A_1^c \cap A_2^c).
  \end{split}\] Indeed, as \(A_1 \cap A_2 = \varnothing\),
\(A_1^c \cap A_2 = A_2\) and so, \[\begin{split}
    \mu^*(B) & = \mu^*(B \cap A_1) + \mu^*(B \cap A_2) + \mu^*(B \cap A_1^c \cap A_2^c)\\
      & = \cdots\\
      & = \sum_{j = 1}^n \mu^*(B \cap A_j) + \mu^*\left(B \cap \bigcap_{j = 1}^n A_j^c\right)\\
      & = \sum_{j = 1}^n \mu^*(B \cap A_j) + \mu^*\left(B \cap \overline{\bigcup_{j = 1}^n A_j}\right)
  \end{split}\] for all \(n \in \mathbb{N}\). By sub-additivity, we have
\[\mu^*\left(B \cap \bigcup_{j = 1}^n A_j\right) \le \sum_{j = 1}^n \mu^*(B \cap A_j),\]
and so, by taking \(n \to \infty\), we have
\[\mu^*(B) \ge \mu^*\left(B \cap \bigcup_{j = 1}^\infty A_j\right) + 
    \mu^*\left(B \cap \overline{\bigcup_{j = 1}^\infty A_j}\right),\]
implying \(\bigcup A_n \in \Sigma\). \qed

\begin{theorem}[Hahn-Carathéodory Extension Theorem]
  Let \(X\) be an arbitrary set, \(\mathcal{A}\) an algebra over \(X\) and 
  \(\tilde{\mu} : \mathcal{A} \to [0, \infty]\) a premeasure on \(X\). Then by 
  defining \(\mu^*\) as in proposition \ref{cath_mea}, \(\Sigma\) as in
  lemma \ref{cath_sig}, and by defining \(\mu = \mu^* \mid_\Sigma\), then,
  \begin{itemize}
    \item \((X, \Sigma, \mu)\) is a measure space;
    \item \(\mathcal{A} \subseteq \Sigma\);
    \item for all \(A \in \mathcal{A}\), \(\mu(A) := \mu^*(A) = \tilde{\mu}(A)\).
  \end{itemize}
\end{theorem}
\proof

By construction, we have \(\mu^*\) is an outer measure on \(X\) and
\(\Sigma\) is a \(\sigma\)-algebra on \(X\), and so, for the first part
of the proof, it remains to show that \(\mu = \mu^*\mid_\Sigma\) is a
measure on the measurable space \((X, \Sigma)\).

Indeed, \(\mu(\varnothing) = 0\) as \(\mu^*(\varnothing) = 0\) since it
is an outer measure, so let us consider \(\sigma\)-additivity. Let
\(A, B \in \Sigma\), \(A \cap B = \varnothing\), then \[\begin{split}
    \mu(A \cup B) & = \mu^*(A \cup B) = \mu^*((A \cup B) \cap A) + \mu^*((A \cup B) \cap A^C) \\
      & = \mu^*(A) + \mu^*(B),
  \end{split}\] and so, by induction \(\mu\) is finitely additive. So,
if \((A_n)_{n = 1}^\infty\) is a sequence of pairwise disjoint
measurable sets, then, for all \(m \in \mathbb{N}\),
\[\sum_{n = 1}^m \mu(A_n) = \mu\left(\bigcup_{n = 1}^m A_n\right) 
    \le \mu\left(\bigcup_{n = 1}^\infty A_n\right),\] where the
inequality is due to the monotonicity of outer measures. So, by taking
\(m \to \infty\) we have
\[\sum_{n = 1}^\infty \mu(A_n) \le \mu\left(\bigcup_{n = 1}^\infty A_n\right).\]
However, by the sub-additivity of outer measures,
\[\sum_{n = 1}^\infty \mu(A_n) \ge \mu\left(\bigcup_{n = 1}^\infty A_n\right),\]
and so \(\sum \mu(A_n) \ge \mu\left(\bigcup A_n\right)\) resulting in
\(\mu\) being a measure.

For the second part of the theorem, it suffices to show that for all
\(A \in \mathcal{A}\), for all \(B \subseteq X\),
\[\mu^*(B) \ge \mu^*(B \cap A) + \mu^*(B \cap A^c).\] By recalling the
definition of \(\mu^*\), for all \(\epsilon > 0\), there exists some
sequence \((K_j)_{j = 1}^\infty \subseteq \mathcal{A}\),
\(B \subseteq \bigcup K_j\),
\[\sum_{j = 1}^\infty \tilde{\mu}(K_j) \le \mu^*(B) + \epsilon.\] Now,
as \(B \subseteq \bigcup K_j\),
\(B \cap A \subseteq \bigcup (K_j \cap A)\) and
\(B \cap A^c \subseteq \bigcup (K_j \cap A^c)\), and so, by
sub-additivity \[\begin{split}
    \mu^*(B \cap A) + \mu^*(B \cap A^c) 
      & \le \sum \mu^*(K_j \cap A) + \sum \mu^* (K_j \cap A^c)\\
      & = \sum (\tilde{\mu} (K_j \cap A) + \tilde{\mu} (K_j \cap A^c))\\
      & = \sum \tilde{\mu}(K_j) \le \mu^*(B) + \epsilon,
  \end{split}\] where we used the last part of the theorem to exchange
\(\mu^*\) for \(\tilde{\mu}\). Thus, as \(\epsilon > 0\) was arbitrary,
the inequality follows.

Lastly, to show that \(\mu^*\) agrees with \(\tilde{\mu}\) on
\(\mathcal{A}\), let \(A \in \mathcal{A}\). By the monotonicity of outer
measures, we have \(\mu^*(A) \ge \tilde{\mu}(A)\) and so, it suffices to
show there exists some \((K_j) \subseteq \mathcal{A}\) such that
\(A \subseteq \bigcup K_j\) and
\(\sum \tilde{\mu}(K_j) = \tilde{\mu}(A)\). But, this is trivial by
simply choosing \(A_1 = A\) and \(A_n = \varnothing\) for all \(n > 1\)
and so, we are done! \qed

With the Hahn-Carathéodory extension theorem, we can obtain a measure
(and also a measurable space), just from a premeasure on some algebra.
This is a very useful theorem as algebras and premeasures are rather
weak notions and can be easily constructed through many means. However,
this theorem does not yet guarantee the uniqueness, so one might ask
whether or not there are other extensions. We shall take a look at that
now.

\begin{definition}[\(\sigma\)-Finite]
  A premeasure \(\tilde{\mu}\) on the algebra \(\mathcal{A}\) on \(X\) is \(\sigma\)-finite 
  if there exists disjoint sets \((S_n)_{n = 1}^\infty \subseteq \mathcal{A}\) such 
  that \(X = \bigcup_{n = 1}^\infty S_n\) and \(\tilde{\mu}(S_n) < \infty\) for all 
  \(n \ge 1\). 
\end{definition}

\begin{theorem}[Uniqueness of the Hahn-Carathéodory Extension]
  Under the assumption of the Hahn-Carathéodory extension theorem, and if 
  \(\tilde{\mu}\) is \(\sigma\)-finite. Then, for all outer measures 
  \(\nu : \mathcal{P}(X) \to [0, \infty]\), 
  \[\nu\mid_\Sigma = \mu,\]
  where \(\mu\) is the outer measure obtained from the Hahn-Carathéodory extension.
\end{theorem}
\proof

Let \(A \in \Sigma\), and we will first show \(\nu(A) \le \mu(A)\). Let
\((A_n)_{n = 1}^\infty \subseteq \mathcal{A}\) be some sequence such
that \(A \subseteq \bigcup A_n\). Then, by the sub-additivity of
\(\nu\), \[\nu(A) \le \sum \nu(A_n) = \sum \tilde{\mu}(A_n).\] So, by
taking the infimum over all possible \((A_n)\), the inequality is
achieved.

To show the reverse inequality, let us suppose \((*)\) that there exists
some \(S \in \mathcal{A}\) such that \(A \subseteq S\) and
\(\mu(S) < \infty\). Then,
\[\nu(A) + \nu(S \cap A^c) \le \mu(A) + \mu(S \cap A^c) = \mu(S) = \tilde{\mu}(S) = \nu(S).\]
Now, by sub-additivity, \[\nu(S) \le \nu(A) + \nu(S \cap A^c),\] and so,
\(\nu(A) + \nu(S \cap A^c) \ge \mu(A) + \mu(S \cap A^c)\),
hence\footnote{
    Wlog. \(\mu(A) < \infty\) since if otherwise, \(\infty < \nu(A) + \nu(S \cap A^c)\). 
    Since \(\nu(S \cap A^c) \le \nu(S) \le \mu(S) < \infty\), \(\nu(A) = \infty\) 
    and hence \(\nu(A) \ge \mu(A)\).},
\[\nu(A) - \mu(A) \ge \mu(S \cap A^c) - \nu(S \cap A^c) \ge 0,\] as
\(\mu \ge \nu\) as shown previously. Thus, \(\nu(A) \ge \mu(A)\) and so,
\(\nu\mid_\Sigma = \mu\).

Now, to relax the assumption that such \(S\) exists, we shall use the
\(\sigma\)-finiteness of \(\tilde{\mu}\). Let
\((S_n) \subseteq \mathcal{A}\) be the sequence as described by the
\(\sigma\)-finiteness of \(\tilde{\mu}\), and define
\(A_n = A \cap S_n\). Then, by construction, the sets \(A_n\) are
pairwise disjoint and \(A = \bigcup A_n\). Since \(\mu = \nu\) on sets,
which \((*)\) is satisfied, we have
\[\mu\left(\bigcup_{n = 1}^m A_n\right) = \nu\left(\bigcup_{n = 1}^m A_n\right),\]
for all \(m \ge 1\) (since we can choose \(S = \bigcup_{n = 1}^m S_n\)).
Now, by monotonicity,
\[\nu(A) = \nu\left(\bigcup_{n = 1}^\infty A_n\right) \ge \nu\left(\bigcup_{n = 1}^m A_n\right)
    = \mu\left(\bigcup_{n = 1}^m A_n\right),\] for all \(m\), the
inequality is achieved by taking \(m \to \infty\). \qed

\hypertarget{the-lebesgue-measure}{%
\subsection{The Lebesgue Measure}\label{the-lebesgue-measure}}

With the Hahn-Carathéodory extension theorem in mind, we may construct
the Lebesgue measure on Euclidean spaces. As with the proof of the
extension theorem, we shall construct a premeasure assigning \(b - a\)
to every interval \((a, b)\) and then, extend that to a general measure.

\begin{definition}
  For \(a = (a_1, \cdots, a_n) \in \mathbb{R}^n\), \(b = (b_1, \cdots, b_n) \in \mathbb{R}^n\),
  we define the interval between \(a\) and \(b\) to be the set 
  \[(a, b) := \prod_{k = 1}^n(a_k, b_k) = 
    \{(x_1, \cdots, x_n) \mid a_k < x_k < b_k \hspace{2mm} 1 \le k \le n \},\]
  if \(a_k < b_k\) for all \(k\) and \((a, b) = \varnothing\) otherwise.
\end{definition}

Similarly, we define the half open and closed intervals in
\(\mathbb{R}^n\) with \((a, b] := \prod_{k = 1}^n(a_k, b_k]\),
\([a, b) := \prod_{k = 1}^n[a_k, b_k)\) and
\([a, b] := \prod_{k = 1}^n[a_k, b_k]\). Indeed, we also allow
\(\pm \infty\) as an endpoint and we shall from this point forward refer
to such sets as intervals in \(\mathbb{R}^n\).

\begin{definition}[Elementary Figure]
  A set \(I \subseteq \mathbb{R}^n\) is an elementary figure if its the union 
  of finitely many disjoint intervals; and in this section, we shall denote 
  \(\mathcal{A} \subseteq \mathcal{P(}\mathbb{R}^n)\) for the set of elementary 
  figures in \(\mathbb{R}^n\).
\end{definition}

We see that \(\mathcal{A}\) is an algebra on \(X = \mathbb{R}^n\) and
so, we may define a premeasure on \(\mathcal{A}\). Let
\(\tilde{\lambda} : \mathcal{A} \to [0, \infty]\) be the function such
that for all \(a, b \in \mathbb{R}^n\),
\[\tilde{\lambda}((a, b)) = \tilde{\lambda}((a, b]) = 
  \tilde{\lambda}([a, b)) = \tilde{\lambda}([a, b]) := \prod_{k = 1}^n(b_k - a_k),\]
if \(a_k < b_k\) and 0 otherwise. Furthermore, for
\(\bigcup_{k = 1}^m I_k \in \mathcal{A}\),
\[\tilde{\lambda}\left(\bigcup_{k = 1}^m I_k \right) = \sum_{k = 1}^m \tilde{\lambda}(I_k),\]
where \(I_1, \cdots, I_k\) are disjoint intervals.

\begin{lemma}
  The map \(\tilde{\lambda}\) is a premeasure on \(\mathcal{A}\).
\end{lemma}
\proof

As it is trivially true that \(\tilde{\lambda}(\varnothing) = 0\) to
show that \(\tilde{\lambda}\) is a premeasure, it suffices to show
\(\sigma\)-additivity within \(\mathcal{A}\).

We note that \(\tilde{\lambda}\) is finitely additive by definition,
hence monotone on \(\mathcal{A}\), and therefore if
\(I = \bigcup_{k = 1}^\infty\) where \(I_k\) are disjoint intervals,
\[\tilde{\lambda}(I) \ge \tilde{\lambda}\left(\bigcup_{k = 1}^m I_k\right) = 
    \sum_{k = 1}^m\tilde{\lambda}(I_k),\] for all \(m \in \mathbb{N}\).
Thus, by taking \(m \to \infty\),
\[\tilde{\lambda}(I) \ge \sum_{k = 1}^\infty \tilde{\lambda}(I_k).\] For
the reverse inequality, we use the what is called a \emph{compactness
argument} to reduce to finite additivity. Wlog. we may assume
\(\sum \tilde{\lambda}(I_k) < \infty\) (since if otherwise the reverse
inequality is trivial) and that \(I\) is a single interval with end
points \(a, b\) (since if the inequality is true for a single interval,
it is also true for the sum of finitely many intervals).

Let \(\bar{I}\) be the closure of \(I\) and for all \(L > 0\) we define
\(\bar{I}_L := \bar{I} \cap [-L, L]^n\). By Heine-Borel, \(\bar{I}_L\)
is compact and moreover, by taking \(L \to \infty\),
\[\tilde\lambda(\bar{I}_L) \to \tilde\lambda(\bar{I}) = \tilde\lambda(I).\]
Now, for all intervals \(J\) with end points \(\alpha, \beta\), we
define the interval \(J^\epsilon \supseteq J\) with endpoints
\(\alpha^\epsilon \neq \alpha\), \(\beta^\epsilon \neq \beta\) such that
\[\tilde\lambda(J^\epsilon) \le (1 + \epsilon)^n \tilde\lambda(J).\]
Lastly, for all \(k \in \mathbb{N}\), we define the open intervals
\(\tilde I_k\) with \(\tilde I_k \supseteq I_k^\epsilon\) satisfying
\[\tilde\lambda(\tilde I_k) < (1 + \epsilon)^n \lambda(I_k) + \epsilon 2^{-k}.\]
So
\(\bar{I}_L \subseteq \bar{I} \subseteq \bigcup_{k = 1}^\infty I_k^\epsilon  \subseteq \bigcup_{k = 1}^\infty \tilde I_k\),
and \(\{\tilde I_k\}\) forms an open cover of \(\bar{I}_L\), and hence,
by compactness, there exists a finite subcover for \(\bar{I}_L\), that
is, there exists some \(m\) such that (by reordering),
\(\bar{I}_L \subseteq \bigcup_{k = 1}^m I_k\). It follows that
\[\tilde\lambda(\bar{I}_L) \le \tilde\lambda\left(\bigcup_{k = 1}^m \tilde I_k\right) 
    \le \sum_{k = 1}^m \tilde\lambda(\tilde I_k) 
    \le (1 + \epsilon)^n \sum_{k = 1}^\infty \lambda(I_k) + \epsilon.\]
Thus, by taking \(\epsilon \to 0\) and then \(L \to \infty\) we have the
reverse inequality and so \(\tilde\lambda\) is a premeasure on
\(\mathcal{A}\). \qed

With this lemma, one can immediately apply the Hahn-Carathéodory
extension theorem resulting in the Lebesgue measure on Euclidean spaces,
and furthermore, by considering
\(\tilde\lambda((z, z + 1]) = 1 < \infty\), and
\(\bigcup_{z \in \mathbb{Z}^n}(z, z + 1] = \mathbb{R}^n\), we have
\(\tilde\lambda\) is \(\sigma\)-finite, and hence, the Lebesgue measure
is unique.

\begin{lemma}
  Let \(\mathcal{B}(\mathbb{R}^n)\) be the Borel \(\sigma\)-algebra on \(\mathbb{R}^n\), 
  then \(\mathcal{B}(\mathbb{R}^n) \subseteq \Sigma\) where \(\Sigma\) is the 
  \(\sigma\)-algebra induced by the Lebesgue measure.
\end{lemma}
\proof

Since \(\mathcal{B}(\mathbb{R}^n)\) is the \(\sigma\)-algebra generated
by the set of open sets of \(\mathbb{R}^n\), it suffices to show that
for all open sets \(O \subseteq \mathbb{R}^n\),
\(O \in \sigma(\mathcal{A}) \subseteq \Sigma\).

For all \(m \in \mathbb{N}\), define
\[C_m := \{[z, z + 2^{-m}) \mid z \in 2^{-m}\mathbb{Z}^n\} \subseteq \mathcal{A},\]
that is the grid of half open cubes covering \(\mathbb{R}^n\) with
individual cubes having length \(2^{-m}\). Then, by letting
\(C_m' := \{U \in C_m \mid U \subseteq O\}\), we have
\(C = \bigcup_{m \in \mathbb{N}} C_m'\) which is a countable set of half
open cubes. Now, we see that \(C = O\) since \(C \subseteq O\) trivially
and for all \(o \in O\), as \(O\) is open, there exists some
\(\epsilon > 0\) such that \(B_\epsilon(o) \subseteq O\) and so, \(o\)
is contained in one of the cubes with length \(< 2^{-m}\) where
\(m > 2 / \epsilon\) and so
\(O \in \sigma(\mathcal{A}) \subseteq \Sigma\). \qed

With that, we see that all Borel sets in \(\mathbb{R}^n\) are Lebesgue
measurable and restricting \(\tilde\lambda\) onto
\(\mathcal{B}(\mathbb{R}^n)\), we have the following measure space on
\(\mathbb{R}^n\).

\begin{definition}
  The Lebesgue measure \(\lambda : \mathcal{B}(\mathbb{R}^n) \to [0, \infty]\) is 
  the Hahn-Carathéodory extension of \(\tilde\lambda\) restricted onto 
  \(\mathcal{B}(\mathbb{R}^n)\).
\end{definition}

We note that, if \((X, \mathcal{F}, \mu)\) is a measure space and
\(A \in \mathcal{F}\), then by defining
\(\mathcal{F}\mid_A := \{A \cap B \mid B \in \mathcal{F}\}\) and
\(\mu\mid_A(B) := \mu(B)\) for all \(B \in \mathcal{F}\),
\(B \subseteq A\), it is easy to see that \(\mathcal{F}\mid_A\) is a
\(\sigma\)-algebra on \(A\) and \(\mu\mid_A\) is a measure on
\((A, \mathcal{F}\mid_A)\) and is called the restriction of \(\mu\) to
\(A\). Indeed, with this in mind, we see that the Lebesgue measure can
be restricted on small sets such as intervals; in particular, by
restricting the Lebesgue measure on \([0, 1]\), the resulting measure
\(\lambda\mid_{[0, 1]}\) is a probability measure.

\hypertarget{lebesgue-measurable-sets}{%
\subsection{Lebesgue Measurable Sets}\label{lebesgue-measurable-sets}}

We investigate which real sets are Lebesgue measurable.

\begin{prop}
  For all \(A \in \mathcal{B}(\mathbb{R}^n)\), 
  \[\lambda(A) = \inf_{G \supseteq A; G \text{open}} \lambda(G).\]
\end{prop}
\proof

Since measures are monotone, \((\le)\) is established. By recalling the
construction of the Lebesgue measure, and by the properties of \(\inf\),
for all \(\epsilon > 0\), there exists some
\((K_n)_{n = 1}^\infty \subseteq \mathcal{A}\), such that
\(A \subseteq \bigcup K_n\) and
\[\lambda(A) + \epsilon = \lambda^*(A) + \epsilon \ge 
    \sum_{n = 1}^\infty \tilde\lambda (K_n) \ge 
    \tilde\lambda\left(\bigcup K_n\right) = \lambda\left(\bigcup K_n\right)
    \ge \inf_{G \supseteq A; G \text{open}} \lambda(G).\] since the
union of open sets is open, \(\bigcup K_n\) is open and contains \(A\).
So, as \(\epsilon > 0\) was arbitrary, \((\ge)\) is established. \qed

By inspection of the proof, we find the statement to be true for any
real sets provided with change \(\lambda\) to the outer measure
\(\lambda^*\) on the left hand side. This \emph{regularity} of
measurable sets is expressed in the following.

\begin{prop}\label{exist_open_near}
  If \(A \in \mathcal{B}(\mathbb{R}^n)\), then for all \(\epsilon > 0\), there 
  exists some \(G \supseteq A\), \(G\) open such that \(\lambda(G \setminus A) < \epsilon\).
\end{prop}
\proof

One can of course prove it using the construction of the Lebesgue
measure, however, a stronger proposition holds. \qed

\begin{prop}
  Let \(\mathcal{A}\) be an algebra and \(\mu\) an measure on \(\sigma(\mathcal{A})\) 
  which is \(\sigma\)-finite on \(\mathcal{A}\). Then, for all \(A \in \sigma(\mathcal{A})\), 
  and \(\epsilon > 0\), there exist disjoint sets \((A_n)_{n = 1}^\infty \in \mathcal{A}\) 
  such that \(A \subseteq \bigcup A_n\) and 
  \(\mu\left(\bigcup A_n \setminus A\right) < \epsilon\).
\end{prop}
\proof

See problem sheet 4. \qed

Furthermore, by applying proposition \ref{exist_open_near} to \(A^c\)
for some \(A \in \mathcal{B}(\mathbb{R}^n)\), there exists some
\(\tilde G\) open, containing \(A^c\) such that
\(\lambda(\tilde G \setminus A^c) < \epsilon\) for all \(\epsilon > 0\).
Then, by defining \(F := \tilde G^c\), we have \(F \subseteq A\) closed,
such that
\[\lambda(A \setminus F) = \lambda(A \cap F^c) = \lambda (A \cap \tilde G) 
  = \lambda(\tilde G \setminus A^c) < \epsilon.\] Thus,
\[\lambda(G \setminus F) = 
  (\lambda(G) - \lambda(A)) + (\lambda(A) - \lambda(F)) 
  = \lambda(G \setminus A) + \lambda(A \setminus F) < 2 \epsilon.\] So,
for all \(A \in \mathcal{B}(\mathbb{R}^n)\), there exists some open
\(G \supseteq A\) and some closed \(F \subseteq A\) such that
\(\lambda(G \setminus F) < \epsilon\) for all \(\epsilon > 0\).

\begin{prop}[Transitional Invariance of \(\lambda\)]
  Let \(\Phi_{x_0} : \mathbb{R}^n \to \mathbb{R}^n : x \mapsto x + x_0\) for 
  some \(x_0 \in \mathbb{R}^n\). Then, 
  \[\lambda(\Phi_{x_0}(A)) = \lambda(A),\]
  for all \(A \in \mathcal{B}(\mathbb{R}^n)\).
\end{prop}
\proof

Since, for all \(A \in \mathcal{B}(\mathbb{R}^n)\), there exists some
open \(G \supseteq A\) such that \(\lambda(G) - \lambda(A) < \epsilon\),
it suffices to show transitional invariance for open sets. Now, as shown
previously, any open sets in \(\mathbb{R}^n\) can be written as a
disjoint union of countable intervals, the result follows since Lebesgue
measures are transitional invariant on intervals by definition. \qed

\begin{prop}
  \(\mathcal{B}(\mathbb{R}) \neq \mathcal{P}(\mathbb{R})\).
\end{prop}
\proof

As we have seen last term, the Vitali set is a classical example of a
non-measurable real set.

Define the equivalence relation \(\sim\) such that, for all
\(x, y \in (0, 1]\), \(x \sim y \iff x - y \in \mathbb{Q}\). Then, with
axiom of choice, for each equivalence class
\([x] \in (0, 1] \setminus \sim\), we choose exactly one \(v \in [x]\)
and we define the Vitali set \(V\) to be the set of these choices.

Now, let \(A := \mathbb{Q} \cap (-1, 1]\) and then,
\[(0, 1] \subseteq \bigcup_{q \in A} (q + V) \subseteq [-1, 2],\] where
the first inclusion is true since, for all \(x \in (0, 1]\), \(x\)
belongs to an equivalence class \([x']\) where \(x' \in V\). Now, as
\(x \in [x']\) implies \(x - x' \in \mathbb{Q}\), we can simply choose
\(q = x - x'\) and so \(x = q + x' \in \bigcup q + V\).

Thus, if \(V\) is measurable, then
\(1 \le \lambda(\bigcup(q + V)) \le 3\). Now, by observing that for all
\(p, q \in \mathbb{Q}\), if \(p \neq q\) then
\(p + V \cap q + V = \varnothing\),
\[\lambda\left(\bigcup_{q \in A}(q + V)\right) = \sum_{q \in A} \lambda(q + V)
    = \sum_{q \in A} \lambda(V),\] since \(\lambda\) is transitional
invariant. However, as \(\lambda(V)\) is bounded above by 3,
\(\lambda(V) = 0\) \# as \(\lambda(V) \ge 1\).

So \(V \not\in \mathcal{B}(\mathbb{R})\) and hence
\(\mathcal{B}(\mathbb{R}) \neq \mathcal{P}(\mathbb{R})\). \qed

\end{document}
