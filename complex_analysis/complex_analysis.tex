% Options for packages loaded elsewhere
\PassOptionsToPackage{unicode}{hyperref}
\PassOptionsToPackage{hyphens}{url}
\PassOptionsToPackage{dvipsnames,svgnames*,x11names*}{xcolor}
%
\documentclass[
]{article}
\usepackage{lmodern}
\usepackage{amssymb,amsmath}
\usepackage{ifxetex,ifluatex}
\ifnum 0\ifxetex 1\fi\ifluatex 1\fi=0 % if pdftex
  \usepackage[T1]{fontenc}
  \usepackage[utf8]{inputenc}
  \usepackage{textcomp} % provide euro and other symbols
\else % if luatex or xetex
  \usepackage{unicode-math}
  \defaultfontfeatures{Scale=MatchLowercase}
  \defaultfontfeatures[\rmfamily]{Ligatures=TeX,Scale=1}
\fi
% Use upquote if available, for straight quotes in verbatim environments
\IfFileExists{upquote.sty}{\usepackage{upquote}}{}
\IfFileExists{microtype.sty}{% use microtype if available
  \usepackage[]{microtype}
  \UseMicrotypeSet[protrusion]{basicmath} % disable protrusion for tt fonts
}{}
\makeatletter
\@ifundefined{KOMAClassName}{% if non-KOMA class
  \IfFileExists{parskip.sty}{%
    \usepackage{parskip}
  }{% else
    \setlength{\parindent}{0pt}
    \setlength{\parskip}{6pt plus 2pt minus 1pt}}
}{% if KOMA class
  \KOMAoptions{parskip=half}}
\makeatother
\usepackage{xcolor}
\IfFileExists{xurl.sty}{\usepackage{xurl}}{} % add URL line breaks if available
\IfFileExists{bookmark.sty}{\usepackage{bookmark}}{\usepackage{hyperref}}
\hypersetup{
  pdftitle={Comple Analysis},
  pdfauthor={Kexing Ying},
  colorlinks=true,
  linkcolor=Maroon,
  filecolor=Maroon,
  citecolor=Blue,
  urlcolor=red,
  pdfcreator={LaTeX via pandoc}}
\urlstyle{same} % disable monospaced font for URLs
\usepackage[margin = 1.5in]{geometry}
\usepackage{graphicx}
\makeatletter
\def\maxwidth{\ifdim\Gin@nat@width>\linewidth\linewidth\else\Gin@nat@width\fi}
\def\maxheight{\ifdim\Gin@nat@height>\textheight\textheight\else\Gin@nat@height\fi}
\makeatother
% Scale images if necessary, so that they will not overflow the page
% margins by default, and it is still possible to overwrite the defaults
% using explicit options in \includegraphics[width, height, ...]{}
\setkeys{Gin}{width=\maxwidth,height=\maxheight,keepaspectratio}
% Set default figure placement to htbp
\makeatletter
\def\fps@figure{htbp}
\makeatother
\setlength{\emergencystretch}{3em} % prevent overfull lines
\providecommand{\tightlist}{%
  \setlength{\itemsep}{0pt}\setlength{\parskip}{0pt}}
\setcounter{secnumdepth}{5}
\usepackage{tikz}
\usepackage{amsthm}
\usepackage{mathtools}
\usepackage{lipsum}
\usepackage[ruled,vlined]{algorithm2e}
\theoremstyle{definition}
\newtheorem{theorem}{Theorem}
\newtheorem{prop}{Proposition}
\newtheorem{corollary}{Corollary}[theorem]
\newtheorem*{remark}{Remark}
\theoremstyle{definition}
\newtheorem{definition}{Definition}[section]
\newtheorem{lemma}{Lemma}[section]
\newcommand{\diag}{\mathop{\mathrm{diag}}}
\newcommand{\Arg}{\mathop{\mathrm{Arg}}}
\newcommand{\hess}{\mathop{\mathrm{Hess}}}

\title{Comple Analysis}
\author{Kexing Ying}
\date{January 11, 2021}

\begin{document}
\maketitle

{
\hypersetup{linkcolor=}
\setcounter{tocdepth}{2}
\tableofcontents
}
\newpage

\hypertarget{complex-numbers}{%
\section{Complex Numbers}\label{complex-numbers}}

We recall some properties about the complex numbers \(\mathbb{C}\).

From \textbf{Analysis II} we recall the topological properties of
\(\mathbb{R}^2\). As there exists a natural homeomorphism from
\(\mathbb{R}^2\) to \(\mathbb{C}\), we conclude that the complex numbers
also has these properties.

\begin{prop}
  The set of complex numbers \(\mathbb{C}\) forms a metric space with the induced 
  metric from the Pythagorean norm, that is, the metric
  \[d : \mathbb{C} \times \mathbb{C} \to \mathbb{R} : (z, w) \mapsto \left| z - w \right|.\]
\end{prop}
\proof

One can trivially show that the Pythagorean norm is a norm on
\(\mathbb{C}\), and hence, the induced metric is a metric on
\(\mathbb{C}\). \qed

\begin{theorem}
  The complex numbers equipped with the distance as defined above is homeomorphic 
  to \(\mathbb{R}^2\) equipped with Euclidean metric.
\end{theorem}
\proof

This is true following the homeomorphism \((x, y) \mapsto x + iy\). \qed

\begin{corollary}
  The complex numbers is complete and a subset of \(\mathbb{C}\) is compact if 
  and only if \(\mathbb{C}\) is closed and bounded.
\end{corollary}
\proof

Follows from the Heine-Borel theorem and the fact that \(\mathbb{R}^2\)
is complete. \qed

Certain definitions are also induced for the complex numbers by the fact
that it is a metric space. We shall define them here again for
referencing.

\begin{definition}
  An open disk (ball) in \(\mathbb{C}\) centred at \(z_0 \in \mathbb{C}\) with 
  radius \(r > 0\) is the set 
  \[D_r(z_0) := \{ z \in \mathbb{C} \mid \left| z - z_0 \right| < r\}.\]
  The boundary of a disk is the set 
  \[C_r(z_0) := \{z \in \mathbb{C} \mid \left | z - z_0 \right| = r\}.\]
  Lastly, we write \(\mathbb{D} := D_1(0)\) for shorthand.
\end{definition}

\begin{definition}
  Let \(S \subseteq \mathbb{C}\) and \(z_0 \in S\). We call \(z_0\) an interior 
  point of \(S\) if and only if there exists some \(r > 0\) such that 
  \(D_r(z_0) \subseteq S\). We call the set of interior points of \(S\), \(S^o\) 
  -- the interior of \(S\) and we call \(S\) open if and only if every element of \(S\) 
  is an interior point of \(S\), i.e. \(S = S^o\). 
\end{definition}

We see that the above definition for open is equivalent to that which is
induced by the metric space.

\begin{definition}
  Let \(S\) be a subset of \(\mathbb{C}\), then
  \begin{itemize}
    \item \(S\) is closed if and only if \(S^c\) is open, or, 
      equivalently, \(S\) is closed if and only if for all convergent sequences 
      \((x_n) \subseteq S\), \((x_n)\) converges in \(S\).
    \item the closure of \(S\), \(\overline{S}\) is the smallest closed set 
      containing \(S\), or equivalently, the union of \(S\) and its limit points.
    \item the boundary of \(S\) is defined to be \(\partial S = 
      \overline{S} \setminus S^o\).
    \item if \(S\) is bounded, then the diameter of \(S\) is 
      \[\text{diam}(S) = \sup_{z, w \in S} \left| z - w \right|.\]
    \item \(S\) is (path) connected if and only if for all \(z, w \in S\), there exists 
      some continuous function \(\gamma : [0, 1] \to S\) such that \(\gamma(0) = z\) 
      and \(\gamma(1) = w\).
  \end{itemize}
\end{definition}

We remark that there is no confusion regarding the definition of
connectedness in \(\mathbb{C}\) since path-connectedness is a stronger
notion than connectedness in arbitrary topological spaces while in
\(\mathbb{R}^n\), open sets are path-connected if they are connected,
and so, by traversing the homeomorphism, an open set
\(S \subseteq \mathbb{C}\) is connected if and only if it is
path-connected.

As \(\mathbb{C}\) is complete the following proposition follows as
compact sets in \(\mathbb{C}\) are closed and bounded.

\begin{prop}
  Let \((S_n)\) be a sequence of non-empty decreasing subsets of \(\mathbb{C}\) 
  such that \(\text{diam}(S_n) \to 0\) as \(n \to 0\), then there exists a unique 
  point \(w \in \mathbb{C}\) such that \(w \in \bigcap_n S_n\).
\end{prop}
\proof

This result was previously proved for closed and bounded sets in
arbitrary complete metric spaces and so, this result follows as an
application of that. \qed

For good measure, let us also recall some lemmas from school regarding
algebraic manipulations of the complex numbers.

\begin{theorem}
  Let \(z_1 = r_1(\cos \theta_1 + i\sin \theta_1)\) and let 
  \(z_2 = r_2(\cos \theta_2 + i\sin \theta_2)\), then
  \[z_1 z_2 = r_1 r_2(\cos(\theta_1 + \theta_2) + i\sin(\theta_1 + \theta_2)).\]
\end{theorem}

\begin{corollary}[De Moivre's Formula]
  Let \(z = r(\cos \theta + i\sin \theta)\), then 
  \[z^n = r^n(\cos n\theta + i\sin n\theta).\]
\end{corollary}

We note that the above implies \(\arg z_1 + \arg z_2 = \arg z_1 z_2\)
but it is in general \textbf{not} true that
\(\mathop{\mathrm{Arg}}z_1 + \mathop{\mathrm{Arg}}z_2 = \mathop{\mathrm{Arg}}z_1 z_2\)
where \(\mathop{\mathrm{Arg}}z\) denote the principle argument of \(z\).

\newpage

\hypertarget{complex-functions}{%
\section{Complex Functions}\label{complex-functions}}

\end{document}
