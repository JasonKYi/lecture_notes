% Options for packages loaded elsewhere
\PassOptionsToPackage{unicode}{hyperref}
\PassOptionsToPackage{hyphens}{url}
\PassOptionsToPackage{dvipsnames,svgnames*,x11names*}{xcolor}
%
\documentclass[
]{article}
\usepackage{lmodern}
\usepackage{amssymb,amsmath}
\usepackage{ifxetex,ifluatex}
\ifnum 0\ifxetex 1\fi\ifluatex 1\fi=0 % if pdftex
  \usepackage[T1]{fontenc}
  \usepackage[utf8]{inputenc}
  \usepackage{textcomp} % provide euro and other symbols
\else % if luatex or xetex
  \usepackage{unicode-math}
  \defaultfontfeatures{Scale=MatchLowercase}
  \defaultfontfeatures[\rmfamily]{Ligatures=TeX,Scale=1}
\fi
% Use upquote if available, for straight quotes in verbatim environments
\IfFileExists{upquote.sty}{\usepackage{upquote}}{}
\IfFileExists{microtype.sty}{% use microtype if available
  \usepackage[]{microtype}
  \UseMicrotypeSet[protrusion]{basicmath} % disable protrusion for tt fonts
}{}
\makeatletter
\@ifundefined{KOMAClassName}{% if non-KOMA class
  \IfFileExists{parskip.sty}{%
    \usepackage{parskip}
  }{% else
    \setlength{\parindent}{0pt}
    \setlength{\parskip}{6pt plus 2pt minus 1pt}}
}{% if KOMA class
  \KOMAoptions{parskip=half}}
\makeatother
\usepackage{xcolor}
\IfFileExists{xurl.sty}{\usepackage{xurl}}{} % add URL line breaks if available
\IfFileExists{bookmark.sty}{\usepackage{bookmark}}{\usepackage{hyperref}}
\hypersetup{
  pdftitle={Complex Analysis},
  pdfauthor={Kexing Ying},
  colorlinks=true,
  linkcolor=Maroon,
  filecolor=Maroon,
  citecolor=Blue,
  urlcolor=red,
  pdfcreator={LaTeX via pandoc}}
\urlstyle{same} % disable monospaced font for URLs
\usepackage[margin = 1.5in]{geometry}
\usepackage{graphicx}
\makeatletter
\def\maxwidth{\ifdim\Gin@nat@width>\linewidth\linewidth\else\Gin@nat@width\fi}
\def\maxheight{\ifdim\Gin@nat@height>\textheight\textheight\else\Gin@nat@height\fi}
\makeatother
% Scale images if necessary, so that they will not overflow the page
% margins by default, and it is still possible to overwrite the defaults
% using explicit options in \includegraphics[width, height, ...]{}
\setkeys{Gin}{width=\maxwidth,height=\maxheight,keepaspectratio}
% Set default figure placement to htbp
\makeatletter
\def\fps@figure{htbp}
\makeatother
\setlength{\emergencystretch}{3em} % prevent overfull lines
\providecommand{\tightlist}{%
  \setlength{\itemsep}{0pt}\setlength{\parskip}{0pt}}
\setcounter{secnumdepth}{5}
\usepackage{tikz}
\usepackage{amsthm}
\usepackage{mathtools}
\usepackage{lipsum}
\usepackage[ruled,vlined]{algorithm2e}
\usepackage{physics}
\theoremstyle{definition}
\newtheorem{theorem}{Theorem}
\newtheorem{prop}{Proposition}
\newtheorem{corollary}{Corollary}[theorem]
\newtheorem*{remark}{Remark}
\theoremstyle{definition}
\newtheorem{definition}{Definition}[section]
\newtheorem{lemma}{Lemma}[section]
\newcommand{\diag}{\mathop{\mathrm{diag}}}
\newcommand{\Arg}{\mathop{\mathrm{Arg}}}
\newcommand{\hess}{\mathop{\mathrm{Hess}}}

\title{Complex Analysis}
\author{Kexing Ying}
\date{January 11, 2021}

\begin{document}
\maketitle

{
\hypersetup{linkcolor=}
\setcounter{tocdepth}{2}
\tableofcontents
}
\newpage

\hypertarget{complex-numbers}{%
\section{Complex Numbers}\label{complex-numbers}}

We recall some properties about the complex numbers \(\mathbb{C}\).

From \textbf{Analysis II} we recall the topological properties of
\(\mathbb{R}^2\). As there exists a natural homeomorphism from
\(\mathbb{R}^2\) to \(\mathbb{C}\), we conclude that the complex numbers
also has these properties.

\begin{prop}
  The set of complex numbers \(\mathbb{C}\) forms a metric space with the induced 
  metric from the Pythagorean norm, that is, the metric
  \[d : \mathbb{C} \times \mathbb{C} \to \mathbb{R} : (z, w) \mapsto \left| z - w \right|.\]
\end{prop}
\proof

One can trivially show that the Pythagorean norm is a norm on
\(\mathbb{C}\), and hence, the induced metric is a metric on
\(\mathbb{C}\). \qed

\begin{theorem}
  The complex numbers equipped with the distance as defined above is Lipschitz 
  equivalent to \(\mathbb{R}^2\) equipped with Euclidean metric; so, they are 
  also homeomorphic.
\end{theorem}
\proof

Trivial. \qed

\begin{corollary}
  The complex numbers is complete and a subset of \(\mathbb{C}\) is compact if 
  and only if \(\mathbb{C}\) is closed and bounded.
\end{corollary}
\proof

Follows from the Heine-Borel theorem and the fact that \(\mathbb{R}^2\)
is complete. \qed

Certain definitions are also induced for the complex numbers by the fact
that it is a metric space. We shall define them here again for
referencing.

\begin{definition}
  An open disk (ball) in \(\mathbb{C}\) centred at \(z_0 \in \mathbb{C}\) with 
  radius \(r > 0\) is the set 
  \[D_r(z_0) := \{ z \in \mathbb{C} \mid \left| z - z_0 \right| < r\}.\]
  The boundary of a disk is the set 
  \[C_r(z_0) := \{z \in \mathbb{C} \mid \left | z - z_0 \right| = r\}.\]
  Lastly, we write \(\mathbb{D} := D_1(0)\) for shorthand.
\end{definition}

\begin{definition}
  Let \(S \subseteq \mathbb{C}\) and \(z_0 \in S\). We call \(z_0\) an interior 
  point of \(S\) if and only if there exists some \(r > 0\) such that 
  \(D_r(z_0) \subseteq S\). We call the set of interior points of \(S\), \(S^o\) 
  -- the interior of \(S\) and we call \(S\) open if and only if every element of \(S\) 
  is an interior point of \(S\), i.e. \(S = S^o\). 
\end{definition}

We see that the above definition for open is equivalent to that which is
induced by the metric space.

\begin{definition}
  Let \(S\) be a subset of \(\mathbb{C}\), then
  \begin{itemize}
    \item \(S\) is closed if and only if \(S^c\) is open, or, 
      equivalently, \(S\) is closed if and only if for all convergent sequences 
      \((x_n) \subseteq S\), \((x_n)\) converges in \(S\).
    \item the closure of \(S\), \(\overline{S}\) is the smallest closed set 
      containing \(S\), or equivalently, the union of \(S\) and its limit points.
    \item the boundary of \(S\) is defined to be \(\partial S = 
      \overline{S} \setminus S^o\).
    \item if \(S\) is bounded, then the diameter of \(S\) is 
      \[\text{diam}(S) = \sup_{z, w \in S} \left| z - w \right|.\]
    \item \(S\) is (path) connected if and only if for all \(z, w \in S\), there exists 
      some continuous function \(\gamma : [0, 1] \to S\) such that \(\gamma(0) = z\) 
      and \(\gamma(1) = w\).
  \end{itemize}
\end{definition}

We remark that there is no confusion regarding the definition of
connectedness in \(\mathbb{C}\) since path-connectedness is a stronger
notion than connectedness in arbitrary topological spaces while in
\(\mathbb{R}^n\), open sets are path-connected if they are connected,
and so, by traversing the homeomorphism, an open set
\(S \subseteq \mathbb{C}\) is connected if and only if it is
path-connected.

As \(\mathbb{C}\) is complete the following proposition follows as
compact sets in \(\mathbb{C}\) are closed and bounded.

\begin{prop}
  Let \((S_n)\) be a sequence of non-empty decreasing subsets of \(\mathbb{C}\) 
  such that \(\text{diam}(S_n) \to 0\) as \(n \to 0\), then there exists a unique 
  point \(w \in \mathbb{C}\) such that \(w \in \bigcap_n S_n\).
\end{prop}
\proof

This result was previously proved for closed and bounded sets in
arbitrary complete metric spaces and so, this result follows as an
application of that. \qed

For good measure, let us also recall some lemmas from school regarding
algebraic manipulations of the complex numbers.

\begin{theorem}
  Let \(z_1 = r_1(\cos \theta_1 + i\sin \theta_1)\) and let 
  \(z_2 = r_2(\cos \theta_2 + i\sin \theta_2)\), then
  \[z_1 z_2 = r_1 r_2(\cos(\theta_1 + \theta_2) + i\sin(\theta_1 + \theta_2)).\]
\end{theorem}

\begin{corollary}[De Moivre's Formula]
  Let \(z = r(\cos \theta + i\sin \theta)\), then 
  \[z^n = r^n(\cos n\theta + i\sin n\theta).\]
\end{corollary}

We note that the above implies \(\arg z_1 + \arg z_2 = \arg z_1 z_2\)
but it is in general \textbf{not} true that
\(\mathop{\mathrm{Arg}}z_1 + \mathop{\mathrm{Arg}}z_2 = \mathop{\mathrm{Arg}}z_1 z_2\)
where \(\mathop{\mathrm{Arg}}z\) denote the principle argument of \(z\).

\newpage

\hypertarget{complex-functions}{%
\section{Complex Functions}\label{complex-functions}}

As with all spaces, we would like to study the properties of mappings
between the complex numbers.

\begin{definition}[Mapping]
  Let \(\Omega_1, \Omega_2 \subseteq \mathbb{C}\). Then, 
  \[f : \Omega_1 \to \Omega_2\]
  is said to be a mapping from \(\Omega_1\) to \(\Omega_2\) if for any 
  \(z = x + iy \in \Omega_1\), there exists only one complex number 
  \(w = u + iv\) such that \(w = f(z)\).

  In this case, we denote \(w = f(z) = u(x, y) + iv(x, y)\).
\end{definition}

We define a special mapping -- the Möbius transformation.

\begin{definition}[The Möbius Transformation]
  The Möbius transformation is a mapping such that 
  \[w = f(z) = \frac{az + b}{cz + d},\]
  for some \(a, b, c, d \in \mathbb{C}\) where \(cz + d \neq 0\) on the domain.
\end{definition}

As the complex plane is a metric space, we again have the induces notion
of continuity.

\begin{definition}[Continuity]
  Let \(f : \Omega_1 \to \Omega_2\) be some complex mapping and let \(z_0 \in \Omega_1\). 
  We say \(f\) is continuous at \(z_0\) if for every \(\epsilon > 0\) there exists 
  some \(\delta >0\) such that for all \(z \in \mathbb{C}\), 
  \(\left| z - z_0 \right| < \delta\), we have \(\left| f(z) - f(z_0) \right|\).

  We say \(f\) is continuous on \(\Omega_1\) if it is continuous at every point 
  in \(\Omega_1\).
\end{definition}

Since the complex plane is homeomorphic to the Euclidean space
\(\mathbb{R}^2\), one might think to establish a notion of derivative on
\(\mathbb{C}\). This is achieved, however, not through the definition on
general Euclidean spaces, but through another definition.

\begin{definition}[Holomorphic]
  Let \(\Omega_1, \Omega_2 \subseteq \mathbb{C}\) be open sets and let 
  \(f : \Omega_1 \to \Omega_2\). Then we say \(f\) is holomorphic (differentiable) 
  at some \(z_0 \in \Omega_1\) if the limit 
  \[\lim_{h \in \mathbb{C} \to 0} \frac{f(z_0 + h) - f(z_0)}{h}\]
  exists. Here we restricts \(z_0 + h \in \Omega_1\) which is fine since \(\Omega_1\) 
  is open, and hence, there exists some \(\delta > 0\) such that 
  \(B_\delta(z_0) \subseteq \Omega_1\). If \(f\) is holomorphic at \(z_0\) then 
  we call the quotient its derivative and denote it by \(f'(z_0)\). 
  
  Let \(S \subseteq \mathbb{C}\) be some complex set, then, we say \(f\) is 
  holomorphic on \(S\) if 
  \begin{itemize}
    \item \(S\) is open and \(f\) is holomorphic on every point of \(S\);
    \item \(S\) is closed and \(f\) is holomorphic on some open set containing \(S\).
  \end{itemize}
  If \(f\) is holomorphic on \(\mathbb{C}\) itself then we say \(f\) is entire.
\end{definition}

We note that we are allowed to make this definition as there exists a
notion of division on \(\mathbb{C}\) while the same cannot be said for
general Euclidean spaces.

The function \(f(z) = \overline{z}\) is not holomorphic. Indeed, the
quotient \[\frac{f(z_0 + h) - f(z_0)}{h} = \frac{\overline{h}}{h}\]
doest not have a limit as \(n \to \infty\) and so our claim.

\begin{prop}
  A function \(f\) is holomorphic at \(z_0 \in \Omega\) if and only if there exists 
  a complex number \(a\) such that 
  \[f(z_0 + h) - f(z_0) - ah = o(h),\]
  or equivalently (without the syntactic sugar), 
  \[f(z_0 + h) - f(z_0) - ah = h\psi(h)\]
  where \(\psi : D_\epsilon(0) \to \mathbb{C}\) is a function such that 
  \(\lim_{h \to 0} \psi(h) = 0\) for some \(\epsilon > 0\).
\end{prop}
\proof

Straight away, by dividing both side by \(h\), we have
\[\frac{f(z_0 + h) - f(z_0)}{h} - a = \psi(h) \to 0\] as \(h \to 0\).
\qed

By taking \(h \to 0\) on both sides of the equation, we have
\(f(z) \to f(z_0)\) as \(z \to z_0\) and so, the following corollary.

\begin{corollary}
  A holomorphic function \(f\) is continuous.
\end{corollary}

As one might imagine, the normal properties of derivatives hold for this
definition as well.

\begin{prop}
  If \(f, g\) are holomorphic in \(\Omega\) then, 
  \begin{itemize}
    \item \(f + g\) is holomorphic in \(\Omega\) and \((f + g)' = f' + g'\);
    \item \(fg\) is holomorphic in \(\Omega\) and \((fg)' = f'g + fg'\);
    \item if \(g(z_0) \neq 0\), then \(f / g\) is holomorphic at \(z_0\) and 
    \((f/g)' = \frac{f'g + fg'}{g^2};\)
  \end{itemize}
  Moreover, if \(f : \Omega \to U\) and \(g : U \to \mathbb{C}\) are both holomorphic, 
  the chain rule holds, that is 
  \[(g \circ f)'(z) = g'(f(z))f'(z).\]
\end{prop}
\proof

Omitted. One can use the above proposition to make life easier. \qed

\hypertarget{cauchy-riemann-equations}{%
\subsection{Cauchy-Riemann Equations}\label{cauchy-riemann-equations}}

Consider the limit
\[f'(z_0) = \lim_{h = h_1 + ih_2 \to 0} \frac{f(z_0 + h) - f(z_0)}{h}.\]
Assuming that \(h = h_1\), namely \(h_2 = 0\) and by writing,
\[f(z_0) = f(x_0 + iy_0) = u(x_0, y_0) + iv(x_0, y_0),\] we have,
\[ \begin{split}
  f'(z_0) & = \lim_{h \to 0} \frac{f(z_0 + h) - f(z_0)}{h} \\
    & = \lim_{h_1 \in \mathbb{R} \to 0} \frac{u(x_0 + h_1, y_0) + 
      iv(x_0 + h_1, y_0) - u(x_0, y_0) - iv(x_0, y_0)}{h_1} \\
    & = \frac{\partial u}{\partial x}(x_0, y_0) + i \frac{\partial v}{\partial x}{x_0, y_0} 
      = u'_x(x_0, y_0) + i v'_x(x_0, y_0).
\end{split} \] Similarly, if we let \(g = ih_2\) by letting \(h_1 = 0\),
we have
\[f'(z_0) = \frac{1}{i}\frac{\partial u}{\partial v}(x_0, y_0) + 
  \frac{\partial v}{\partial y}(x_0, y_0) = -iu'_y(x_0, y_0) + v'_y(x_0, y_0).\]
So, if \(f\) is holomorphic at \(z_0\), then the two limit should agree,
and hence
\[\frac{\partial u}{\partial x} = \frac{\partial v}{\partial y}, \hspace{2mm} 
  \frac{\partial u}{\partial y} = - \frac{\partial v}{\partial x}.\]
These two equations together are called the \emph{Cauchy-Riemann
equations}.

\begin{definition}[Cauchy-Riemann Equations]
  Let \(f(z) = u(x, y) + iv(x, y)\) be a mapping, then the Cauchy-Riemann equations 
  are the system of equations 
  \[u'_x = v'_y; \hspace{2mm} u'_y = - v'_x.\] 
\end{definition}

With the Cauchy-Riemann equations, we have a necessary condition for a
function to be holomorphic. As shown above, we have found that the
conjugate function \(f = z \mapsto \overline{z}\) is not holomorphic,
and we see that as well with its Cauchy-Riemann equations since
\(u'_x = 1 \neq -1 = v'_y\).

The Cauchy-Riemann equations links real and complex analysis in some
sense. By defining the operators
\[ \frac{\partial}{\partial z} = \frac{1}{2}\left(\frac{\partial}{\partial x} + 
  \frac{1}{i} \frac{\partial}{\partial y}\right); \hspace{2mm} 
\frac{\partial}{\partial \bar{z}} = \frac{1}{2}\left(\frac{\partial}{\partial x} - 
  \frac{1}{i} \frac{\partial}{\partial y}\right), \hspace{2mm} \] we
have the following theorem.

\begin{theorem}
  Let \(f = u + iv\). If \(f\) is holomorphic at \(z_0\), then 
  \[\frac{\partial f}{\partial \bar{z}}(z_0) = 0,\]
  and 
  \[f'(z_0) = \frac{\partial f}{\partial z}(z_0) = 2 \frac{\partial u}{\partial z}(z_0).\]
\end{theorem}
\proof

Trivially follows by using the Cauchy-Riemann equations (except perhaps
for showing \(f'(z_0) = \partial u / \partial z\) which follows since we
can write \(f'(z_0) = u'_x(z_0) + iv'_x(z_0)\) and so, the result
follows by rewriting with the Cauchy-Riemann equations). \qed

Similar to the necessary and sufficient conditions for the existence of
derivatives for general Euclidean spaces, we would like a similar
theorem for determining whether or not a complex valued function is
holomorphic. This is achieved with the following theorem.

\begin{theorem}
  Suppose \(f = u + iv\) is a complex-valued function defined on some open set 
  \(\Omega\). If \(u, v\) are continuously differentiable and satisfy the 
  Cauchy-Riemann equations on \(\Omega\), then \(f\) is holomorphic on \(\Omega\) 
  and 
  \[f'(z) = \frac{\partial f}{\partial z}(z).\]
\end{theorem}

The proof of this theorem is similar to the equivalent statement in
\(\mathbb{R}^n\), namely, a function is differentiable if and only if
it's partial derivatives are continuously differentiable and its
derivative is the Jacobian. \proof Let \(h = h_1 + ih_2\), then,
\[u(x + h_1, y + h_2) - u(x, y) = u_x'(x, y)h_1 + u_y'(x, y)h_2 + o(\left|h\right|),\]
and
\[v(x + h_1, y + h_2) - v(x, y) = v_x'(x, y)h_1 + v_y'(x, y)h_2 + o(\left|h\right|).\]
By the Cauchy-Riemann equations,
\[v_x' = -u_y'; \hspace{2mm} v_y'= u_x',\] we find that, \[\begin{split}
      f(z + h) - f(z) & = u(x + h_1, y + h_2) + iv(x + h_1, y + h_2) - u(x, y) - iv(x, y)\\
        & = (u(x + h_1, y + h_2) - u(x, y)) + i(v(x + h_1, y + h_2) - v(x, y))\\
        & = u_x'(x, y)h_1 + u_y'(x, y)h_2 + o(\left|h\right|) + i(v_x'(x, y)h_1 + v_y'(x, y)h_2 + o(\left|h\right|))\\
        & = u_x' h_1 + u_y' h_2 + i u_x' h_2- i u_y h_1 + o(\left|h\right|)\\
        & = (u_x' - iu_y')(h_1 + ih_2) + o(\left|h\right|).
    \end{split}\] This gives us
\[f(z + h) - f(z) - (u_x' - iu_y')h = o(\left|h\right|),\] implying
\(f\) is differentiable at \(z\) with derivative
\[f'(z) = u_x' - iu_y' = 2\pdv{u}{z} = \pdv{f}{z}.\] \qed

Lastly, by transforming the complex numbers into their polar forms, we
have the following Cauchy-Riemann equations,
\[u_r' = \frac{1}{r}v_\theta'; \hspace{2mm} v_r' = - \frac{1}{r}u_\theta'.\]
The derivation of this is completely mundane so is omitted here (see
lecture slides). This form of Cauchy-Riemann equations can be useful
when dealing with certain functions where the function is easier to deal
with in polar coordinates.

\hypertarget{complex-exponentials}{%
\subsection{Complex Exponentials}\label{complex-exponentials}}

We recall the definition and properties of a complex power series.

\begin{definition}
  A power series is an expansion of the form 
  \[\sum_{n = 0}^\infty a_n z^n,\]
  where \(a_n \in \mathbb{C}\) for all \(n \in \mathbb{N}\).
\end{definition}

We call the aforementioned power series convergent at some
\(z \in \mathbb{C}\) if the partial sum
\(S_N(z) = \sum_{n = 0}^N a_n z^n\) has a limit in \(\mathbb{C}\),
\(S(z) = \lim_{N \to \infty} S_N(z)\). If that is the case we write
\(\sum_{n = 0}^\infty a_n z^n\).

Furthermore, we define a power series \(\sum a_n z^n\) to be absolutely
if \(sum \left| a_n \right| \left| z \right|^n\) converges. As we have
seen from last year, absolute convergence implies converges, and
furthermore, if \(S(z) = \sum a_n z^n\) then
\(\lim_{N \to \infty} S(z) - S_N(z) = 0\).

\begin{theorem}
  Given \(\sum a_n z^n\) be a power series, then there exists some 
  \(R \in \mathbb{R}^+\) such that for all \(z \in \mathbb{C}\), if 
  \(\left| z \right| < R\) then the series converges absolutely and if 
  \(\left| z \right| > R\) then the series diverges. Moreover, 
  \[\frac{1}{R} = \limsup_{n \to \infty} \left| a_n \right|^{1 / n}.\]
  The number \(R\) is called the radius of convergence and the domain 
  \(D_R\) is refered as the disc of convergence. 
\end{theorem}
\proof

See last year's notes. \qed

The power series provide us with an important class of holomorphic
functions.

\begin{theorem}
  The power series \(f(z) = \sum_{n = 0} a_n z^n\) defines a holomorphic function 
  in its disc of convergence. In fact the derivative of \(f\) is simply the sum of 
  the derivative of its individual terms, that is, 
  \[f'(z) = \sum_{n = 1}^\infty n a_n z^{n - 1}.\]
  Moreover, \(f\) has the same 
  radius of convergence as \(f\).
\end{theorem}
\proof

We note that \(\sum_{n = 1}^\infty a_n z^{n - 1}\) and
\(\sum_{n = 1}^\infty na_n z^n\) have the same radius of convergence by
considering the above proposition. Indeed, if
\(\sum_{n = 1}^\infty a_n z^{n - 1}\) has radius of convergence \(R\),
then
\(R = (\limsup_{n \to \infty} \left| a_n \right|^{1 / n})^{-1}  = (\limsup_{n \to \infty} \left| n a_n \right|^{1 / n})^{-1}\),
which is the radius of convergence of \(\sum_{n = 1}^\infty na_n z^n\).
Beware that we used the fact that
\[\lim_{n \to \infty} n^{1 / n} = \lim_{n \to \infty} \exp \left(\frac{\log n}{n}\right) = 1.\]
Thus, it remains to show that
\(f'(z) = \sum_{n = 1}^\infty n a_n z^{n - 1}\).

Let \(R\) be the radius of convergence of \(f\),
\(\left| z_0 \right| < r < R\) and define
\(g(z) = \sum_{n = 1}^\infty n a_n z^{n - 1}\) and,
\[S_N(z) = \sum_{n = 0}^N a_n z^n; \hspace{2mm} E_N(z) = \sum_{n = N + 1}^\infty a_n z^n.\]
Then, Wlog. assume \(\left| z_0 + h \right| < r\), so \[\begin{split}
    \frac{f(z_0 + h) - f(z_0)}{h} - g(z_0) & = 
    \left(\frac{S_N(z_0 + h) - S_N(z_0)}{h} - S'_N(z_0)\right)\\ 
      & + (S'_N(z_0) - g(z_0)) + \left(\frac{E_N(z_0 + h) - E_N(z_0)}{h}\right),
    \end{split}\] for arbitrary \(N \in \mathbb{N}\). Thus, by taking
\(N \to \infty, h \to 0\), we see that the first two term of the right
hand side vanished while further examination is required for the last
term. Indeed, by the triangle inequality,
\[\left|\frac{E_N(z_0 + h) - E_N(z_0)}{h}\right| \le 
    \sum_{n = N + 1}^\infty \left|a_n\right| \left|\frac{(z_0 + h)^n - z_0^n}{h}\right| 
    \le \sum_{n = N + 1}^\infty \left| a_n \right| n r^{n - 1},\] where
the last term tends to \(0\) as \(N \to \infty\). Hence,
\[\frac{f(z_0 + h) - f(z_0)}{h} - g(z_0) \to 0,\] as \(h \to 0\) and the
result is complete. \qed

As the term-wise derivate of a power series is also a power series, the
above theorem results in the following corollary.

\begin{corollary}
  A power series is infinitely differentiable within its disc of convergence, and 
  its higher derivatives are also power series obtained by term-wise differentiation.
\end{corollary}

This reason for this revision and seemingly digression from complex
functions is because we would like to introduce complex exponentials. We
recall from last year that we defined the exponential function as
\[\exp : \mathbb{R} \to \mathbb{R} : x \mapsto \sum_{n = 0}^\infty \frac{x^n}{n!},\]
and we write \(e^x := \exp(x)\). We define the complex exponential on
top of this definition.

\begin{definition}
  Given \(z = x + iy \in \mathbb{C}\), we define the complex exponential, 
  \[e^z := e^x \cos y + i e^x \sin y.\]
\end{definition}

As with the real exponential, the complex exponential has several nice
properties.

\begin{prop}
  Let us denote \(e^{(\cdot)}\) for the complex exponential, then, 
  \begin{itemize}
    \item if \(z = x + 0i\), then \(e^z = e^x\);
    \item \(e^{(\cdot)}\) is entire (holomorphic for all \(z \in \mathbb{C})\);
    \item \(\pdv{z}e^z = e^z\);
    \item if \(g(z)\) is holomorphic, then \(\pdv{z}e^{g(z)} = e^{g(z)}g'(z)\);
    \item \(e^{z_1 + z_2} = e^{z_1} e^{z_2}\);
    \item \(\left| e^z \right| = e^x\);
    \item \((e^{z})^n = e^{nz}\);
    \item \(\arg e^z = y + 2\pi k\) for \(k = 0, \pm 1, \pm 2, \cdots\).
  \end{itemize} 
\end{prop}
\proof

Straight forward either by definition or by the Cauchy-Riemann
equations. \qed

From the definition of the complex exponential function, we define the
complex trigonometric functions.

\begin{definition}[Complex Trigonometric Functions]
  For any \(z \in \mathbb{C}\), we define 
  \[\sin z = \frac{1}{2i}(e^{iz} - e^{-iz}); \hspace{2mm} \cos z = \frac{1}{2}(e^{iz} + e^{-iz}).\]
\end{definition}

To see why the trigonometric functions are extended in such a way, one
can simply consider Euler's identity where we have
\[e^{i \theta} = \cos \theta + i \sin \theta; \hspace{2mm} e^{-i\theta} = cos \theta - i \sin \theta,\]
for any \(\theta \in \mathbb{R}\). Thus, by assuming
\(\theta \in \mathbb{C}\), and solving the system of equations, we have
the definition of the complex trigonometric functions.

As the complex trigonometric functions are simply linear combinations of
the complex exponential, we again get many nice properties for free.

\begin{prop}
  Let \(z = x + iy \in \mathbb{C}\), then 
  \begin{itemize}
    \item \(\sin (\cdot)\) and \(\cos (\cdot)\) are entire functions;
    \item \(\pdv{z}\sin z = \cos z\) and \(\pdv{z}\cos z = -\sin z\);
    \item \(\sin^2 z + \cos^2 z = 1\);
    \item \(\sin(z_1 \pm z_2) = \sin z_1 \cos z_2 \pm \cos z_1 \sin z_2\) and 
      \(\cos(z_1 \pm z_2) = \cos z_1 \cos z_2 \mp \sin z_1 \sin z_2\);
  \end{itemize}
\end{prop}

Lastly, we define the complex logarithmic functions.

\begin{definition}[Complex Logarithm]
  Let \(z = r(\cos\theta + i\sin\theta) = re^{i\theta}\), then 
  \[\log z = \log \left|z\right| + i \arg z = \log r + i(\theta + 2\pi k),\]
  for \(k = 0, \pm 1, \pm 2, \cdots\).
\end{definition}

Clearly,
\(e^{\log z} = e^{\log r + i(\theta + 2\pi k)} = r e^{i (\theta + 2\pi k)} = re^{i\theta} = z\)
so it behaves as we would expect. However, as \(\log\) results in an
infinite family of values, and so, is not a function. To account for
this, we consider the principle logarithmic function.

\begin{definition}[Principle Logarithm]
  The principle logarithmic function is 
  \[\text{Log} : \mathbb{C} \to \mathbb{C} : z \mapsto \log\left| z \right| + i\mathop{\mathrm{Arg}}z.\]
\end{definition}

Again, we find that the complex logarithms behave as one might expect.

\begin{prop}
  Let \(z \in \mathbb{C}\), then 
  \begin{itemize}
    \item \(\log(z_1 z_2) = \log z_1 + \log z_2\);
    \item  the principle logarithm is holomorphic on \(\mathbb{C} \setminus (-\infty, 0]\).
  \end{itemize}
\end{prop}

With these definitions in place, we may finally define complex powers.

\begin{definition}
  For all \(\alpha \in \mathbb{C}\), we define 
  \[z^\alpha = e^{\alpha \log z}\]
  as a multi-valued function and we define its principle value to be 
  \[z^\alpha = e^{\alpha \text{Log} z}.\]
\end{definition}
\begin{prop}
  Let \(z, \alpha_1, \alpha_2 \in \mathbb{C}\), then 
  \(z^{\alpha_1 + \alpha_2} = z^{\alpha_1}z^{\alpha_2}\).
\end{prop}

\hypertarget{parametrised-curve}{%
\subsection{Parametrised Curve}\label{parametrised-curve}}

As we shall study integration along curves in the next section, let us
first look at parametrised curves.

\begin{definition}[Parametrised Curve]
  A parametrised curve is a function \(z : [a, b] \subseteq \mathbb{R} \to \mathbb{C}\).
  We say that the parametrised curve is smooth if \(z'\) exists and is continuous 
  on \([a, b]\) and \(z'(t) \neq 0\) for all \(t \in [a, b]\). At the points 
  \(t = a\) and \(t = b\), the derivative of \(z\) at \(t\) is defined to be the 
  one-sided limits 
  \[z'(a) = \lim_{h \to 0, h > 0} \frac{z(a + h) - z(a)}{h}, \hspace{2mm} 
    z'(b) = \lim_{h \to 0, h < 0} \frac{z(b + h) - z(b)}{h}.\]
  Similarly, we say that the parameterised curve is piecewise smooth if \(z\) is 
  continuous on \([a, b]\) and there exist a finite number of points 
  \(a = a_0 < a_1 < \cdots < a_n = b\) such that \(z\) is smooth on \([a_k, a_{k + 1}]\). 
  In particular, the right-hand and the left-hand derivative of \(z\) at \(a_k\) need not 
  to agree.
\end{definition}

\begin{definition}[Equivalence of Parametrisations]
  Let \(z : [a, b] \to \mathbb{C}\) and \(\hat{z} : [c, d] \to \mathbb{C}\) be two 
  parametrisations: Then, we say \(z\) is equivalent to \(hat{z}\) if there exists a 
  continuous bijective \(t : [c, d] \to [a, b]\) such that \(t' > 0\) on \([c, d]\) 
  and 
  \[\hat{z}(s) = z(t(s)),\]
  for all \(s \in [c, d]\).
\end{definition}

The requirement for \(t' > 0\) says precisely that the orientation is
preserved, since as \(s\) travels from \(c\) to \(d\), \(t(s)\) travels
from \(a\) to \(b\).

Parametrising curves allows us more easily integrate functions along
curves.

\begin{definition}[Complex Integration]
  Given a smooth curve \(\gamma \subseteq \mathbb{C}\) is parameterised by 
  \(z : [a, b] \to \mathbb{C}\), and \(f : \gamma \to \mathbb{C}\) is a 
  continuous function, we define 
  \[\int_\gamma f(z) \dd z = \int_a^b f(z(t)) z'(t) \dd t.\]
\end{definition}

In order for this definition to be meaningful, we must show that the
integration along \(\gamma\) is independent from the parametrisation.
That is, if \(\hat{z}\) is an equivalent parametrisation to \(z\),
then\\
\[\int_\gamma f(z) \dd z = \int_a^b f(z(t)) z'(t) \dd t = \int_c^d f(\hat{z}(t)) \hat{z}'(t) \dd t.\]
However, this is easy to see by the chain rule:
\[\int_c^d f(\hat{z}(s)) \hat{z}'(s) \dd s
  = \int_c^d f(z(t(s)) z'(t(s)) t'(s) \dd t = \int_a^b f(z(t)) z'(t) \dd t.\]
As one might expect, if \(\gamma \subseteq \mathbb{C}\) is a piecewise
smooth curve and if \(z\) is a piecewise smooth parametrisation of
\(\gamma\), then
\[\int_\gamma f(z) \dd z = \sum_{k = 0}^{n - 1} \int_{a_k}^{a_{k + 1}} f(z(t)) z'(t) \dd t.\]

Intuitively, if we have a curve in space, it is obvious that we can
reverse the orientation to obtain a new curve. We formalise this notion
by defining \(\gamma^-\) for any curve \(\gamma\) such that \(\gamma\)
and \(\gamma^-\) consists of the same points. Then, if
\(z : [a, b] \to \mathbb{C}\) is any parametrisation of \(\gamma\),
there is a corresponding parametrisation of \(\gamma^-1\) defined by
\[z^- : [a, b] \to \mathbb{C} : t \mapsto z(b + a - t).\]

\begin{definition}[Closed]
  A smooth or piecewise smooth curve is closed if \(z(a) = z(b)\) for any of 
  its parametrisations \(z : [a, b] \to \mathbb{C}\).
\end{definition}
\begin{definition}[Simple]
  A smooth or piecewise smooth curve is simple if for all \(s, t \in [a, b]\), 
  \(s \neq t\) implies \(z(s) \neq z(t)\) for any of its parametrisations 
  \(z : [a, b] \to \mathbb{C}\).
\end{definition}

We see that a curve is simple if and only if all of its parametrisations
are injective. Indeed, if \(z(x) = z(y)\), then \(x = y\) by
contrapositive.

\begin{definition}[Contour]
  A contour is a simple closed curve which is piecewise continuously differentiable.
\end{definition}

\end{document}
