% Options for packages loaded elsewhere
\PassOptionsToPackage{unicode}{hyperref}
\PassOptionsToPackage{hyphens}{url}
\PassOptionsToPackage{dvipsnames,svgnames*,x11names*}{xcolor}
%
\documentclass[
]{article}
\usepackage{lmodern}
\usepackage{amssymb,amsmath}
\usepackage{ifxetex,ifluatex}
\ifnum 0\ifxetex 1\fi\ifluatex 1\fi=0 % if pdftex
  \usepackage[T1]{fontenc}
  \usepackage[utf8]{inputenc}
  \usepackage{textcomp} % provide euro and other symbols
\else % if luatex or xetex
  \usepackage{unicode-math}
  \defaultfontfeatures{Scale=MatchLowercase}
  \defaultfontfeatures[\rmfamily]{Ligatures=TeX,Scale=1}
\fi
% Use upquote if available, for straight quotes in verbatim environments
\IfFileExists{upquote.sty}{\usepackage{upquote}}{}
\IfFileExists{microtype.sty}{% use microtype if available
  \usepackage[]{microtype}
  \UseMicrotypeSet[protrusion]{basicmath} % disable protrusion for tt fonts
}{}
\makeatletter
\@ifundefined{KOMAClassName}{% if non-KOMA class
  \IfFileExists{parskip.sty}{%
    \usepackage{parskip}
  }{% else
    \setlength{\parindent}{0pt}
    \setlength{\parskip}{6pt plus 2pt minus 1pt}}
}{% if KOMA class
  \KOMAoptions{parskip=half}}
\makeatother
\usepackage{xcolor}
\IfFileExists{xurl.sty}{\usepackage{xurl}}{} % add URL line breaks if available
\IfFileExists{bookmark.sty}{\usepackage{bookmark}}{\usepackage{hyperref}}
\hypersetup{
  pdftitle={Statistical Modelling I},
  pdfauthor={Kexing Ying},
  colorlinks=true,
  linkcolor=Maroon,
  filecolor=Maroon,
  citecolor=Blue,
  urlcolor=red,
  pdfcreator={LaTeX via pandoc}}
\urlstyle{same} % disable monospaced font for URLs
\usepackage[margin = 1.5in]{geometry}
\usepackage{graphicx}
\makeatletter
\def\maxwidth{\ifdim\Gin@nat@width>\linewidth\linewidth\else\Gin@nat@width\fi}
\def\maxheight{\ifdim\Gin@nat@height>\textheight\textheight\else\Gin@nat@height\fi}
\makeatother
% Scale images if necessary, so that they will not overflow the page
% margins by default, and it is still possible to overwrite the defaults
% using explicit options in \includegraphics[width, height, ...]{}
\setkeys{Gin}{width=\maxwidth,height=\maxheight,keepaspectratio}
% Set default figure placement to htbp
\makeatletter
\def\fps@figure{htbp}
\makeatother
\setlength{\emergencystretch}{3em} % prevent overfull lines
\providecommand{\tightlist}{%
  \setlength{\itemsep}{0pt}\setlength{\parskip}{0pt}}
\setcounter{secnumdepth}{5}
\usepackage{tikz}
\usepackage{amsthm}
\usepackage{amssymb}
\usepackage{mathtools}
\usepackage{lipsum}
\usepackage[ruled,vlined]{algorithm2e}
\usepackage{physics}
\theoremstyle{definition}
\newtheorem{theorem}{Theorem}
\newtheorem{prop}{Proposition}
\newtheorem{corollary}{Corollary}[theorem]
\newtheorem{example}{Example}
\newtheorem*{remark}{Remark}
\theoremstyle{definition}
\newtheorem{definition}{Definition}[section]
\newtheorem{lemma}{Lemma}[section]
\newcommand{\diag}{\mathop{\mathrm{diag}}}
\newcommand{\Arg}{\mathop{\mathrm{Arg}}}
\newcommand{\hess}{\mathop{\mathrm{Hess}}}
\newcommand{\Var}{\mathop{\mathrm{Var}}}
\newcommand{\bias}[1]{{\mathop{\mathrm{bias}}}_{#1}}
\newcommand{\se}[1]{{\mathop{\mathrm{SE}}}_{#1}}
\newcommand{\mse}[1]{{\mathop{\mathrm{MSE}}}_{#1}}

\title{Statistical Modelling I}
\author{Kexing Ying}
\date{January 11, 2021}

\begin{document}
\maketitle

{
\hypersetup{linkcolor=}
\setcounter{tocdepth}{2}
\tableofcontents
}
\newpage

\hypertarget{introduction}{%
\section{Introduction}\label{introduction}}

In this module, we will consider and analyse the relationship between
measurements through the use of statistical models. This is realised in
several ways including quantifying distributions, comparing
distributions and predicting observations. We shall study these methods
through deriving, evaluating and applying estimators, confidence
intervals and hypothesis tests based, first on parametric models, and
later based on the theory of linear models.

\begin{definition}[Statistical Model]
  A statistical model is a specification of the distribution of \(Y\) up to an 
  unknown parameter \(\theta\).
\end{definition}

\begin{definition}[Parameter Space]
  Given a statistical model \(Y\) up to some parameter \(\theta\), the set \(\Theta\) 
  of all possible parameter values is called the parameter space.
\end{definition}

In this module we will assume \(\Theta \subseteq \mathbb{R}^p\) for some
\(p \in \mathbb{N}\) so that we consider \emph{parametric models}. A
\emph{semiparametric model} is a statistical model which parameters
belong to a more general space, e.g.~functions spaces.

As with last years \textbf{Probability and Statistics}, we will denote
the data \(\mathbf{y} = (y_1, \cdots, y_n) \in \mathbb{R}^n\) as a
vector and \(\mathbf{Y} = (Y_1, \cdots, Y_n)\) a random vector. In this
case, the statistical model specifies the joint distribution of
\(Y_1, \cdots, Y_n\) up to some unknown parameter \(\theta\). If
\(Y, \cdots, Y_n\) are independent and identically distributed (iid.),
then we call it a \emph{random sample}.

Furthermore, in some situations, the random vector
\(\mathbf{Y} = (Y_1, \cdots, Y_n)\) might be dependent on random or
nonrandom values \(x_1, \cdots, x_n\). The \(x_i\)'s are an example of
covariates. An example of this could be that, in a clinical trial, some
patients are given a treatment while others received a placebo. If we
would like to model the outcome of the \(i\)-th patient by their
survival time \(Y_i\), it is clear that as covariate for the \(i\)-th
patient, we may use the indicator function for whether or not \(i\)
received the treatment as a covariate.

As another example, say we would like to ask whether or not taller
people have a higher income. To answer this question we might create a
statistical model in which \(Y_i\) is the income, \(x_i\) is the height
and \[Y_i = \beta_0 + x_i \beta_1 + \epsilon_i\] for
\(i = 1, \cdots, n\), \(\epsilon \sim N(0, \sigma^2)\) iid. and
\(\theta = (\beta_0, \beta_1, \sigma^2)\),
\(\Theta = \mathbb{R}^2 \times [0, \infty)\).

Having formulated a model, we can draw inferences from the sample. By
estimating the unknown parameters we attempts to ``fit the model''.
Through this, we receive a model that can provide us with point
estimates, or better yet, tools that can help us make decisions through
some combination of hypothesis tests and confidence intervals.

However, with all statistical models, we have to accept that it will not
perfectly reflect reality. But, that is not the point of statistical
models anyway. Statistical models are meant to be useful and in general
we would like a model to

\begin{itemize}
  \item agree with the observed data reasonably;
  \item be relatively simple;
  \item interpretable, e.g. parameters have a physical interpretation.
\end{itemize}

With these aims in mind, we might conduct sensitivity analysis in which
we discard models that are not adequate for the data through a iterative
process.

\newpage

\hypertarget{point-estimation}{%
\section{Point Estimation}\label{point-estimation}}

From the introduction, we seen that during the process of fitting the
model, we need to estimate \(\theta\) in the statistical model, and
furthermore, during the inference process, we need to point estimate,
interval estimate or hypothesis test to address our question. We recall
from first year that this can be achieved through several methods and we
shall quickly review them here.

\hypertarget{review}{%
\subsection{Review}\label{review}}

We recall the following definitions.

\begin{definition}[Realisation, Statistic, Estimate, Estimator]
~
\begin{itemize}
  \item Data \(y_1, \cdots, y_n\) is called a realisation of \(Y_1, \cdots, Y_n\).
  \item A function \(t\) of observable random variables is called a statistic.
  \item An estimate of \(\theta\) is \(t(y_1, \cdots, y_n)\).
  \item An estimator of \(\theta\) is \(T = t(Y_1, \cdots, Y_n)\).
\end{itemize}
\end{definition}

\begin{example}
  Let \(Y_1, \cdots, Y_n \sim N(\mu, 1)\) iid. for some unknown \(\mu \in \mathbb{R}\).
  There are many methods for estimating \(\mu\). 
  \begin{itemize}
    \item the sample mean \(\hat{\mu} = \frac{1}{n} \sum y_i\);
    \item the sample median;
    \item the \(k\)-trimmed mean where we discard the highest and lowest \(k\) observed \(y_i\)
      before computing the mean;
    \item \(\cdots\) 
  \end{itemize}
  For the sample mean estimate, the corresponding estimator is 
  \(T = \bar{Y} = \frac{1}{n}\sum Y_i\).
\end{example}

As we can see from the example, there are many possible estimations for
the same parameter. To justify the use of a specific estimator, one
might use a frequentist's perspective and generate many data and
tabulate the results of each estimator. Through this process, one can
justify a particular estimator through observed data.

As estimators are random variables, we can formalise this idea by
considering properties of its sampling distribution (that is the
distribution of the estimator), e.g.~
\[\mathbb{P}_\theta(T \in \mathcal{A}), \hspace{2mm} E_\theta(T), \hspace{2mm} {\mathop{\mathrm{Var}}}_\theta(T), \cdots\]
We saw this idea last year in the form of \emph{bias} and \emph{mean
square error}. We recall the definitions here.

\begin{definition}[Bias]
  Let \(T\) be an estimator of \(\theta \in \Theta \subseteq \mathbb{R}\). Then 
  the bias of \(T\) is 
  \[{\mathop{\mathrm{bias}}}_{\theta}(T) = E_\theta(T) - \theta.\]
  If \({\mathop{\mathrm{bias}}}_{\theta}(T) = 0\) for all \(\theta \in \Theta\), then we say \(T\) is 
  unbiased for \(\theta\).
\end{definition}

If the parameter space is higher dimensional, say
\(\Theta \subseteq \mathbb{R}^k\), we may be instead be interested in
the value of \(g(\theta)\) for some \(g : \Theta \to \mathbb{R}\). Then,
we can naturally extend the definition of bias to this by
\[{\mathop{\mathrm{bias}}}_{\theta}(T) = E_\theta(T) - g(\theta).\]

\begin{example}
  Let \(Y_1, \cdots, Y_n \sim N(\mu, \sigma^2)\) iid. 
  \(\theta = (\mu, \sigma^2) \in \Theta = \mathbb{R} \times (0, \infty)\). Then, 
  say if we are in \(\mu\), we may define \(g : \Theta \to \mathbb{R} : 
  (\mu, \sigma^2) \mapsto \mu\).
\end{example}
\begin{definition}[Mean Square Error]
  Let \(T\) be an estimator of \(\theta \in \Theta \subseteq \mathbb{R}\). Then 
  the mean square error of \(T\) is 
  \[{\mathop{\mathrm{MSE}}}_{\theta}(T) = E_\theta[(T - \theta)^2].\]
\end{definition}

In addition to this, we have the standard error of a estimator

\begin{definition}[Standard Error]
  Let \(T\) be an estimator of \(\theta \in \Theta \subseteq \mathbb{R}\). Then 
  the standard error of \(T\) is 
  \[{\mathop{\mathrm{SE}}}_{\theta}(T) = \sqrt{{\mathop{\mathrm{Var}}}_\theta(T)}.\]
\end{definition}

From last year, we saw the following proposition.

\begin{prop}
  Let \(T\) be an estimator of \(\theta \in \Theta \subseteq \mathbb{R}\). Then 
  \[{\mathop{\mathrm{MSE}}}_{\theta}(T) = {\mathop{\mathrm{Var}}}_\theta(T) + ({\mathop{\mathrm{bias}}}_{\theta}(T))^2.\]
\end{prop}

If we restrict out estimators to be unbiased, often times, we find that
the remaining possible estimators well-behaved and we can often find the
best estimators by minimising the MSE. However, a biased estimator might
have a small MSE than an unbiased estimator (recall sample variance),
and it is not necessarily true that such an estimator even exists.

\hypertarget{cramuxe9r-rao-lower-bound}{%
\subsection{Cramér-Rao Lower Bound}\label{cramuxe9r-rao-lower-bound}}

As the mean square error provide us with a method of quantifying how
good an estimator is, we are motivated by minimising the mean square
error for a family of estimators. That is, if \(\theta \in \Theta\) is a
parameter, then is there an estimator \(T\) of \(\theta\) such that for
all estimators of \(\theta\), \(S\),
\[{\mathop{\mathrm{MSE}}}_{\theta}(T) \le {\mathop{\mathrm{MSE}}}_{\theta}(S).\]
Unfortunately, the answer to this question is in general, no, however,
for unbiased estimators, the answer is often yes. Indeed, if \(T\) is an
unbiased estimator, then,
\[{\mathop{\mathrm{MSE}}}_{\theta}(T) = {\mathop{\mathrm{Var}}}_\theta(T) = {\mathop{\mathrm{bias}}}_{\theta}(T)^2 = {\mathop{\mathrm{Var}}}_\theta(T),\]
so it suffices to minimise the variance.

\begin{theorem}[Cramér-Rao Lower Bound]
  Suppose \(T = T(X)\) is an unbiased estimator for \(\theta \in \Theta \subseteq \mathbb{R}\) 
  based on \(X = (X_1, \cdots, X_n)\) with joint pdf \(f_\theta(x)\). Then under 
  mild regularity conditions (which is elaborated on in the proof below),
  \[{\mathop{\mathrm{Var}}}_{\theta}(T) \ge \frac{1}{I(\theta)},\]
  where 
  \[I(\theta) = E_\theta\left[\left\{\pdv{\theta} \log f_\theta(X)\right\}^2\right],\]
  and we call \(I(\theta)\) the \textit{Fisher information} of the sample.
\end{theorem}

By computing, we find the Fisher information to equal the following.
\[I(\theta) = -E_\theta\left[\pdv[2]{\theta} \log f_\theta(X) \right].\]
Indeed, \[\begin{split}
  E_\theta\left[\pdv[2]{\theta} \log f_\theta(X) \right] 
    & = E_\theta\left[\pdv{\theta} \frac{f'_\theta(X)}{f_\theta(X)}\right]\\
    & = E_\theta\left[- \frac{f'_\theta(X)}{f^2_\theta(X)}f'_\theta(X) + \frac{f''_\theta(X)}{f_\theta(X)}\right]\\
    & = E_\theta\left[-\left(\pdv{\theta} \log f_\theta(X)\right)^2\right] + E_\theta\left[ \frac{f''_\theta(X)}{f_\theta(X)}\right].   
\end{split}\] So, the result follows as, \[\begin{split}
  E_\theta\left[ \frac{f''_\theta(X)}{f_\theta(X)}\right] 
    & = \int_{x \in A} \frac{f''_\theta(x)}{f_\theta(x)}f_\theta(x) \dd x\\
    & = \int_{x \in A} f''_\theta(x) \dd x = \pdv[2]{\theta} \int_{x \in A} f_\theta(x) \dd x = 0,
\end{split}\] where we denoted \(A\) as the support of \(f_\theta\).
This is a useful identity whenever the second derivative is easy to
compute.

\begin{corollary}
  Suppose \(X_1, \cdots, X_n\) is a random sample. Then, is \(f_\theta^{(1)}\) is 
  the pdf of single observation, then 
  \[I_f(\theta) = n I_{f_\theta^{(1)}}(\theta).\]
\end{corollary}
\proof

Since a random sample is iid. \(f_\theta(x) = \prod f_\theta^{(1)}\) and
so
\[I_{f}(\theta) = -E_\theta\left[\pdv[2]{\theta}\log f_\theta(X) \right] 
    = \sum_{i = 1}^n - E_\theta\left(\pdv[2]{\theta} \log f_\theta^{(1)}(X_i)\right)
    = n I_{f_\theta^{(1)}}(\theta).\] \qed

From this, we can conclude that the Fisher information is proportional
to the sample size.

\begin{example}
  Let us find the Fisher information for the random sample 
  \(X_1, \cdots, X_n \sim \text{Bern}(\theta)\).
  
  By the above corollary, we have \(I_f(\theta) = n I_{f_\theta^{(1)}}(\theta)\).
  So, since the pmf of a Bernoulli random variable is 
  \(f_{\theta}^{(1)}(x) = \theta^x(1 - \theta)^{1 - x}\), we have 
  \[\pdv{\theta} \log f_\theta^{(1)}(x) = \frac{x}{\theta} - \frac{1 - x}{1 - \theta} 
    = \frac{x - \theta}{\theta(1 - \theta)},\]
  hence,
  \[I_{f_\theta^{(1)}}(\theta) = E\left[\left(\frac{x - \theta}{\theta(1 - \theta)}\right)^2\right] 
    = \frac{1}{\theta^2(1 - \theta)^2} \mathop{\mathrm{Var}}(X) = \frac{1}{\theta(1 - \theta)}.\]
  Thus, the Fisher information of the random sample is just 
  \(I_f(\theta) = n / \theta(1 - \theta)\).

  With the Fisher information, we can apply the Cramér-Rao lower bound theorem 
  allowing us to conclude that an unbiased estimator \(T\) for \(\theta\) has variance 
  \(\mathop{\mathrm{Var}}(T) \ge \theta(1 - \theta) / n = \mathop{\mathrm{Var}}(\bar{X})\). This allows us to conclude 
  that the sample mean \(\bar{X}\) minimises the mean square error among unbiased 
  estimators for \(\theta\). 
\end{example}

Let us now prove the Cramér-Rao lower bound theorem.

\proof (Cramér-Rao lower bound theorem). Let us first specify the
regularity conditions for the Cramér-Rao lower bound theorem.

\begin{itemize}
    \item Assume that the set \(A := \text{supp} f_\theta = \{x \in \mathbb{R}^n \mid f_\theta(x) > 0\}\)
      is independent of \(\theta\).
    \item \(\Theta\) is an open interval in \(\mathbb{R}\).
    \item For all \(\theta \in \Theta\) there exists \(\pdv{f_\theta}{\theta}\).
    \item Differentiation and integration commutes (for the specific cases where it 
      is used).
  \end{itemize}

As we saw last year, the space of random variables form an inner product
space with the inner product \[\langle X, Y \rangle = E[XY],\] and so,
the Cauchy-Schwarz inequality applies. That is for all random variables
\(X, Y\) \[[E(XY)]^2 \le E(X^2)E(Y^2).\] So, we have \[\begin{split}
    {\mathop{\mathrm{Var}}}_\theta(T) I_f(\theta) & = E_\theta[(T - E_\theta T)^2] 
      E_\theta\left[\left(\pdv{\theta}\log f_\theta(X)\right)^2\right]\\
      & \ge \left(E_\theta\left[(T - E_\theta(T)) \pdv{\theta}\log f_\theta(X) \right]\right)^2.
    \end{split}\] Thus, it suffices to show that the expectation on the
right hand side evaluates to 1. \[\begin{split}
    E_\theta\left[(T - E_\theta(T)) \pdv{\theta}\log f_\theta(X) \right] & =
      E_\theta\left[(T - E_\theta(T)) \frac{\pdv{\theta}f_\theta(X)}{f_\theta(X)}\right]\\
      & = \int_{x \in A} (T(x) - E_\theta(T))\frac{\pdv{\theta}f_\theta(x)}{f_\theta(X)} f_\theta(x) \dd x\\
      & = \int_{x \in A} T(x) \pdv{\theta}f_\theta(x) \dd x - \int_{x \in A} E_\theta(T) \pdv{\theta}f_\theta(x) \dd x\\
      & = \pdv{\theta} \int_{x \in A} T(x) f_\theta(x) \dd x - E_\theta(T) \pdv{\theta} \int_{x \in A} f_\theta(x) \dd x\\
      & = \pdv{\theta} E_\theta(T) - 0 = \pdv{\theta}\theta = 1.
   \end{split}\] \qed

\hypertarget{asymptotic-properties-of-estimators}{%
\subsection{Asymptotic Properties of
Estimators}\label{asymptotic-properties-of-estimators}}

While the Cramér-Rao lower bound theorem provides us with a lower bound
for the variance for non-biased estimators, as we have previously seen,
it is not always true that there exists an unbiased estimator. So,
rather than giving up, we instead study the estimators as the sample
size becomes large.

As we have seen, evaluating an estimator \(T = T(X_1, \cdots, X_n)\) of
\(\theta\) depends on its \emph{sampling distribution}. From the
sampling distribution, one can possibly find properties about the
estimator such as is bias, mean square error and so on. However, it is
not necessarily true that an estimator has a closed form. Indeed, often
times, the estimator is defined as a solution to some equation.

To simplify this, one often consider \(T_n = T_n(X_1, \cdots, X_n)\) as
a sequence of random variables indexed by \(n \in \mathbb{N}\) and
consider the stochastic convergence of the variables in question. We
recall from last term's probability course, there are three different
notions of convergence for random variables,

\begin{itemize}
  \item convergence in probability;
  \item convergence almost surely (almost everywhere);
  \item convergence in distribution.
\end{itemize}

Let us quickly define them here again.

\begin{definition}[Convergence in Probability]
  Let \((X_n)_{n = 1}^\infty\) be a sequence of random variables. Then, \((X_n)\) 
  converges to the random variable \(X\) in probability if for all \(\epsilon > 0\),
  \[\lim_{n \to \infty} \mathbb{P}(|X_n - X| > \epsilon) = 0.\]
\end{definition}
\begin{definition}[Convergence Almost Surely]
  Let \((X_n)_{n = 1}^\infty\) be a sequence of random variables. Then, \((X_n)\) 
  converges to the random variable \(X\) almost surely if 
  \[\mathbb{P}\left(\lim_{n \to \infty} X_n = X\right) = 1,\]
  where \(\lim_{n \to \infty} X_n = X\) is denoting the event 
  \[\{\omega \in \Omega \mid X_n(\omega) \to X(\omega)\}.\]
  With other words, \(X_n \to X\) almost surely, if the set of points \(\omega\) 
  such that \(X_n(\omega)\) does not converge to \(X(\omega)\) has measure 0.
\end{definition}
\begin{definition}
  Let \((X_n)_{n = 1}^\infty\) be a sequence of random variables. Then, \((X_n)\) 
  converges to the random variable \(X\) with cdf \(F_X\) in distribution if 
  \[\lim_{n \to \infty} \mathbb{P}(X_n \le x) = F_X(x),\]
  for all \(x\) at which \(F_X\) is continuous.
\end{definition}

We also recall the following chain of implications,
\[X_n \to_\text{as} X \implies X_n \to_\text{p} X \implies X_n \to_\text{d} X.\]
If \(X = c\) is a constant, then
\[X_n \to_\text{p} X \iff X_n \to_\text{d} X.\] We apply this notion
onto estimators.

\begin{definition}[Consistency]
  A sequence of estimators \((T_n)_{n = 1}^\infty\) for \(g(\theta)\) is called 
  (weakly) consistent if for all \(\theta \in \Theta\), \(\epsilon > 0\)
  \[\lim_{n \to \infty} \mathbb{P}(|T_n - g(\theta)| > \epsilon) = 0.\]
\end{definition}

While it is possible to prove consistency for certain estimators, it is
often a non-trivial task. Instead, we often look at whether a sequence
of estimators are asymptotically unbiased and prove the a special class
of these estimators are consistent.

\begin{definition}[Asymptotically Unbiased Estimators]
  A sequence of estimators \((T_n)_{n = 1}^\infty\) for \(g(\theta)\) is called 
  asymptotically unbiased if for all \(\theta \in \Theta\), 
  \[E_\theta(T_n) \to g(\theta).\]
\end{definition}

We see that \(E_\theta(T_n)\) is simply a value and this is simply the
convergence for real sequences.

Before moving on to prove results about estimators, let us quickly
recall the Markov inequality.

\begin{prop}
  Let \(X\) be a random variable with \(X(\Omega) \subseteq [0, \infty)\), then 
  for all \(a \in \mathbb{R}\),
  \[\mathbb{P}(|X| \ge a) \le \frac{E(|X|)}{a}.\]
\end{prop}
\proof

See first year notes. \qed

\begin{lemma}[MSE Consistency]
  Let \((T_n)_{n = 1}^\infty\) be asymptotically unbiased for \(g(\theta)\) for 
  all \(\theta \in \Theta\). Then, if \({\mathop{\mathrm{Var}}}_\theta(T_n) \to 0\) as \(n \to \infty\), 
  \(T_n\) is consistent for \(g(\theta)\).
\end{lemma}
\proof

Let \(\epsilon > 0\), then, by Markov's inequality, \[\begin{split}
    \mathbb{P}_\theta(|T_n - g(\theta)| & \ge \epsilon) 
    = \mathbb{P}_\theta((T_n - g(\theta))^2 \ge \epsilon^2)
    \le \frac{1}{\epsilon^2} E_\theta(T_n - g(\theta))^2 \\
    & = \frac{{\mathop{\mathrm{MSE}}}_{\theta}(T_n)}{\epsilon^2} = 
    \frac{1}{\epsilon}({\mathop{\mathrm{Var}}}_\theta(T_n) + (E_\theta(T_n) - g(\theta))^2). 
  \end{split}\] Since the right hand side tends to 0 as
\(n \to \infty\), so is the left hand side. \qed

Thus, to show that a sequence of estimators is consistent, it suffices
to show that it is asymptotically unbiased and its variance tends to 0.

However, while consistency is a nice property for a sequence of
estimators to have, it is a very minimal requirement. So, in order to
derive hypothesis tests and confidence intervals, we also need the
sampling distribution of \(T_n\). As we have seen previously, the sample
mean estimators \(T_n\) for a normal distribution \(N(\theta, 1)\) has
distribution \(T_n \sim N(\theta, 1 / n)\), and so, by centring and
scaling, we have \[\sqrt{n}(T_n - \theta) \sim N(0, 1),\] for all
\(n \ge 1\). This means we can work with the CDF of \(T_n\) allowing us
to easily analyse the behaviours of these estimators. However, this is
in general not the case and in most cases, we cannot derive easily the
distributions of the estimators. Nonetheless, often, we may approximate
their distribution with a normal distribution.

\begin{definition}[Asymptotically Normal]
  A sequence of estimators \(T_n\) for \(\theta \in \mathbb{R}\) is asymptotically 
  normal if, for some \(\sigma^2(\theta)\), 
  \[\sqrt{n}(T_n - \theta) \to N(0, \sigma^2(\theta)),\]
  in distribution.
\end{definition}

From last term, we recall the central limit theorem (CLT).

\begin{theorem}[Central Limit Theorem]
  Let \(Y_1, \cdots, Y_n\) be iid. random variables with \(E(Y_i) = \mu\) and 
  \(\mathop{\mathrm{Var}}(Y_i) = \sigma^2 < \infty\). Then the sequence \(\sqrt{n}(\bar{Y} - \mu)\) 
  converges in distribution to a \(N(0, \sigma^2)\) distribution.
\end{theorem}
\proof

See the \emph{probability for statistics} course. \qed

The central limit theorem allows us to conclude that a large class of
estimators are asymptotically normal. Indeed, sample means and
estimators which can be written as a combination of sample means under
weak conditions are certainly asymptotically normal. However, we would
also consider other estimators.

\begin{lemma}[Slutsky's lemma]
  Let \(X_n, X\) and \(Y_n\) be random variables (or random vectors). If \(X_n \to X\) 
  in distribution and \(Y_n \to c\) in probability for some constant \(c\), then 
  \begin{itemize}
    \item \(X_n + Y_n \to X + c\) in distribution;
    \item \(Y_n X_n \to cX\) in distribution;
    \item \(Y^{-1}_n X_n \to c^{-1} X\) in distribution if \(c \neq 0\).
  \end{itemize}
\end{lemma}
\proof

See the \emph{probability for statistics} course. \qed

Another useful result for determining whether or not a sequence of
estimators are asymptotically normal is the \(\delta\)-method. The
\(\delta\)-method allows us to consider whether or not the
transformation of a asymptotically normal estimator remains
asymptotically normal.

\begin{theorem}[\(\delta\)-Method]
  Suppose that \(T_n\) is an asymptotically normal estimator of \(\theta\) with 
  \[\sqrt{n}(T_n - \theta) \to_\text{d} N(0, \sigma^2(\theta)),\]
  and \(g : \Theta \subseteq \mathbb{R} \to \mathbb{R}\) is a differentiable 
  function with \(g'(\theta) \neq 0\). Then 
  \[\sqrt{n}(g(T_n) - g(\theta)) \to_\text{d} N(0, g'(\theta)^2 \sigma^2(\theta)).\]
\end{theorem}
\proof

Since \(g\) is differentiable,
\[g(T_n) = g(\theta) + g'(\theta)(T_n - \theta) + o((T_n - \theta)^2),\]
where \(R\) is the remainder. So
\[\sqrt{n}(g(T_n) - g(\theta)) = g'\sqrt{n}(\theta)(T_n - \theta) + o((T_n - \theta)^2).\]
Thus, assuming the remainder is negligible, we have
\[\sqrt{n}(g(T_n) - g(\theta)) \to_\text{d} N(0, g'(\theta)^2 \sigma^2(\theta)).\]
\qed

Lastly, a useful result we will often use (perhaps implicitly) is the
continuous mapping theorem. Alike sequential continuity for metric
spaces, the continuous mapping theorem will allow us to preserve
stochastic convergence under continuous mappings.

\begin{theorem}[Continuous Mapping Theorem]
  Let \(g : \mathbb{R}^k \to \mathbb{R}^m\) be continuous at every point of a set 
  \(C\) such that \(\mathbb{P}(X \in C) = 1\). Then if \(X_n \to X\), 
  then \(g(X_n) \to g(X)\) for all three notions of convergence, i.e. 
  convergence in distribution, in probability and almost surely.
\end{theorem}

\hypertarget{maximum-likelihood-estimation}{%
\subsection{Maximum Likelihood
Estimation}\label{maximum-likelihood-estimation}}

We recall from first year the maximum likelihood estimator, that is the
estimator that maximises the probability of observing the given
realisations.

\begin{definition}[Likelihood Function]
  Given the realisation \(\mathbf{x}\) of the random object \(\mathbf{X}\), the 
  likelihood function for \(\theta\) is 
  \[L(\theta) = L(\theta \mid \mathbf{x}) = f_\mathbf{X}(\mathbf{x} \mid \theta).\]
\end{definition}

\begin{definition}[Maximum Likelihood Estimator]
  The maximum likelihood estimator of \(\theta \in \Theta^n\) is an estimator 
  \(\hat{\theta}\) such that 
  \[L(\hat{\theta}) = \sup_{\theta \in \Theta} L(\theta)\]
  where \(L\) is the likelihood function.
\end{definition}

The maximum likelihood estimator is often well defined. However, it is
possible to construct situations in which the MLE does not exist or is
not unique. We also recall that given a strictly increasing function
\(f\), the maximum likelihood estimator can also be obtained by
maximising \(f \circ L\). This is most commonly seen in the
log-likelihood function where we maximise \(\log L\).

Maximum likelihood estimators has some nice properties. In short,
maximum likelihood estimators are functionally invariant, consistent and
asymptotically normal.

\begin{prop}
  If \(g\) is a bijective function and if \(\hat{\theta}\) is a MLE of \(\theta\), 
  then \(\hat{\phi} = g(\hat{\theta})\) is a MLE of \(\phi = g(\theta)\).
\end{prop}
\proof

Let us denote \(\tilde{L}\) for the likelihood function of \(\phi\),
then \(\tilde{L} = L \circ g^{-1}\) and so,
\[\tilde{L}(\hat{\phi}) = L(g^{-1}(\hat{\phi})) = L(g^{-1}(g(\hat{\theta})))
    = L(\hat{\theta}) \ge L(g^{-1}(\phi)) = \tilde{L}(\phi).\] \qed

Suppose we now relax the bijective condition on \(g\). If \(g\) is not
surjective, then there exists \(\phi \in \psi\) such that
\(\phi \notin g(\Theta)\), and so, for these values no model is defined.
If this is the case it does not make sense to speak of the likelihood of
these parameters and so, we define their likelihood to be 0. With that,
we see that the original proposition remains true.

On the other hand, if \(g\) is not injective, then \(\phi\) does not
uniquely identify \(\theta\) and so, there might exists multiply
\(\theta\) such that \(g(\theta) = \phi\). However, by defining the
induced likelihood for all \(\theta\),
\[\tilde{L} : \mathbb{R} \to \mathbb{R} : \phi \mapsto \sup\{L(\theta) \mid g(\theta) = \phi\},\]
we see that the invariance over functions is retained.

\begin{prop}
  Let \(X_1, \cdots\) be iid. observations with pdf \(f_\theta(x)\) where 
  \(\theta \in \Theta\) and \(\Theta\) is an open interval. Furthermore, let 
  \(\theta_0 \in \Theta\) be some parameter. Then under regularity conditions, 
  \begin{itemize}
    \item there exists a consistent sequence \((\hat{\theta}_n)_{n = 1}^\infty\) 
      of maximum likelihood estimators;
    \item if \((\hat{\theta}_n)_{n = 1}^\infty\) is a consistent sequence of MLEs, 
    then 
    \[\sqrt{n}(\hat{\theta}_n - \theta_0) \to_\text{d} N(0, I_f(\theta_0)^{-1}),\]
    where \(I_f(\theta)\) is the Fisher information of a sample of size 1.
  \end{itemize}
\end{prop}

We note that this proposition requires the Fisher information of a
distribution which is often not known in practical situations. So, to
use this result we need to estimate \(I_f(\theta_0)\). In general, this
can be approximated by

\begin{itemize}
  \item \(I_f(\hat{\theta})\);
  \item \(\frac{1}{n}\sum_{i = 1}^n \left(\pdv{\theta} 
    \log(f(x_i \mid \theta))\mid_{\theta = \hat{\theta}}\right)^2\);
  \item \(- \frac{1}{n}\sum_{i = 1}^n \pdv[2]{\theta} 
    \log(f(x_i \mid \theta))\mid_{\theta = \hat{\theta}}\).
\end{itemize}

\proof (Sketch of the existence of consistent MLEs). Let us denote
\(L(\theta) := \prod_{i = 1}^n f_\theta(X_i)\) and
\(S_n(\theta) = \frac{1}{n} \log L(\theta) = \frac{1}{n} \sum_{i = 1}^n \log f_\theta(X_i)\).
Since \(\log\) is strictly increasing, we see that \(\hat{\theta}\)
maximises \(L(\theta)\) if and only if it maximises \(S(\theta)\). Then,
by the weak law of large numbers, given iid. \(Z_1, \cdots, Z_n\) where
\(Z_i = \log f_{\theta}(X_i)\),
\(S_n(\theta) \to E_{\theta_0}(Z_1) = E_{\theta_0}(\log f_{\theta_0}(X_i))\)
in probability. So, \(S_n\) is a consistent estimator for
\(E_{\theta_0}(\log f_{\theta_0}(X_i))\).

Let us now define \(R(\theta) := E_{\theta_0}(\log f_\theta (X_1))\) and
I claim that \(\theta_0\) maximises \(R\). Indeed, by considering
\(z - 1 \ge \log z\) for all \(z \in \mathbb{R}^+\), we have for all
\(\theta\), \[R(\theta) - R(\theta_0) = 
    E_{\theta_0} \left[\log \frac{f_\theta(X_1)}{f_{\theta_0}(X_1)}\right]
    \le E_{\theta_0} \left[\frac{f_\theta(X_1)}{f_{\theta_0}(X_1)} - 1\right]
    = \int \left[\frac{f_\theta(x)}{f_{\theta_0}(x)} - 1\right] f_{\theta_0}(x) \dd x.\]
But this is simply,
\[\int f_{\theta}(x) - f_{\theta_0}(x) \dd x = 1 - 1 = 0,\] and hence,
\(R(\theta) \le R(\theta_0)\) for all \(\theta \in \Theta\).

With this in mind, we have \(S_n(\theta) \to R(\theta)\) pointwise and
\(S_n(\hat{\theta}) \to R(\theta_0)\) in probability. Then, by using
analysis techniques, we may show \(\hat{\theta} \to \theta_0\) in
probability, completing the proof of the first claim. \qed

This particular sketch of the proof (which classically Wald used)
requires that the map \(\theta \to R(\theta)\) to be continuous for some
compact set \(K \subseteq \Theta\) in which for all \(\epsilon > 0\),
\(\mathbb{P}(|\hat{\theta} - \theta_0| > \epsilon, \hat{\theta} \in K) \to 0\).
There is a modern approach in which one shows that
\[\sup_{\theta \in \Theta} |S_n(\theta) - R(\theta)| \to 0\] in
probability. This approach relaxes the condition some what and will be
examined in the third year course \emph{Statistical Theory}.

\newpage

\hypertarget{confidence-regions-hypothesis-testing}{%
\section{Confidence Regions \& Hypothesis
Testing}\label{confidence-regions-hypothesis-testing}}

So far, we have being considering point estimators for single values.
This does not reflect any uncertainty. Indeed, as this estimator is
simply resulted from a random sample, it does not tell us how variable
this estimate would be if we drew another sample. To account for this,
we may, instead of estimating a single value, we provide an interval of
values that contains the true parameter with a certain probability.

As an example, let us recall an example from first years statistics.
Given a random sample \(Y_1, \cdots, Y_n \sim N(\mu, \sigma_0^2)\) with
\(\sigma^2_0\) known, we would like to find the confidence intervale
\(I\) such that \(\mathbb{P}(\mu \in I) = 1 - \alpha\) for some
\(\alpha > 0\), e.g.~\(\alpha = 0.05\). By using the sample mean, we
have \(\bar{Y} = \frac{1}{n} \sum Y_i \sim N(\mu, \sigma_0^2 / n)\). So,
by standardising, we have
\[\frac{\bar{Y} - \mu}{\sigma_0 / \sqrt{n}} \sim N(0, 1),\] and so,
\(c_{\alpha / 2} = \Phi^{-1}(1 - \alpha / 2)\) where \(c_{\alpha / 2}\)
is the value such that,
\[1 - \alpha =  \mathbb{P}\left(- c_{\alpha / 2} < 
  \frac{\bar{Y} - \mu}{\sigma_0 / \sqrt{n}} \sim N(0, 1) < c_{\alpha / 2} \right).\]
Hence, the \(1 - \alpha\) confidence interval of \(\mu\) is a
realisation of the \textbf{random} interval
\[I(\mathbf{Y}) = (\bar{Y} - c_{\alpha / 2} \sigma_0 / \sqrt{n}, 
  \bar{Y} + c_{\alpha / 2} \sigma_0 / \sqrt{n}).\] We note that, this
does not mean that given a realisation of the random interval \(I\),
\(\mathbb{P}(\mu \in I) = 1 - \alpha\). Indeed, if \(I\) is realised,
either \(\mathbb{P}(\mu \in I) = 0\) or \(\mathbb{P}(\mu \in I) = 1\).

\begin{definition}[\(1 - \alpha\) Confidence Interval]
  A \(1 - \alpha\) confidence interval for \(\theta \in \Theta\) is a random 
  interval \(I_\theta\) that contains \(\theta\) with probability \(\ge 1 - \alpha\), 
  that is, 
  \[\mathbb{P}_\theta(\theta \in I) \ge 1 - \alpha.\]
\end{definition}

A confidence interval can be any types of interval including unbounded
ones. Indeed, if the confidence interval is unbounded, we say the
confidence interval is a one-sided confidence interval. An application
of such an confidence interval could be that we would like to measure
the pollutant in drinking water with a maximum percentage. Then, given a
random sample of measurements \(Y_1, \cdots, Y_n\), we would like to
find a confidence interval for such that
\[\mathbb{P}(\theta \le h(Y)) = 1 - \alpha.\] So, the confidence
interval in this case would be in the form \((-\infty, h(y)]\).

\hypertarget{construction-of-confidence-intervals}{%
\subsection{Construction of Confidence
Intervals}\label{construction-of-confidence-intervals}}

\begin{definition}[Pivotal Quantity]
  A pivotal quantity for \(\theta\) is a function \(t(Y, \theta)\) of the data 
  \(\theta\) (and \textbf{not} any over parameters).
\end{definition}

With the pivotal quantity for \(\theta\), \(t(Y, \theta)\), we can find
constants \(a_1, a_2\) such that
\[\mathbb{P}(a_1 \le t(Y, \theta) \le a_2) \ge 1 - \alpha\] since we
know the distribution of \(t\). In many cases, we may rearrange the
terms to give
\[\mathbb{P}(h_1(Y) \le \theta \le h_2(Y)) \ge 1 - \alpha\] where
\([h_1(Y), h_2(Y)]\) is a random interval. This is a \(1 - \alpha\)
confidence interval for \(\theta\).

As an example of a pivotal quantity, suppose we would like to construct
an confidence interval for the random sample
\(Y_1, \cdots, Y_n \sim N(\mu, \sigma^2)\) where both \(\mu\) and
\(\sigma^2\) are unknown. Then, we may define the pivotal quantity
\((Y, \mu) \mapsto \frac{\bar{Y} - \mu}{S / \sqrt{n}}\) where \(S\) is
the sample standard deviation. This pivotal quantity follows the
Student-\(t\) distribution with \(n- 1\) degrees of freedom allowing us
to construct a confidence interval according the the method above.

On the other hand, if we would like to construct an confidence interval
for \(\sigma^2\), we can use the pivotal quantity
\((Y, \sigma^2) \mapsto \frac{1}{\sigma^2} \sum(Y_i - \bar{Y})^2\) which
has \(\chi^2\)-distribution with \(n - 1\) degrees of freedom.

However, we see that these constructions are rather specialised to
normal distributions and without justification, cannot be applied to
other distributions. Nonetheless, as we have discussed asymptotic
behaviours of estimators, in which many estimators are asymptotically
normal, we can use this fact to extend our theory of confidence
intervals.

\begin{definition}[Asymptotic Confidence Interval]
  A sequence of random intervals \(I_n\) is called an asymptotic \(1 - \alpha\) 
  confidence interval for \(\theta\) if 
  \[\lim_{n \to \infty} \mathbb{P}_\theta(\theta \in I_n) \ge 1 - \alpha.\]
\end{definition}

Suppose \(\hat{\sigma}_n\) is consistent for \(\sigma(\theta)\) and
thus, \(\hat{\sigma}_n \to \sigma(\theta)\) in probability for all
\(\theta\). By Slutsky's lemma, we have
\[\sqrt{n} \frac{T_n - \theta}{\hat{\sigma}_n} \to N(0, 1)\] in
distribution. Using the left hand side as the pivotal quantity leads us
to the approximate \emph{confidence limits}
\[T \pm c_{\alpha / 2} \hat{\sigma}_n / \sqrt{n}\] where
\(\Phi(c_{\alpha / 2}) = 1 - \alpha / 2\). In general, the easiest
choice for \(\hat{\sigma}_n\) is simply \(\text{SE}(T_n)\).

Lastly, we might be interested in constructing a confidence region for
more than one parameters.

\begin{definition}[Simultaneous Confidence Interval]
  Suppose \(\theta = (\theta_1, \cdots, \theta_k)^T \in \Theta \subseteq \mathbb{R}^k\) 
  and we have \((L_i(Y), U_i(Y))\) such that for all \(\theta\), 
  \[\mathbb{P}(L_i(Y) < \theta_i < U_i(Y) \mid i = 1, \cdots, k) \ge 1 - \alpha.\]
  Then, \((L_i(y), U_i(y))\) is a \(1 - \alpha\) simultaneous confidence interval 
  for \(\theta_1, \cdots, \theta_k\).
\end{definition}

\begin{theorem}[Bonferroni Correction for Simultaneous Confidence Intervals]
  Suppose \([L_i, U_i]\) is a \(1 - \alpha / k\) confidence interval for 
  \(\theta_i\). Then, \(\prod [L_i, U_i]\) is a \(1 - \alpha\) simultaneous 
  confidence interval for \(\theta = (\theta_1, \cdots, \theta_k)^T\).
\end{theorem}
\proof

\[\mathbb{P}(\theta_i \in [L_i, U_i] \mid i = 1, \cdots, k) = 1 - 
    \mathbb{P}\left(\bigcup_{i = 1}^k \{\theta_i \not\in [L_i, U_i]\}\right)
    \ge 1 - \sum_{i = 1}^k \mathbb{P}(\theta_i \not\in [L_i, U_i]) \ge 1 - \alpha.\]
\qed

We note that the Bonferroni corrections are conservative and it is very
possible that the resulting simultaneous confidence interval has a
higher coverage probability that that is suggested by the Bonferroni
correction. We see this in the example where we attempt to find the
simultaneous confidence interval of two \emph{independent} random
samples \(X_1, \cdots, X_n \sim N(\mu, 1)\) and
\(Y_1, \cdots, Y_n \sim N(\theta, 1)\). Then by the usual method, we
find \(I, J\) the \(1 - \alpha\)-confidence intervals for \(\mu\) and
\(\theta\) respectively. By the Bonferroni correction, \(I \times J\) is
a \(1 - 2\alpha\)-confidence region for \((\mu, \theta)\) while in
actuality, \[\mathbb{P}_{\mu, \theta}((\mu, \theta) \in I \times J) = 
  \mathbb{P}(\mu \in I) \mathbb{P}(\theta \in J) = (1 - \alpha)^2.\]
Choosing \(\alpha = 0.1\), we see that Bonferroni guarantees the
coverage probability to be above \(0.8\) while the actual probability is
\(0.9^2 = 0.81\).

\hypertarget{hypothesis-testing}{%
\subsection{Hypothesis Testing}\label{hypothesis-testing}}

\begin{definition}[Null and Alternative Hypothesis]
  Given a model \(f_\theta\) where \(\theta \in \Theta \subseteq \mathbb{R}^d\), 
  the null-hypothesis \(H_0\) and the alternative hypothesis \(H_1\) are propositions 
  that \(\theta \in \Theta_0\) and \(\theta \in \Theta_1\) respectively for 
  some \(\Theta_0, \Theta_1\) a partition of \(\Theta\) such that 
  \(\Theta_0 \cap \Theta_1 = \varnothing\) and \(\Theta_0 \cup \Theta_1 = \Theta\). 
\end{definition}

The goal of an hypothesis test is to determine whether
\(\theta \in \Theta_0\) or \(\theta \in \Theta_1\), or with other words,
whether to accept \(H_0\) or reject \(H_0\) and hence accept \(H_1\).
This is normally achieved through the observation of a particular subset
of the sample space and we call this sample space for which \(H_0\) is
rejected the rejection region (or critical region).

In some literature, we might find some authors reframe from using the
word \emph{accept} (such as we were told in year one). In practice
however, since we are acting based on the result of these tests, it
makes some practical meaning to say we accept the null-hypothesis
\(H_0\) or we accept the alternative hypothesis \(H_1\).

As the accuracy of the hypothesis tests is arbitrary, it is possible to
make errors. The below table demonstrates the two types of errors.

\begin{center}
  \begin{tabular}{ c | c c }
  & \(\theta \in \Theta_0 (H_0)\) & \(\theta \in \Theta_1 (H_1)\)\\
  \hline
  \(\neg\) reject \(H_0\) & \checkmark & Type II error\\
  reject \(H_0\) & Type I error & \checkmark 
  \end{tabular}
\end{center}

\begin{definition}[Level of a Test]
  A hypothesis test is of level \(\alpha\) for \(0 < \alpha < 1\) if 
  \[\mathbb{P}_\theta(\text{reject } H_0) \le \alpha\]
  for all \(\theta \in \Theta\).
\end{definition}

Usually we choose \(\alpha << 1\) with common values being \(0.05\) and
\(0.01\). However, it is not clear whether or not these values are
optimal for general experiments and often times, \(\alpha\) is chosen to
be much smaller, e.g.~ \(\alpha \sim 10^{-6}\).

\begin{definition}[Power]
  Let \(\Theta\) be a parameter space and \(\Theta_0 \subseteq \Theta\) 
  and \(\Theta_1 = \Theta \setminus \Theta_0\) so \(H_0 : \theta \in \Theta_0\) 
  and \(H_1 : \theta \in \Theta_1\) are null and alternative hypothesis'. Suppose 
  we can constructed some test for this hypothesis, then, the power function is 
  the mapping 
  \[\beta : \Theta \to [0, 1] : \theta \mapsto P_\theta(\text{reject }H_0).\]
\end{definition}

Conceptually, if \(\theta \in \Theta_0\), we would like
\(\beta(\theta)\) to be small while if \(\theta \in \Theta_1\), we would
like \(\beta(\theta)\) to be large.

\begin{definition}[\(p\)-Value]
  The \(p\)-value of a particular hypothesis test is 
  \[p = \sup_{\theta \in \Theta_0}\mathbb{P}_\theta(
    \text{observing something "at least as extreme" as the observation}).\]
  That is, if a test is based on the statistic \(T\) with rejection for 
  large values of \(T\), then 
  \[p = \sup_{\theta \in \Theta_0}\mathbb{P}_\theta(T \ge t)\]
  where \(t\) is the observed value. 

  In any case, we reject \(H_0\) if and only if \(p \le \alpha\) and this results 
  a \(\alpha\)-level test.
\end{definition}

The hypothesis tests are related to confidence intervals in that we can
construct a test from any confidence regions.

Let \(Y\) be the random observation of the experiment and suppose
\(A(Y)\) is the \(1 - \alpha\) confidence region for the parameter
\(\theta \in \Theta\), i.e.
\[\mathbb{P}_\theta(\theta \in A(Y)) \ge 1 - \alpha,\] for all
\(\theta \in \Theta\). Then, by defining \(H_0 : \theta \in \Theta_0\)
and \(H_0 = \theta \not\in \Theta_0\) where \(\Theta_0\) is some subset
of \(\Theta\) with level \(\alpha\) such that we reject \(H_0\) if
\(\Theta_0 \cap A(y) = \varnothing\). In this case, we see that
\[\mathbb{P}_\theta(\text{Type I error}) = \mathbb{P}_\theta(\text{reject } H_0) 
  = \mathbb{P}_\theta(\Theta_0 \cap A(Y) = \varnothing) \le 
  \mathbb{P}_\theta(\theta \not\in A(Y)) \le \alpha.\] The reverse is
also possible -- constructing a confidence region from a hypothesis
test. Suppose that for all \(\theta_0 \in \Theta\), we have a level
\(\alpha\) test \(\phi_{\theta_0}\) for
\(H_0^{\theta_0} : \theta = \theta_0\) and
\(H_1^{\theta_0} : \theta \neq \theta_0\) such that
\[\mathbb{P}_{\theta_0}(\phi_{\theta_0}\text{ reject } H_0) \le \alpha.\]
Then, by defining
\[A := \{\theta_0 \in \Theta \mid \phi_{\theta_0} \text{ does not reject } H_0^{\theta_0}\},\]
we find \(A\) to be a \(1 - \alpha\) confidence region for \(\theta\).
Indeed, for all \(\theta \in \Theta\),
\[\mathbb{P}_\theta(\theta \in A) = \mathbb{P}_\theta(\phi_\theta
  \text{ does reject }) = 1 - P_{\theta}(\phi_\theta \text{ rejects}) 
  \ge 1 - \alpha.\] Through this method, we may construct a test for
multiple parameter test through the use of simultaneous confidence
regions. Indeed, if \(I \times J\) is a \(1 - 2\alpha\) confidence
region for \((\mu, \theta)\), a level \(2\alpha\) test of
\(H_0 : (\mu, \theta) = (\mu_0, \theta_0)\) against
\(H_1 : (\mu, \theta) \neq (\mu_0, \theta_0)\) is given by
\[R = \{(\bar{X}, \bar{Y}) \mid (\mu_0, \theta_0) \not\in I \times J\},\]
where \(E(X_i) = \mu\) and \(E(Y_i) = \theta\).

\end{document}
