% Options for packages loaded elsewhere
\PassOptionsToPackage{unicode}{hyperref}
\PassOptionsToPackage{hyphens}{url}
\PassOptionsToPackage{dvipsnames,svgnames*,x11names*}{xcolor}
%
\documentclass[
]{article}
\usepackage{lmodern}
\usepackage{amssymb,amsmath}
\usepackage{ifxetex,ifluatex}
\ifnum 0\ifxetex 1\fi\ifluatex 1\fi=0 % if pdftex
  \usepackage[T1]{fontenc}
  \usepackage[utf8]{inputenc}
  \usepackage{textcomp} % provide euro and other symbols
\else % if luatex or xetex
  \usepackage{unicode-math}
  \defaultfontfeatures{Scale=MatchLowercase}
  \defaultfontfeatures[\rmfamily]{Ligatures=TeX,Scale=1}
\fi
% Use upquote if available, for straight quotes in verbatim environments
\IfFileExists{upquote.sty}{\usepackage{upquote}}{}
\IfFileExists{microtype.sty}{% use microtype if available
  \usepackage[]{microtype}
  \UseMicrotypeSet[protrusion]{basicmath} % disable protrusion for tt fonts
}{}
\makeatletter
\@ifundefined{KOMAClassName}{% if non-KOMA class
  \IfFileExists{parskip.sty}{%
    \usepackage{parskip}
  }{% else
    \setlength{\parindent}{0pt}
    \setlength{\parskip}{6pt plus 2pt minus 1pt}}
}{% if KOMA class
  \KOMAoptions{parskip=half}}
\makeatother
\usepackage{xcolor}
\IfFileExists{xurl.sty}{\usepackage{xurl}}{} % add URL line breaks if available
\IfFileExists{bookmark.sty}{\usepackage{bookmark}}{\usepackage{hyperref}}
\hypersetup{
  pdftitle={Groups and Rings},
  pdfauthor={Kexing Ying},
  colorlinks=true,
  linkcolor=Maroon,
  filecolor=Maroon,
  citecolor=Blue,
  urlcolor=red,
  pdfcreator={LaTeX via pandoc}}
\urlstyle{same} % disable monospaced font for URLs
\usepackage[margin = 1.5in]{geometry}
\usepackage{graphicx}
\makeatletter
\def\maxwidth{\ifdim\Gin@nat@width>\linewidth\linewidth\else\Gin@nat@width\fi}
\def\maxheight{\ifdim\Gin@nat@height>\textheight\textheight\else\Gin@nat@height\fi}
\makeatother
% Scale images if necessary, so that they will not overflow the page
% margins by default, and it is still possible to overwrite the defaults
% using explicit options in \includegraphics[width, height, ...]{}
\setkeys{Gin}{width=\maxwidth,height=\maxheight,keepaspectratio}
% Set default figure placement to htbp
\makeatletter
\def\fps@figure{htbp}
\makeatother
\setlength{\emergencystretch}{3em} % prevent overfull lines
\providecommand{\tightlist}{%
  \setlength{\itemsep}{0pt}\setlength{\parskip}{0pt}}
\setcounter{secnumdepth}{5}
\usepackage{tikz}
\usepackage[all]{xy}
\usepackage{amsthm}
\usepackage{mathtools}
\usepackage{lipsum}
\usepackage[ruled,vlined]{algorithm2e}
\newtheorem{theorem}{Theorem}
\newtheorem{lemma}{Lemma}
\newtheorem{corollary}{Corollary}[theorem]
\newtheorem*{remark}{Remark}
\newtheorem{definition}{Definition}[theorem]
\newtheorem{proposition}{Proposition}[theorem]
\newcommand{\image}{\mathop{\mathrm{Im}}}

\title{Groups and Rings}
\author{Kexing Ying}
\date{May 15, 2020}

\begin{document}
\maketitle

{
\hypersetup{linkcolor=}
\setcounter{tocdepth}{2}
\tableofcontents
}
\hypertarget{groups}{%
\section{Groups}\label{groups}}

\hypertarget{on-the-first-isomorphism-theorem}{%
\subsection{On the First Isomorphism
Theorem}\label{on-the-first-isomorphism-theorem}}

We recall that if \(G, H\) are groups and \(f : G \to H\) is a group
homomorphism, then the first isomorphism theorem states that
\(G / \ker f \cong \mathop{\mathrm{Im}}f\). This can be visualised with
the following diagram where \(q\) is the surjective map
\(q : G \to G / \ker f : g \mapsto g \ker f\). \[
  \xymatrix @ R=1.2cm @ C=1.2cm {
  G \ar[r]^{q} \ar[rd]_{f} & G / \ker f \ar[d]^{f'} \\
  & \mathop{\mathrm{Im}}f
  }
\]

\begin{theorem}
  Let \(G\) be a group and \(N \triangleleft G\), then there is a 1 to 1 
  correspondence, that is a bijection, between the subgroups of \(G / N\) and 
  the subgroups of \(G\) containing \(N\). The same is true for normal subgroups.
\end{theorem}
\proof

See
\href{https://github.com/ImperialCollegeLondon/group-theory-game/blob/master/src/hom/isomorphism.lean\#L391}{here}.
\qed

This theorem is commonly refereed to as the \emph{correspondence
theorem} or the \emph{fourth isomorphism theorem}.

\hypertarget{rings}{%
\section{Rings}\label{rings}}

\hypertarget{recap}{%
\subsection{Recap}\label{recap}}

We shall omit ring axioms but note that we will in general refer to
rings without the multiplicative identity unless it is prefixed with
\textbf{unital}. Simply put,

\begin{definition}[Unital Ring]
  An \textit{unital ring} is a triplet \((R, +, \times)\) such that \((R, +)\) 
  forms an additive abelian group and \((R, \times)\) forms a multiplicative 
  monoid such that \(\times\) distributes over \(+\).
\end{definition}
\begin{definition}[Ring]
  A \textit{ring} is a unital ring without the necessary condition of the 
  multiplicative identity. 
\end{definition}

Some obvious properties can be deduced right away.

\begin{theorem}
  Let \(R\) be a ring, 
  \begin{itemize}
    \item (Zero annihilates) \(0 x = x 0 = 0\);
    \item (Negation distributes) \(-xy = (-x)y = x(-y)\).
  \end{itemize}
\end{theorem}
\proof

Omitted. \qed

\begin{definition}[Unit]
  Let \(R\) be a ring. We say \(x \in R\) is a unit if and only if it has an 
  multiplicative inverse. We write 
  \(U(R) := \{x \in R \mid \exists x^{-1} \in R, x x^{-1} = 1_R = x^{-1} x\}\).
\end{definition}

\begin{proposition}[Unit Group]
  Let \(R\) be an unital ring, then \(U(R)\) is a multiplicative group and we 
  call it the \textit{unit group}.
\end{proposition}

Furthermore, a lot of obvious definitions common to all algebraic
structures are exactly what they sound like. These include
\textbf{subring}, \textbf{ring homomorphism}, and \textbf{unital ring
homomorphism}.

\begin{theorem}
  Let \(\phi : R \to S\) be an ring homomorphism. Then \(\phi(0_R) =0_S\) and 
  \(\forall x \in R\), \(\phi(-x) = -\phi(x)\). Furthermore, if \(\phi\) is an 
  unital ring homomorphism, then \(\forall x \in U(R)\),
  \(\phi(x) \in U(S)\) and \(\phi(x^{-1}) = \phi(x)^{-1}\).
\end{theorem}
\proof

First two property follows from \(R\) and \(S\) being additive groups
while the last follows from the properties of the unit group. \qed

From this theorem, we see that \(\phi(U(R)) \le U(S)\).

Given an abelian group \((G, +)\), we can construct a trivial ring
structure by extending it with the binary operation
\(\times : G \to G \to G : a, b \mapsto 0_G\). We call this a
\textbf{trivial multiplicative structure}. We call a ring
\textbf{trivial} if it only contains one element, thus \(0 = 1\) if the
ring is unital. In fact, the reverse is also true; an unital ring
contains only one element (so trivial) if \(0 = 1\) as
\(\forall x \in R, x = x \times 1_R = x \times 0_R = 0_R\).

\hypertarget{integral-domains-polynomial-rings}{%
\subsection{Integral Domains \& Polynomial
Rings}\label{integral-domains-polynomial-rings}}

\begin{definition} [Zero divisor]
  Let \(R\) be a ring and \(x \in R\). We say \(x\) is a left zero divisor if 
  there is some \(y \in = R^* = R \setminus\{0_R\}\) such that \(xy = 0_R\). 
  Similar definition for the right zero divisor.
\end{definition}

The ring \(M_2(\mathbb{F})\) has zero divisors for any field
\(\mathbb{F}\) while \(\mathbb{Z} / p\mathbb{Z}\) does not have any zero
divisors for \(p\) a prime. We say a ring \(R\) is an integral domain if
it is a non-trivial commutative unital ring with no zero divisors.

\begin{theorem}
  Let \(R\) be an integral domain. Then \(\forall x \in R^*\),\(y, z \in R\),
  \(xy = xz \implies y = z\).
\end{theorem}
\proof

Fix \(x, y, z\) and suppose \(y \neq z\), then \(y + (-z) \neq 0\) and
so \(x (y + (-z)) \neq 0\) as \(x\) is not a zero divisor. \# \qed

\begin{theorem}
  If \(R\) is a finite integral domain, then it is a field.
\end{theorem}
\proof

We need to show \(U(R) \supseteq R^*\). Let \(a \in R^*\), then by the
previous theorem, the map \(x \mapsto ax\) is injective. As \(R\) is
finite, the map is also surjective. \qed

A similar argument can be used to show that given an integral domain
\(R\) that is a finite vector space over some field \(F\), \(R\) is a
field.

Let \(R\) be a commutative unital ring, we define \[
  R[X] := \left\{ \sum_{i = 0}^n a_i X^i \mid a_i \in R, n \in \mathbb{N} \right\},
\] the set of \(R\)-polynomials. \(R[X]\) forms a commutative ring with
the obvious operations.

The following statements are equivalent:

\begin{enumerate} 
  \item \(R\) is an integral domain;
  \item \(R[X]\) is a integral domain;
  \item for every \(p, q \in R[X]^*\), \(\deg pq = \deg p + \deg q\);
  \item for every \(p \in R[X]^*\), \(p\) has at most \(\deg p\) number of roots.
\end{enumerate}

where \(R\) is a non-trivial commutative unital ring. \proof
\(2 \implies 1\) trivially and \(3 \implies 2\) by contrapositive. We
will now show that \(1 \implies 3\), \(4 \implies 1\) and
\(1 \implies 4\).

Suppose \(R\) is an integral domain, \(p, q \in R[X]^*\) such that
\(p(x) = a_n x^n + a_{n - 1} x^{n - 1} + \cdots + a_0\) and
\(q(x) = b_m x^m + b_{m - 1} x^{m - 1} + \cdots + b_0\) where
\(a_n \neq 0_R \neq b_m\) (so \(\deg p = n\) and \(\deg q = m\)).

So, we have \[(pq)(x) = \left(\sum_{i = 0}^n a_i x^i\right) 
              \left(\sum_{j = 0}^m b_j x^j\right)
            = \sum_{i = 0}^n \sum_{j = 0}^m a_i b_j x^{i + j},\]
i.e.~\(\deg pq \le n + m\). Now, as the coefficient of \(x^{n + m}\) is
\(a_n b_m\), both of which are non-zero, as \(R\) is an integral domain,
\(a_n b_m \neq 0_R\), thus, \(\deg pq = n + m\).

We will prove \(4 \implies 1\) by contrapositive. Suppose \(R\) is not
an integral domain, i.e.~there exist \(a, b \in R^*\) such that
\(ab = 0_R\). Consider the polynomial \(R[X]^* \ni p = x \mapsto ax\).
While \(\deg p = 1\), \(p\) has two roots, \(0_R\) and \(b\)
respectively, contradicting 4.

Lastly, we show \(1 \implies 4\) by induction on the degrees. Let
\(p \in R[X]\), if \(\deg p = 0\) then there exists some \(a \in R^*\),
\(p = x \mapsto a\) which does not have any roots since \(a \neq 0_R\).
Now, suppose \(\deg p = n + 1\) and let \(\lambda\) be a root. Then \[
    p(x) = p(x) - p(\lambda) = \sum_{i = 0}^{n + 1} a_i (x^i - \lambda ^ i)
      = (x - \lambda) \sum_{i = 0}^{n + 1} a_i (x^{n - 1} + \cdots + \lambda x ^{n - 1})
      = (x - \lambda) q(x),
  \] for some \(q \in R[X]\) with degrees less than or equal to \(n\).
Now, by the inductive hypothesis, \(q\) has at most \(n\) roots so let
us define the set \[r := \{ x \mid x = \lambda \vee q(x) = 0_R \}. \] It
is obvious that all elements of \(r\) are roots of \(p\) and
\(\mid r \mid \leq n + 1\) so it suffices to show that these are the
only roots. Let \(\mu \in R \setminus r\), then
\(x - \mu \neq 0_R \neq q(\mu)\) and hence
\(p(\mu) = (x - \mu) q(\mu)) \neq 0_R\) as by assumption, \(R\) is an
integral domain. \qed

A direct corollary of the above, specifically \(1 \iff 2\) means
\(U(R[X]\) is the set of constant polynomials \(p = x \mapsto a \in R\)
where \(a \in U(R)\). This means that \(U(R) \cong U(R[X])\) by the
homomorphism \(i = a \mapsto (x \mapsto a)\).

\begin{definition}[Nilpotent]
  Given some ring \(R\), \(x \in R\), \(x\) is called nilpotent if and only if 
  there exists some \(d \in \mathbb{N}\) such that \(x^d = 0_R\).
\end{definition}

Given some integral domain \(R\), we can construct a field
\(\text{Frac}(R)\). Let \(\text{Frac}(R)\) be the equivalence classes of
\(R \times R^*\) by the relation
\((a, b) \sim (a', b') \iff ab' = a'b\). Then \(\text{Frac}(R)\) is a
field by equipping it with
\[+ = (a, b), (a', b') \mapsto (ab' + a'b, bb'),\] and
\[\times = (a, b), (a', b') \mapsto (aa', bb').\]

By checking using the definition above, with \(\iota\) being an
injective unital homomorphism, we see that for all fields
\(\mathbb{F}\), if there exist some \(\phi\) such that
\(\phi : R \to \mathbb{F}\) is an injective unital ring homomorphism,
then there exists an unique homomorphism \(\psi\) such that the
following diagram commutes. \[
\xymatrix @ R=1.2cm @ C=1.2cm {
 R \ar[r]^{\iota} \ar[rd]_{\phi} & \text{Frac}(R) \ar[d]^{\psi} \\
                                 & \mathbb{F} \\
}
\]

\hypertarget{ideals-quotients}{%
\subsection{Ideals \& Quotients}\label{ideals-quotients}}

\begin{definition}[Ideal]
  Given \(R\) a ring, we say \(I\) is an ideal if it is an additive subgroup of 
  \(R\) and for all \(r \in R\), \(x \in I\), \(xr, rx \in I\). We denote this 
  by \(I \triangleleft R\).
\end{definition}

The relation between a ring and its ideals is similar to that of normal
subgroups and groups. A ring has two trivial ideals, the zero ideal and
itself, so the only ring with less than two ideals is the trivial ring
\(\{0\}\). Also, given some ring homomorphism \(\phi : R \to S\),
\(\ker \phi \triangleleft R\).

By some easy checking, we see that ideals are closed under finite sum
and intersections, i.e.~if \((I_i)_{i = 1}^n\) is a sequence of ideals,
so is \(\sum_{i = 1}^n I_i\), and if \(\mathcal{I}\) is a non-empty
family of ideals, \(\bigcap \mathcal{I}\) is also an ideal. The second
point is important as it allows us to talk about ideals generated by
sets. We write \(\langle r_1, \cdots, r_n \rangle\) for the ideal
generated by \((r_i)_{i = 1}^n \subseteq R\) and \(\langle S \rangle\)
for the ideal generated by the set \(S \subseteq R\).

It is easy to see that; given some ring \(R\), for all
\(S \subseteq R\), \(I \triangleleft R\),
\(S \subseteq I \implies \langle S \rangle \le_{Gp} I\) and
\(1_R \in S \implies \langle S \rangle = R\).

\begin{theorem}
  Let \(R\) be a non-trivial unital commutative ring, then \(R\) is a field if 
  and only if the only ideals in \(R\) are \(\{0_R\}\) and \(R\) itself.
\end{theorem}
\proof

Forward direction follows as \(1_R\) is in any non-trivial ideals, while
the backwards direction follows by considering \(xR = R\) for all
\(x \in R\). \qed

From this we see that that any ring homomorphisms from a field
\(\mathbb{F}\) to a ring \(R\), \(\phi : \mathbb{F} \to R\) is either
\(x \mapsto 0_R\) or injective. With this we can see that the sequence
of rings \[
  \mathbb{Z} \hookrightarrow \mathbb{Q} \hookrightarrow \mathbb{R} 
  \hookrightarrow \mathbb{C},
\] while has forward injective ring homomorphisms with the inclusion map
has only the zero ring homomorphisms backwards.

\begin{theorem}
  Let \(R\) be a unital ring with ideal \(I \triangleleft R\), then 
  \(R / I := \{r + I \mid r \in R\}\) is a ring with the operations 
  \((a + I) \hat{+} (b + I) = (a + b) + I\) and \((a + I)(b + T) = ab + I\).
\end{theorem}
\begin{definition}[Quotient Map]
  Given ring \(R\), and \(I \triangleleft R\), we define the quotient map 
  \(q : R \to R / I : x \mapsto x + I\). \(q\) is a surjective unital ring 
  homomorphism with the kernel \(I\).
\end{definition}

We again meet the first isomorphism theorem this time with regards to
rings.

\begin{theorem}
  Let \(R, S\) be unital rings and \(\phi : R \to S\) a unital ring homomorphism,
  then the map
  \[
    \psi : R / \ker \phi \to S : x + \ker \phi \mapsto \phi(x)
  \]
  is a well-defined injective unital ring homomorphism such that the following 
  diagram commute.
  \[
    \xymatrix @ R=1.2cm @ C=1.2cm {
    R \ar[r]^{q} \ar[rd]_{\phi} & R / \ker \phi \ar[d]^{\psi} \\
    & S }
  \]
\end{theorem}

Note that this is equivalent to \(R / \ker \phi \cong S\) whenever
\(\phi\) is surjective.

Similarly, we also meet the \emph{correspondence theorem} again.

\begin{theorem}
  Let \(R\) be a ring and \(I \triangleleft R\), then the map between the set of 
  ideals greater than \(I\) is order isomorphic to the set of ideals of \(R / I\).
\end{theorem}
\proof

Use the map \[
    \mathcal{Q} : \{I' \triangleleft R \mid I \subseteq I'\} \to 
    \{J \triangleleft R / I\} : I' \mapsto q(I').
  \] \(\mathcal{Q}\) is well-defined as for \(I' \triangleleft R\),
\(I \subseteq I'\), \(q(I') \triangleleft R / I\) since for all
\(a, b \in I'\), \(a + b \in I'\) so
\((a + I) \hat{+} (b + I) \in q(I')\) and for all
\(r \in R, a \in I', ra, ar \in I'\) so
\((r + I)(a + I) = ra + I \in q(I')\) and
\((a + I)(r + I) = ar + I \in q(I')\). \qed

Here is a funny exercise. Suppose there exists a proper non-trivial
ideal \(I\) in \(\mathbb{Z}\) greater than \(\langle p \rangle\) for
some prime \(p\), then, there is some
\(x \in (I), x \notin \langle p \rangle\), so \(\gcd(x, p) = 1\). By
Bezout's lemma, there is some \(a, b \in \mathbb{Z}\) such that
\(1 = ax + bp \in I + \langle p \rangle \subseteq I\), so
\(I = \mathbb{Z}\) \# So by the correspondence theorem,
\(\mathbb{Z} / \langle p \rangle\) has no non-trivial proper ideal. In
fact this can be seen by the fact that
\(Z / \langle p \rangle = \mathbb{F}_p\) which is a field.

\hypertarget{product-structure-on-rings}{%
\subsection{Product Structure on
Rings}\label{product-structure-on-rings}}

Let \((R_i)_{i \in I}\) be a family of unital rings, then there is a
natural product ring structure on the Cartesian product
\(\prod_{i \in I} R_i\) such that
\(U \left(\prod_{i \in I} R_i \right) = \prod_{i \in I} U(R_i)\) and the
projection map \(\pi_j : \prod_{i \in I} R_i \to R_j : x \mapsto x_j\)
is an unital ring homomorphism.

Note that the inclusion map \(\iota_j : R_j \to \prod_{i \in I} R_i\) is
a ring homomorphism not necessarily of an unital one. Also, the product
ring does not preserve integral domain, i.e.~for all \(i \in I\),
\(R_i\) is an integral domain does \textbf{not} imply
\(\prod_{i \in I} R_i\) is also an integral domain.

We call ideals \(I, J \triangleleft R\) \textbf{coprime} if
\(I + J = R\). This terminology is used as
\(\langle x \rangle + \langle y \rangle = \mathbb{Z}\) if and only if
\(x, y\) are coprime in \(\mathbb{Z}\).

\begin{lemma}
  Let \(R\) be a ring and \(I_1, I_2 \triangleleft R\) such that \(I_1, I_2\) 
  are coprime, then for all \(a, b \in R\), 
  \[(a + I_1) + (b + I_2) = R.\]
\end{lemma}
\proof

Let \(r \in R\), then, as \(I_1, I_2\) are coprime, there exists
\(x \in I_1, y \in I_2\) such that \(x + y = r + (-a) + (-b)\) so
\((a + x) + (b + y) = r\). \qed

\begin{theorem}
  Let \(R\) be ring and \(I_1, I_2 \triangleleft R\) such 
  that \(I_1, I_2\) are coprime. Then 
  \[
    R / (I_1 \cap I_2) \cong R / I_1 \times R / I_2. 
  \]
\end{theorem}
\proof

Consider the mapping
\(\psi : R \to R / I_1 \times R / I_2 : r \mapsto (r + I_1, r + I_2)\).
By checking, we find \(\psi\) to be a ring homomorphism with kernel
\(I_1 \cap I_2\). So, it suffices to prove that \(\psi\) is surjective
by the first isomorphism theorem.

Let \((a + I_1, b + I_2) \in R / I_1 \times R / I_2\), then by the
previous lemma, \((a + I_1) + ((-b) + I_2) = R\), so there exist
\(x \in a + I_1, y \in (-b) + I_2\) such that \(x + y = 0_R\),
i.e.~\(x = -y\). Now, as \(x \in a + I_1, y \in (-b) + I_2\), we have
\(x - a \in I_1\) and \(y + b \in I_2\). So, by considering
\(\psi(x) = (x + I_1, x + I_2) = (x + I_1, -y + I_2) =  (a + (x - a) + I_1, b + -(y + b) + I_2) = (a + I_1, b + I_2)\)
by the fact that \(\alpha + I_1 = I_1 \iff \alpha \in I_1\). \qed

The theorem above is normally referred to as the \emph{Chinese remainder
theorem} and we that, by induction, we can easily extend it to any
finite number of ideals that are pairwise coprime, i.e.~

\begin{theorem}
  Let \(R\) be ring and \((I_i)_{i = 1}^n\) be a finite sequence of ideals in 
  \(R\) such that \(I_i, I_j\) are pairwise coprime for \(i \neq j\). Then 
  \[
    \frac {R} {\left(\bigcap_{i = 1}^k I_i\right)} \cong 
    \prod_{i = 0}^n R / I_i. 
  \]
\end{theorem}

\hypertarget{the-ring-structure-of-the-integers}{%
\subsection{The Ring Structure of the
Integers}\label{the-ring-structure-of-the-integers}}

The integers is the typical example that comes into mind when discussing
rings and luckily it has many nice properties.

\begin{definition}[Principle Ideal]
  We call an ideal \(I \triangleleft R\) \textit{principle} if and only if it is 
  generated by one element.
\end{definition}

\begin{theorem}
  Every ideal in \(\mathbb{Z}\) is principle.
\end{theorem}
\proof

It is easy to show that every ideal in \(\mathbb{Z}\) is of the form
\(\langle n \rangle\). \qed

\begin{theorem}
  Suppose \(R\) is a unital ring. Then there is a unique unital ring 
  homomorphism \(\phi : \mathbb{Z} \to R\) such that,
  \[
    \phi(k) = 
      \begin{cases}
        0_R,               & k = 0\\
        \phi(k - 1) + 1_R, & k > 0\\
        - \phi(-k),        & k < 0
      \end{cases}.
  \] 
\end{theorem}

By denoting the above unique ring homomorphism by \(\chi_R\) given any
unital ring \(R\), we have by previous results
\(\ker \chi_R \triangleleft \mathbb{Z}\). Now, as \(\mathbb{Z}\) is
principle, there exists some \(n\), \(\langle n \rangle = \ker \chi_R\).
If we restrict this \(n\) to be non-negative, we find that \(n\) to be
unique as if \(\langle x \rangle = \langle y \rangle\) then \(x \mid y\)
and \(y \mid x\), so \(x = \pm y\).

\begin{definition}[Characteristic]
  Given a unital ring \(R\), the characteristic of \(R\) is the unique 
  \(n \in \mathbb{N}\) such that \(\langle n \rangle = \ker \chi_R\).
\end{definition}

By considering the inclusion map of \(\mathbb{Z}\) to \(\mathbb{Z}\),
\(\mathbb{Q}\), \(\mathbb{R}\), and \(\mathbb{C}\), we find these rings
all have characteristics \(0\).

\begin{lemma}
  Let \(\mathbb{F}\) be a field, \(R\) a ring and 
  \(\phi : \mathbb{F} \to R\) an ring homomorphism. Then there is a induced 
  vector space of \(R\) over \(\mathbb{F}\) using the scalar multiplication 
  \(\times : \mathbb{F} \times R \to R : (f, r) \mapsto \phi(f) r\).
\end{lemma}

From this we can deduce,

\begin{theorem}
  Suppose \(R\) be an integral domain of non-zero characteristic, then \(R\) has 
  prime characteristic \(p\) and is a vector space over \(\mathbb{F}_p\).
\end{theorem}
\proof

The first part of the statement is trivial so it suffices to find some
ring homomorphism from \(\mathbb{F}_p \to R\) which is provided by the
first isomorphism theorem. \qed

\hypertarget{prime-and-maximal-ideals}{%
\subsection{Prime and Maximal Ideals}\label{prime-and-maximal-ideals}}

\begin{definition}[Prime]
  We call an ideal \(I \triangleleft R\) prime if and only if it is proper and 
  for all \(a, b \in R\), \(ab \in I\) implies \(a \in I\) or \(b \in I\).
\end{definition}

Straight away, we see that \({0_R}\) is prime in \(R\) if and only if
\(R\) is an integral domain. Another example of a prime ideal is that,
if \(R\) is an integral domain, then \(\langle X \rangle\) is prime in
\(R[X]\). This is true by considering
\(p \in \langle X \rangle \iff p(0) = 0\), or alternatively, deduced
straight away by the following theorem.

\begin{theorem}
  Let \(R\) be a commutative unital ring and \(I \triangleleft R\) be a proper 
  ideal. Then \(I\) is prime if and only if \(R / I\) is an integral domain.
\end{theorem}
\proof

Straightforward contrapositive both ways. \qed

\begin{definition}[Maximal Ideal]
  Let \(R\) be a ring and \(I \triangleleft R\) is proper, we say \(I\) is 
  maximal if and only if for all \(J \triangleleft R\), 
  \(I \subseteq J \implies I = J\) or \(J = R\).
\end{definition}

\begin{theorem}
  Let \(R\) be a commutative unital ring and \(I \triangleleft R\) be a proper 
  ideal. Then \(I\) is maximal if and only if \(R / I\) is a field.
\end{theorem}
\proof

Follows directly from the fact that \(R / I\) is a field if and only if
it has no proper non-trivial ideals. \qed

Since a field is an integral domain, it follows that every maximal ideal
is also prime.

It is not at all obvious that all rings have a maximal ideal, but this
nice property turns out to be true using \emph{Zorn's lemma}.

\begin{theorem}
  Let \((X, \le)\) be a non-empty poset. If each chain \(C\) in \(X\) has an 
  upper bound in \(X\), then \(X\) has a maximal element.
\end{theorem}

\begin{theorem}
  Every unital ring \(R \neq \{0_R\}\) has a maximal ideal.
\end{theorem}
\proof

Let \(P\) to be the set of proper ideals. Then \(P, \subseteq\) is a
poset by lifting the partial order from sets. Thus, by Zorn's lemma, it
suffices to show that every chain in \(P\) has an upper bound in \(P\).
Let \(C\) be a chain in \(P\), then by checking, we find \(\bigcup C\)
is a element of \(P\) so an upper bound of \(C\) in \(P\). \qed

\begin{definition}[Prime]
  Let \(x \in R\), where \(R\) is a ring. We say \(x\) is \textit{prime} if and 
  only if \(\langle x \rangle\) is a prime ideal in \(R\).
\end{definition}

In a commutative unital ring \(R\), we have
\(\langle x \rangle = \{xr \mid r \in R\}\) for elements of \(x \in R\).
We sometimes write \(xR\) for this ideal. It should be noted that the
unital condition is significant as while \(2\mathbb{Z}\) is commutative,
\(\langle 2 \in 2\mathbb{Z} \rangle \neq \{2r \mid r \in 2\mathbb{Z}\}\)
as the latter does not contain \(2\).

Commutative unital rings have a notion of divisibility. Given
\(a, b \in R\), we say \(a \mid b\) if one of the following equivalent
properties hold,

\begin{itemize}
  \item \(b \in \langle a \rangle\);
  \item \(\langle b \rangle \subseteq \langle a \rangle\);
  \item \(\exists x \in R, b = xa\).
\end{itemize}

\begin{definition}[Irreducible]
  We say \(x \in R^*\) is irreducible if \(\langle x \rangle\) is maximal among 
  the set of proper principle ideals.
\end{definition}

We immediately see that \(1_R\) is not irreducible as it generates the
entire ring so \(\langle 1_R \rangle\) is not proper.

\begin{theorem}
  \(a \in R\) is irreducible if and only if whenever \(x \mid a\), 
  \(\langle x \rangle = \langle a \rangle \veebar \langle x \rangle = R\).
\end{theorem}

\begin{theorem}\label{not_irreducible_iff}
  \(a \in R\) is not irreducible if and only if there exists \(x, y \in R^*\) 
  \(x, y \neq 1_R\) such that \(a = xy\).
\end{theorem}

\begin{lemma}
  Let \(R\) be an integral domain, then
  \begin{itemize}
    \item \(\langle a \rangle = \langle b \rangle\) if and only if there is some 
      \(x \in U(R)\) such that \(a = xb\);
    \item \(a \in R^*\) is irreducible if and only if \(a = xy\) implies 
      \(\langle x \rangle = R\) or \(\langle y \rangle = R\);
    \item \(a \in R^*\) is irreducible if and only if \(a = xy\) implies 
      \(\langle x \rangle = \langle a \rangle\) or 
      \(\langle y \rangle = \langle a \rangle\);
    \item if \(a \in R^*\) is prime, then it is irreducible.
  \end{itemize}
\end{lemma}
\proof

The first part is by following your nose while the rest follows directly
from it. \qed

One common question that is often asked is to show some number to be
irreducible in \(\mathbb{Z}[\theta]\) for some algebraic number
\(\theta\). This type of questions can be approached using a single
method.

Suppose we would like to show that \(2, 3, 1 + \sqrt{-5}\) are
irreducible in \(\mathbb{Z}[\sqrt{-5}]\). We will first define the
function
\(\phi : \mathbb{Z}[\sqrt{-5}] \to \mathbb{Z} : a + b\sqrt{-5} \mapsto a^2 + 5b^2\).
By checking, we find \(\phi\) preserves product and thus, divisibility.
Furthermore, we see that \(\alpha \in \mathbb{Z}[\sqrt{-5}]\) is a unit
if and only if \(\phi(\alpha) = 1\). Now, suppose for a contradiction
\(1 + \sqrt{-5}\) is reducible. Then by theorem
\ref{not_irreducible_iff}, there is some
\(a, b \in \mathbb{Z}[\sqrt{-5}]\) such that \(1 + \sqrt{-5} = ab\), so
Wlog. \(\phi(a) = 2\) and \(\phi(b) = 3\) which is not possible \#. The
similar is true for showing 2 and 3 being irreducible.

\begin{lemma}
  Let \(R\) be a ring, then \(R\) is an integral domain if and only if 
  \(\langle 0_R \rangle\) is prime in \(R\).
\end{lemma}

\begin{theorem}
  Let \(R\) be a non-trivial commutative unital ring such that every proper 
  ideal is prime, then \(R\) is a field.
\end{theorem}
\proof

Let \(r \in R^*\), then Wlog. \(\langle r \rangle \neq R\) so
\(\langle r \rangle\) is prime. Now, consider the ideal generated by
\(r^2\). Trivially, by primeness, \(r \in \langle r^2 \rangle\) so there
exists \(a \in R\), \(r = ar^2 \implies 0_R = ar^2 - r = r(ar - 1)\).
Now, as \(\langle 0_R \rangle\) is prime, \(R\) is an integral domain,
so \(ar = 1\). \qed

\hypertarget{principle-ideal-domain}{%
\subsection{Principle Ideal Domain}\label{principle-ideal-domain}}

\begin{definition}[Principle Ideal Domain]
  We call an integral domain \(R\) to be a \textit{principle ideal domain} if 
  and only if for all \(I \triangleleft R\), \(I\) is principle. We sometimes 
  write \(R\) is a \textit{PID}.
\end{definition}

As every ideals of \(\mathbb{Z}\) is of the form \(\langle k \rangle\)
for some \(k \in \mathbb{Z}\), \(\mathbb{Z}\) is a PID.

\begin{theorem}\label{quotient_principle_has_inv_iff}
  Let \(R\) be a PID and \(x \in R^*\). Then \(x\) is irreducible if and only if 
  \(R / \langle x \rangle\) is a field. Furthermore, any non-zero prime ideal is 
  maximal.
\end{theorem}
\proof

Follows directly from the fact that \(R / I\) is a field if and only if
\(I\) is maximal for any \(I \triangleleft R\). \qed

A powerful result of the above theorem is that we have just classified
the finite fields \(\mathbb{F}_p\).
\(\mathbb{F}_p = \mathbb{Z} / \langle p \rangle\) is a field if and only
if \(p\) is prime (in the ideal sense as well as in the integer sense).

\begin{theorem}
  Let \(\mathbb{F}\) be a field. Then \(\mathbb{F}[X]\) is a principle ideal 
  domain.
\end{theorem}
\proof

Let \(I \triangleleft \mathbb{F}[X]\) and Wlog. suppose \(I\) is proper
and non-trivial. Thus, by the well-ordering principle, there is some
\(p \in I\) with minimal degree \(d_p\). For contradiction, suppose
\(I \neq \langle p \rangle\), then there is some
\(q \in I \setminus \langle p \rangle\) with minimal degree \(d_q\). By
construction, we have \(d_p \le d_q\) so
\(r(X) := q(X) - c p(X) X^{d_q - d_p} \in I\), where
\(c = c_q c_p^{-1}\) and \(c_f\) is the coefficient of \(f\) of the term
\(X^{\deg f}\). We see that, by construction, \(\deg r < \deg q\) so, by
the minimum degree assumption of \(q\), \(r \in \langle p \rangle\)
implying \(q \in \langle p \rangle\). \# \qed

While the proof above is neat, it turns out that polynomial over fields
forms what it's called a \emph{Euclidean Domain} which are principle
ideal domains. We will come back to this definition later.

The reverse of the above theorem is also true.

\begin{theorem}
  If \(R[X]\) is a PID, then \(R\) is a field.
\end{theorem}
\proof

As \(\langle X \rangle\) is irreducible in \(R[X]\), we find that
\(R[X] / \langle X \rangle\) is a field by theorem
\ref{quotient_principle_has_inv_iff}. Now, as
\(R[X] / \langle X \rangle \cong R\) by considering the first
isomorphism theorem and the ring homomorphism that maps each polynomial
to its constant coefficient, we find that \(R\) is a field. \qed

\begin{theorem}
  Let \(S\) denote the set of maximal ideals of some ring \(R\), then 
  \[ \bigcup S = R \setminus U(R). \]
\end{theorem}

\hypertarget{fields-and-adjunction-of-elements}{%
\subsection{Fields and Adjunction of
Elements}\label{fields-and-adjunction-of-elements}}

We say a field \(\mathbb{F}\) is a \emph{subfield} of a field
\(\mathbb{K}\) or that \(\mathbb{K}\) is a \emph{field extension} of
\(\mathbb{F}\) if and only if \(\mathbb{F}\) is a unital subring of
\(\mathbb{K}\). If so, then \(\mathbb{K}\) forms a \(\mathbb{F}\)-vector
space and we call its dimension the \emph{degree} of the field
extension, this is denoted by
\(\left| \mathbb{K} : \mathbb{F} \right|\).

\emph{We will come back to this later.}

\hypertarget{more-on-polynomial-rings.}{%
\subsection{More on Polynomial Rings.}\label{more-on-polynomial-rings.}}

Suppose \(\phi : R \to S\) is a unital ring homomorphism between two
integral domains, then the mapping \[
  \hat{\phi} : R[X] \to S[X] : \sum_{i = 0}^n r_i X^i 
  \mapsto \sum_{i = 0}^n \phi(r_i) X^i
\] is also a unital ring homomorphism. This can be used to examine
irreducibility in \(S[X]\) and \(R[X]\) through each other.

\begin{definition}[Primitive]
  We call \(f \in \mathbb{Z}[X]\) \textit{primitive} if and only if there is no 
  prime \(p\) dividing all of the coefficients of \(f\).
\end{definition}

\begin{theorem}
  A non-constant polynomial \(f \in \mathbb{Z}[X]\) is irreducible in 
  \(\mathbb{Z}[X]\) if and only if it is primitive and irreducible in 
  \(\mathbb{Q}[X]\).
\end{theorem}

\end{document}
