% Options for packages loaded elsewhere
\PassOptionsToPackage{unicode}{hyperref}
\PassOptionsToPackage{hyphens}{url}
\PassOptionsToPackage{dvipsnames,svgnames*,x11names*}{xcolor}
%
\documentclass[
]{article}
\usepackage{lmodern}
\usepackage{amssymb,amsmath}
\usepackage{ifxetex,ifluatex}
\ifnum 0\ifxetex 1\fi\ifluatex 1\fi=0 % if pdftex
  \usepackage[T1]{fontenc}
  \usepackage[utf8]{inputenc}
  \usepackage{textcomp} % provide euro and other symbols
\else % if luatex or xetex
  \usepackage{unicode-math}
  \defaultfontfeatures{Scale=MatchLowercase}
  \defaultfontfeatures[\rmfamily]{Ligatures=TeX,Scale=1}
\fi
% Use upquote if available, for straight quotes in verbatim environments
\IfFileExists{upquote.sty}{\usepackage{upquote}}{}
\IfFileExists{microtype.sty}{% use microtype if available
  \usepackage[]{microtype}
  \UseMicrotypeSet[protrusion]{basicmath} % disable protrusion for tt fonts
}{}
\makeatletter
\@ifundefined{KOMAClassName}{% if non-KOMA class
  \IfFileExists{parskip.sty}{%
    \usepackage{parskip}
  }{% else
    \setlength{\parindent}{0pt}
    \setlength{\parskip}{6pt plus 2pt minus 1pt}}
}{% if KOMA class
  \KOMAoptions{parskip=half}}
\makeatother
\usepackage{xcolor}
\IfFileExists{xurl.sty}{\usepackage{xurl}}{} % add URL line breaks if available
\IfFileExists{bookmark.sty}{\usepackage{bookmark}}{\usepackage{hyperref}}
\hypersetup{
  pdftitle={Differentiable Equations},
  pdfauthor={Kexing Ying},
  colorlinks=true,
  linkcolor=Maroon,
  filecolor=Maroon,
  citecolor=Blue,
  urlcolor=red,
  pdfcreator={LaTeX via pandoc}}
\urlstyle{same} % disable monospaced font for URLs
\usepackage[margin = 1.5in]{geometry}
\usepackage{graphicx}
\makeatletter
\def\maxwidth{\ifdim\Gin@nat@width>\linewidth\linewidth\else\Gin@nat@width\fi}
\def\maxheight{\ifdim\Gin@nat@height>\textheight\textheight\else\Gin@nat@height\fi}
\makeatother
% Scale images if necessary, so that they will not overflow the page
% margins by default, and it is still possible to overwrite the defaults
% using explicit options in \includegraphics[width, height, ...]{}
\setkeys{Gin}{width=\maxwidth,height=\maxheight,keepaspectratio}
% Set default figure placement to htbp
\makeatletter
\def\fps@figure{htbp}
\makeatother
\setlength{\emergencystretch}{3em} % prevent overfull lines
\providecommand{\tightlist}{%
  \setlength{\itemsep}{0pt}\setlength{\parskip}{0pt}}
\setcounter{secnumdepth}{5}
\usepackage{tikz}
\usepackage{amsthm}
\usepackage{mathtools}
\usepackage{lipsum}
\usepackage[ruled,vlined]{algorithm2e}
\usepackage{physics}
\theoremstyle{definition}
\newtheorem{theorem}{Theorem}
\newtheorem{prop}{Proposition}
\newtheorem{corollary}{Corollary}[theorem]
\newtheorem*{remark}{Remark}
\theoremstyle{definition}
\newtheorem{definition}{Definition}[section]
\newtheorem{lemma}{Lemma}[section]
\newcommand{\diag}{\mathop{\mathrm{diag}}}
\newcommand{\Arg}{\mathop{\mathrm{Arg}}}
\newcommand{\hess}{\mathop{\mathrm{Hess}}}

\title{Differentiable Equations}
\author{Kexing Ying}
\date{January 11, 2021}

\begin{document}
\maketitle

{
\hypersetup{linkcolor=}
\setcounter{tocdepth}{2}
\tableofcontents
}
\newpage

\hypertarget{introduction}{%
\section{Introduction}\label{introduction}}

While we have seen differential equations in year one, we have mostly
focused on the different methods of solving specific differential
equations. This cannot be expected for general differential equations
and in this year, we will focus on existence and uniqueness of solutions
to differential equations and develop qualitative tools to help us
understand these solutions.

We recall that an algebraic equation is an equation of the form
\(f(x) = 0\) while a differential equation is an equation of the form
\(\dot x = f(x)\) for some function \(f : \mathbb{R} \to \mathbb{R}\).
That is, an algebraic equation has real numbers as solutions while an
differential equation has functions as its solution.

As an example, let us consider the simple differentiable equation
\begin{equation}\label{simple}
  \dot x = a x,
\end{equation} for some \(a \in \mathbb{R}\). Then, a function
\(\lambda : I \to \mathbb{R}\) solves \ref{simple} if
\(\dot \lambda = a \lambda\) for all \(t \in I\) where
\(I \subseteq \mathbb{R}\) is a interval. These types of differentiable
equations occurs often in relation in growth and decay and one can
easily see that the family of functions
\[\lambda_b : \mathbb{R} \to \mathbb{R} = t \mapsto b e^{at}, \hspace{2mm} b \in \mathbb{R},\]
are solutions to \ref{simple}. Of course, we know this already, so an
more interesting question would be whether or not this family contains
all the solutions to \ref{simple}. It turns out to be true, and to show
this we will assume \(\mu : I \to \mathbb{R}\) is a solution to
\(\dot x = a x\). Then,
\[\dv{t}\left(\mu e^{-at}\right) = \dot \mu e^{-at} - a\mu e^{-at} = 0,\]
since \(\dot \mu = a \mu\) and so, \(\mu e^{-at}\) is constant,
i.e.~there exists \(b \in \mathbb{R}\) such that \(\mu e^{-at} = b\) and
hence, \[\mu = b e^{at}.\] This demonstrates that all solutions to
\ref{simple} are members of the aforementioned solution family and
hence, we have found \textbf{all} of the solutions to \ref{simple}.

With the above example, we see that rather than working with solutions
that are in finite-dimensional vector spaces, our solution are in
function spaces which are typically infinite-dimensional. This is
studied in more detail in the next year's \emph{functional analysis}
course, and in general, infinite-dimensional spaces are more difficult
to grasp. However, for the vast majority of materials in this course, a
finite-dimensional thinking suffices while we will also cover some
material from functional analysis to understand the differentiable
equations as well.

\hypertarget{ordinary-differential-equations-and-initial-value-problems}{%
\subsection{Ordinary Differential Equations and Initial Value
Problems}\label{ordinary-differential-equations-and-initial-value-problems}}

There are two types of differential equations -- \emph{autonomous
differential equations} and \emph{nonautonomous differentiable
equations}. Autonomous differential equations are differentiable
equations of the form \(\dot x = f(x)\) such as equation \ref{simple}
while nonautonomous differential equations are equations of the form
\(\dot x = f(t, x)\).

We note that this does not cover higher-order differential equations,
but from last year, we recall that one may reduce a higher-order
differential equations into a first-order differential equation in
vector form and thus, the theories we develop within this course will
also apply to higher-order differential equations.

\begin{definition}[Ordinary Differential Equation]
  Let \(d \in \mathbb{N}\), \(D \subseteq \mathbb{R} \times \mathbb{R}^d\) be 
  open, and a function \(f \to D \to \mathbb{R}^d\). Then, an equation of the form 
  \[\dot x = f(t, x)\]
  is called a \(d\)-dimensional (first-order) ordinary differential equation. 

  A differentiable function \(\lambda : I \to \mathbb{R}^d\) on some interval 
  \(I \subseteq \mathbb{R}\) is called a solution of the differential equation 
  if and only if for all \((t, \lambda(t) \in D\), if \(t \in I\) then, 
  \[\dot \lambda(t) = f(t, \lambda(t)).\]
  We say that an ordinary differential equation is autonomous if \(f\) is 
  independent of \(t\) and nonautonomous otherwise.
\end{definition}

We will only consider ordinary differential equations (ODE) in this
course while partial differential equations, that is differential
equations which solutions are functions which depends on multiple
variables are covered in the second year course \textbf{Partial
Differential Equations in Action}.

\begin{prop}[Constant solutions to autonomous differential euqations]
  Let \(D \subseteq \mathbb{R}^d\) be an open set and \(f : D \to \mathbb{R}^d\) 
  be a function where \(d \in \mathbb{N}\). Then, there exists a constant 
  solution \(\lambda : \mathbb{R} \to \mathbb{R}^d : x \mapsto a\) to the 
  autonomous differential
  \[\dot x = f(x)\]
  for some \(a \in \mathbb{R}^d\) if and only if \(f(a) = 0\).
\end{prop}
\proof

(\(\implies\)) Suppose that
\(\lambda : I \to \mathbb{R}^d : x \mapsto a\) is a solution the
\(\dot x = f(x)\). Then
\[0 = \dot \lambda (t) = f(\lambda (t)) = f(a).\] (\(\impliedby\))
Suppose there exists some \(a \in \mathbb{R}^d\) such that \(f(a) = 0\),
then verifying, we find
\(\lambda : \mathbb{R} \to \mathbb{R}^d : x \mapsto a\) is a solution to
the differential equation. \qed

This proposition allows us to find solutions to many autonomous ODEs as,
indeed, if \(f : D \to \mathbb{R}^d\) has a root \(a \in \mathbb{R}^d\),
the above proposition guarantees that
\(\lambda : \mathbb{R} \to \mathbb{R}^d : x \mapsto a\) is a solution to
\(\dot x = f(x)\).

\begin{definition}[Initial Value Problem]
  Let \(d \in \mathbb{N}\), \(D \subseteq \mathbb{R} \times \mathbb{R}^d\) be an 
  open set, and \(f : D \to \mathbb{R}^d\) be a function. The system of equations 
  from combining the differential equation 
  \[\dot x = f(t, x),\]
  with the initial condition
  \[x(t_0) = x_0\] 
  where \((t_0, x_0) \in D\) is called an initial value problem.

  A solution to the above initial value problem is a function 
  \(\lambda : I \to \mathbb{R}^d\) that is a solution to the differential equation 
  \(\dot x = f(t, x\) and \(\lambda(t_0) = x_0\).
\end{definition}

\end{document}
